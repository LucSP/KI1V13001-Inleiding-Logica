\subsection*{Inhoud \& leerdoelen}

Wat maakt redeneren \emph{logisch}?
Het gebruikelijke antwoord is dat een inferentie strikt logisch of \emph{geldig} is dan en slechts dan wanneer gegarandeerd \emph{waarheid behouden} wordt: als de assumpties waar zijn, dan is de conclusie dat ook.
Het doel van logica is om steekhoudende inferenties te formaliseren.
Het doel van logica is om een theorie te formuleren over geldige gevolgtrekkingen en redeneringen. 

In deze cursus leer je over de basisconcepten, methoden en resultaten van moderne logica.
Heden ten dage is logica een hoogwiskundige discipline; hierom komt ook de nodige wiskunde aan bod in deze cursus.

Na deze cursus kun je de geldigheid van een breed scala aan inferenties verifiëren of onderuit halen gebruikmakend van de wiskundige methoden die in deze cursus behandeld worden.
Ook zul je beschikken over de nodige kennis en kunde om je eigen redeneersystemen te ontwikkelen.

\subsection*{Blackboard en MS Teams}
\label{sec:ms-teams}

We zullen gebruikmaken van zowel Blackboard (\url{https://students.uu.nl/blackboard}) als MS Teams \begin{center}
(\url{https://manuals.uu.nl/nl/handleiding/teams-online-onderwijs-student/})
\end{center}

Op Blackboard ga je het lesmateriaal en (belangrijke) mededelingen vinden.
Op MS Teams zal de live interactie plaats gaan vinden.

Iedereen die meedoet aan de cursus heeft automatisch toegang tot de Blackboardomgeving.
Inschrijven voor de MS Teams omgeving kan middels de volgende code:
\begin{center}
  \textbf{wq62pmh}
\end{center}
Hoe dit inschrijven werkt, vind je hier:
\begin{center}
  \url{https://tinyurl.com/yxj3xonb}
\end{center}

\subsection*{Cursusformat 2020}

Vanwege de huidige coronapandemie zal de cursus dit jaar iets anders verlopen dan voorheen:

\begin{itemize}
  \item
    Er zijn geen live hoorcolleges.
    Deze worden vervangen door zelfstudie van de lecture notes gecombineerd met zogenoemde kennisclips: korte video's over de kernonderwerpen van de cursus.
    Deze worden op maandagen op MS Teams gezet en je kunt deze op een zelfgekozen tijdstip bekijken.

  \item Voor intercatie met mij (de docent) zal er elke maandag van 15:00 tot 16:00 een Q\& A zijn op MS Teams over de stof van de week daarvoor. De eerste Q\& A zal plaatsvinden op maandag 7 september en betreft de organisatie van deze cursus.
  De Q\& A's zullen plaatsvinden in het ''General'' kanaal. %To give you the chance to interact with me as lecturer, there will be weekly Q\&A's on Mondays, 15:00-16:00 in MS Teams.
 %   The Q\&A on Monday will always cover the material from the week before.
%    The first Q\&A will cover the syllabus and course organization and takes place on Monday September 7.
%    The Q\&A meetings always take place in the ``General'' channel of our course Team.
  \item %Workgroups will take place in what UU calls ``bimodal'' form.
  Werkgroepen vinden in, zoals de UU dat gedoopt heeft, ``bimodale'' vorm plaats. Dit betekent dat slechts een (klein) deel van de groep op locatie aanwezig kan zijn. De rest kan via MS Teams de werkgroep digitaal bijwonen. Je TA (zie onderstaande) zal je hier verder over informeren.
%    This means that only a small part of the group will be on-location, while the rest joins digitally via MS Teams.
%    Your TA (see below) will be in touch with you about organizing this.
\end{itemize}

Vanwege de ongewone situatie zijn er, ondanks de maanden voorbereiding, nog een hoop vraagtekens en onbekende factoren. We zijn allemaal bezig het digitaal lesgeven ons eigen te maken en te optimaliseren. Mochten er onverhoopt dingen niet soepel lopen, wees dan alsjeblieft geduldig. Wij ontvangen echter graag feedback over dingen die niet (of juist heel goed) werken, zodat we door kunnen gaan met verbeteren.

Meer informatie over het onderwijs in de eerste twee blokken vind je hier:
\begin{center}
\url{https://students.uu.nl/nieuws/informatie-over-onderwijs-blok-1-en-2}
\end{center}
%This is going to be an unusual block.
%Despite months of preparations, there are still a lot of unknowns and we're continuously working on making this digital teaching work.
%Please be patient if things are not running as smooth as I can promise you we'd like them to but please also do give us feedback on things that don't work,
%this will be the only way in which we can improve.

%You can find more information about teaching in blocks 1 and 2 here:
%\begin{center}
%\url{https://students.uu.nl/nieuws/informatie-over-onderwijs-blok-1-en-2}
%\end{center}

\subsection*{Taal}
\label{sec:language}

De officiële taal van deze cursus is Nederlands. De meeste communicatie zal dan ook in het Nederlands zijn. De cursusmaterialen zullen echter veelal nog in het Engels zijn. Hier wordt nog aan gewerkt. De lecture notes daarentegen zijn in het Engels (en zullen dat ook blijven), aangezien de taal van de wetenschap betreffende logica en wiskunde voornamelijk Engels is. Vertaalhulp zal op verscheidene manieren beschikbaar zijn. 
%The official course language is Dutch.
%Most communication will be in Dutch but much course material (including this syllabus) still needs to be updated/translated.
%The lecture notes are and will remain in English,
%which is the language of scientific discourse in logic and mathematics.
%Translation help will be provided in a variety of ways.
 
\subsection*{Cursusmateriaal}

Het cursusmateriaal bestaat uit de lecture notes, de slides en Kevin Houstons boek \emph{How to Think Like a Mathematician} (CUP 2009).% The course material consists of the lecture notes, the slides, and Kevin Houston's \emph{How to Think Like a Mathematician} (CUP 2009).

Elke week zal worden aangegeven welke lectuur bij de stof van die week hoort. Er wordt verwacht dat je het materiaal bestudeerd hebt voor de bijbehorende Q\& A sessies op maandag.
%Each week there are assigned readings from the lecture notes and Houston's book.
%I expect you to have studied the material before coming to the Q\&A on Monday.

Daarnaast zal ik jullie nog wijzen op ander (extra) leesmateriaal. Dit is niet verplicht, maar kan wel helpen voor een (breder) begrip van de stof.  

%Additionally, I will provide pointers to supplementary readings. These are not mandatory, but they might help you understand the material better.

De lecture notes kun je vinden op Blackboard. Daarop staan ook de slides die vorig jaar bij de (toen nog normale) hoorcolleges gebruikt zijn. De slides voor de kennisclips zullen wekelijks online gezet worden.
%The lecture notes (see below) can be found on Blackboard.
%I've also provided the slides from last year, when I was still lecturing in the normal style.
%The slides for the kennisclips will be provided on a weekly basis.

\subsubsection*{Lecture Notes}

De lecture notes zijn precies wat je van de naam kunt verwachten. Erin vind je wat ik in de betreffende colleges (lectures) had willen vertellen. Het is dus geen studieboek zoals je dat kent, maar er zijn wel voldoende overeenkomsten.

%The lecture notes are just that: notes for my lectures.
%They are what I want to tell you in that particular lecture.
%This means they are not a textbook in the usual sense, though they are similar to one in some respects.

Ik heb de lecture notes geschreven toen ik de cursus zoals deze nu is voor het eerst gaf. Ik update ze nog steeds om er eventuele fouten uit te halen en de stof beter tot zijn recht te laten komen. Hierbij is alle hulp welkom (op de universiteit is het normaal om \emph{samen} met de docent aan het materiaal te werken). Om dit te bewerkstelligen gebruik ik github met de .tex broncode voor de lecture notes:
\begin{center}
  \url{https://github.com/jkorb/logic-introduction}
\end{center}

Als \LaTeX of git je (nu nog) niets zegt, dan is dat geen probleem. Als je hierover wil leren kan dat hier:

\begin{itemize}
  \item \url{https://www.overleaf.com/learn/latex/Free_online_introduction_to_LaTeX_(part_1)}

  \item \url{https://guides.github.com/}
\end{itemize}

Dit is zeker niet verplicht, maar (vooral \LaTeX) zul je later (veel) moeten gebruiken.

Ook kun je opmerkingen of commentaar via email naar je TA of naar mij sturen.

%I have written the notes while giving the course like this for the first time and I am continuously updating them to fix mistakes, typos, etc. and to improve the presentation and order of the material.
%Here I would very much like your input (a distinguishing feature of university education is that students and teacher are working \emph{together} on the material).
%For that purpose, I've created a github repository with the .tex source code for the notes here:
%\begin{center}
%  \url{https://github.com/jkorb/logic-introduction}
%\end{center}

%If \LaTeX and git don't mean anything to you, that's OK.
%You can start learning about them here (if you want to):
%\begin{itemize}
%  \item \url{https://www.overleaf.com/learn/latex/Free_online_introduction_to_LaTeX_(part_1)}

%  \item \url{https://guides.github.com/}
%\end{itemize}
%Note that that's purely optional (but you probably will have to learn these things at some point anyways ;) ).

%You can also just send me comments via email (until I've found a better way of organizing this).

\subsection*{Exercises and Assignments}

``{\it The best way to learn logic is to do exercises. A lot of exercises.}''
\begin{flushright}
Any logic professor, ever.
\end{flushright}

Per college\footnote{Hoewel er dit jaar niet echt hoorcolleges zijn, vragen we wel dat je twee hoofdstukken per week tot je neemt.} wordt er één hoofdstuk uit de lecture notes behandeld. Aan het einde van elk hoofdstuk vind je bijbehorende opgaven. Oplossingen voor een deel van die oefeningen kun je vinden in de appendix van de lecture notes. De oefeningen waar een $[h]$ bij staat, zijn huiswerkopgaven.

Als je je huiswerk op tijd inlevert, kan je TA (opbouwende) feedback\footnote{Voor het concept opbouwende feedback, zie  \url{https://en.wikipedia.org/wiki/Formative_assessment}.} teruggeven. Je TA zal je laten weten hoe en wanneer je het huiswerk kan inleveren.

We adviseren jullie met klem gebruik te maken van deze mogelijkheid.
Het is de best mogelijke voorbereiding op het tentamen. De huisweropdrachten worden niet becijferd.


Op zowel Blackboard als in je rooster (\url{https://mytimetable.uu.nl}) kun je zien in welke werkgroep je zit.
Een overzicht van welke TA bij welke werkgroep hoort vind je hier:
\begin{center}
\begin{tabular}{| c | l | l |}
\hline
  \textbf{WG} & \textbf{TA} & \textbf{Email}\\\hline
  1& Guus Kooij & \texttt{g.kooij2@uu.nl}\\
  2 & Bence Tijssen & \texttt{b.d.tijssen@uu.nl}\\
  3& Marieke Gelderblom& \texttt{m.h.gelderblom@students.uu.nl}\\
  4 & Sander van Rossum & \texttt{s.vanrossum@uu.nl}\\
  5 & Margaretha Don & \texttt{m.j.don@students.uu.nl}\\
  6 & Walter van Rijen & \texttt{w.j.r.vanrijen@uu.nl}\\
  7& Jos Zuijderwijk & \texttt{a.j.h.zuijderwijk@students.uu.nl}\\
  8 & Timo Diedering & \texttt{t.w.diedering@students.uu.nl}\\
  9 & David Bikker & \texttt{d.m.bikker@uu.nl}\\
  10& Wouter Vromen & \texttt{w.vromen@students.uu.nl}\\\hline
\end{tabular}
\end{center}

Oplossingen voor de huiswerkopgaven worden online gezet de week nadat de opgaven behandeld zijn (rond de tijd van de Q\& A). 

\subsection*{Becijfering}

Er zijn bij deze cursus twee examenmomenten:
Een digitale tussentoets en een eindtoets die volgens een glashelder format wordt afgenomen: met pen en papier. Beide tentamens zullen \emph{op locatie} worden afgenomen. De locatie kun je terugvinden in je rooster.

%There are two exam moments in the course: the mid-term exam (which will be digital) and the end-term exam (which will be a classical pen-and-paper exam), both \emph{on campus}.

\emph{Opmerking:} Dit alles is onder voorbehoud. Er kan afgeweken worden van de planning als dat nodig blijkt in verband met de huidige situatie omtrent corona.
%\emph{Remark}: This is assuming that everything will go as planned, which given the situation is not guaranteed.

Cijfers worden gegeven van 1 (niet best) tot en met 10 (heel best). Je voorlopige eindcijfer wordt als volgt berekend:

\begin{itemize}
    \item 50\% tussentoets
    \item 50\% eindtoets
\end{itemize}

%You're graded on a scale from 1 (worst) -- 10 (best). Your preliminary grade will be calculated as follows:

%\begin{itemize}

%\item 50\% mid-term exam

%\item 50\% end-term exam

%\end{itemize}

Merk op dat je geen cijfer krijgt voor de huiswerkopdrachten. Als je voorlopige eindcijfer een 5,5 of hoger is, dan heb je de cursus gehaald en word je voorlopige eindcijfer je definitieve eindcijfer (geen reparatiemogelijkheden).

Als je voorlopige eindcijfer tussen de 4,0 (inclusief) en 5,5 (exclusief) is, kom je in aanmerking voor een hertoets.
Mocht je de hertoets dan niet doen, dan haal je de cursus niet en dan word je voorlopige eindcijfer je definitieve eindcijfer.
Als je ervoor kiest de hertoets wel te maken, dan vervangt de hertoets ofwel de tussentoets, ofwel de eindtoets.
Je kan zelf kiezen welke je doet.
De hertoets zal gaan over de stof van de toets die je laat vervangen (de tussentoets beslaat propositielogica, de eindtoets eerste-orde logica).
Je definitieve eindcijfer wordt dan berekend op dezelfde manier als je voorlopige eindcijfer: 50\% de hertoets en 50\% de toets die je niet hebt laten vervangen.
Als je definitieve eindcijfer dan 5,5 of hoger is, heb je de cursus gehaald.
Als je definitieve eindcijfer een 5,4 of lager is, heb je de cursus niet gehaald (zonder verdere mogelijkheid tot reparatie).  

Als je voorlopige eindcijfer een 3,9 of lager is, heb je de cursus niet gehaald en kom je ook niet in aanmerking voor de hertoets.
Je voorlopige eindcijfer wordt dan ook je definitieve eindcijfer.

De afronding van de cijfers gebeurt volgens de regels
(zie ook \href{http://students.uu.nl/gw/ki/praktische-zaken/regelingen-en-procedures}{\texttt{http://students.uu.nl/gw/ki/praktische-zaken/regelingen-en-procedures}}).

\subsection*{Vragen over de administratie}

Voor vragen omtrent de administratie (wisselen van werkgroep, afwezigheid door ziekte, extra tijd voor tentamens, \dots), neem dan contact op met Dr. De (\texttt{m.j.de@uu.nl}).

\subsection*{Werklast en zelfstudie}

Dit is een cursus van \textbf{7.5 ECTS}.
Aangezien één ECTS voor \textbf{28 uur} werk staat, betekent dit dat je ongeveer \textbf{210 uur} in deze cursus moet steken om hem succesvol af te sluiten.

Er zijn 17 hoorcolleges\footnote{%
  In het huidige format vervangen door kennisclips}
en een gelijk aantal werkgroepen vam twee uur elk.
Dit somt op tot \textbf{68 contacturen}.
Met de zes uur die staat voor beide tentamens bij elkaar, komen we op een totaal van 74 uur.
Dit geeft ruimte voor \textbf{136 uur zelfstudie}.
Naar mijn ervaring heb je ongeveer de tijd van een hoorcollege nodig om een hoorcollege voor te bereiden (lecture notes bestuderen, de stof begrijpen).
Hetzelfde geldt voor de werkgroepen (huiswerk maken, de stof tot je nemen).
Dat betekent dat er nog \textbf{68 uur over is voor zelfstudie}.
Wederom adviseren we je met klem om deze tijd ook te gebruiken.
Je kunt niet verwachten de cursus te halen zonder er zelf meer tijd in te steken.
Omdat elke student op een eigen manier werkt, kan ik geen algemeen advies geven over hoe je deze uren moet inrichten.
Ik kan wel suggesties doen: je kunt meer opgaven maken (uit de lecture notes of je vraagt je TA naar meer opgaven), je kunt het extra materiaal lezen, nogmaals door de slides gaan, de lecture notes herlezen \dots.

\subsection*{Rooster}

De hoofdstukken uit de lecture notes die horen bij de hoorcolleges staan aangegeven tussen haakjes.
We behandelen twee hoofdstukken per week en bij tijd en wijle ook een enkel hoofdstuk uit het boek van Houston.

\begin{description}

  \item[Week 1.]
    Geen college, geen werkgroepen.

   \emph{Kennisclips}: K0

  \item[Week 2.]
    College 1, Introduction (1), Houston ch. 2 and 3.

   \emph{Kennisclips}: K1

  \item[\phantom{Week 2.}]
    College 2, Mathematics Primer (2), Houston ch. 14-18.

    \emph{Kennisclips}: K2, X3--7

  \item[Week 3.]
    College 3, Set Theory (3), Houston ch. 1 + 20--21.

    \emph{Kennisclips}: K3--5. X1--2, X8--10

  \item[\phantom{Week 2.}]
    College 4, Syntax for propositional Logic (4), Houston ch. 6--9

    \emph{Kennisclips}: K7--9, Timo+Wouter 1--2

  \item[Week 4.]
    College 5, Semantics for propositional logic (5), Houston ch. 13

    \emph{Kennisclips}: K10--12(13), Timo+Wouter 3--4

  \item[\phantom{Week 2.}]
    College 6, Proof theory for propositional logic (6), Houston ch. 4-5.

   \emph{Kennisclips}: K13

  \item[Week 5.]
    Herhaling + Q\&A

  \item[Tussentoets.]
    {\bf  01-10, 17.00--20.00, HUIS}

  \item[Week 6.]
    College 7, Soundness and completeness for propositional logic (7), Houston ch. 32--35.

    \emph{Kennisclips}: K14--15

  \item[\phantom{Week 6}]
    College 8, Syntax for First-Order Logic (8), Houston ch. 10--12.

    \emph{Kennisclips}: K20, Timo+Wouter: 5--6

  \item[Week 7]
    College 9, Semantics for First-Order Logic (9)a, reread Houston ch. 1--5.

    \emph{Kennisclips}: K16--17, Timo+Wouter: 7-8

  \item[\phantom{Week 6}]
    College 10, Proof Theory for First-Order Logic (10)

    \emph{Kennisclips}: K18--19

  \item[Week 8]
    College 11, Soundness and Completeness for First-Order Logic (11)

    \emph{Kennisclips}: K21

  \item[\phantom{Week 6}]
    College 12, Conclusion: Outlook, Applications, Limitations, etc. (12)

    \emph{Kennisclips}: K22

  \item[Week 9.]
    Herhaling + Q\&A

  \item[Eindtoets.]
    {\bf 29-10, 13.30--16.30, HUIS}

  \item[Herkansing.]
    {\bf 19-11, 13.30--16.30, HUIS}

\end{description}

\section*{Beleid}

\begin{description}

  \item[Regelingen en Procedures.]
    Voor de algemene regelingen en procedures, zie:
    \href{http://students.uu.nl/gw/ki/praktische-zaken/regelingen-en-procedures}{\texttt{http://students.uu.nl/gw/ki/praktische-zaken/regelingen-en-procedures}}

  \item[Laatkombeleid.]
    Voor je eigen bestwil en voor dat van je medestudenten, kom je niet te laat en ga je niet voortijdig weg.
    Als de omstandigheden je geen andere keuze laten, vraag ik je om zo geruisloos mogelijk de lokalen binnen te komen of weg te gaan.
    Ook in verband met de huidige coronamaatregelen, is het van belang om de eerste beschikbare plek in te nemen die je tegenkomt.


  \item[Vragenbeleid.]
    Vragen stellen kan (en het wordt ook aangemoedigd om dat te doen!) bij de werkgroepen en bij de Q\& A's.
    Belangrijke vragen die niet kunnen wachten, kan je ook over de mail stellen.

  \item[Inclusiebeleid.]
    Iedereen is welkom. Elke vorm van discriminatie wordt niet getolereerd.

  \item[Coronaregelingen.]
    In het licht van de huidige situatie omtrent het nieuwe coronavirus, gelden er enkele regels in de gebouwen.
    De student die als eerste de zaal binnen komt, loopt zo ver mogelijk door naar achteren, de zaal vult zich zo op tot de voorste stoel ook bezet is.
    Het verlaten van de zaal gaat volgens een last-in-first-outprincipe.
    De student die als eerste naar binnen gaat, verlaat de zaal dus als laatste.
    Zorg ervoor dat je anderhalve meter van andere studenten en van je werkgroepbegeleider houdt.
    Indien jij of een van je naasten klachten hebt/heeft \emph{blijf je thuis}! Laat dit bij voorkeur wel z.s.m. weten aan je werkgroepdocent.

\end{description}
    
\end{document}
