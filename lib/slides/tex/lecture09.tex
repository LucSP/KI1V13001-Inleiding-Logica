\setcounter{framenumber}{264}
\begin{frame}
  \LectureNo{9}
  \maketitle
\end{frame}

\begin{frame}{Overview}
\tableofcontents
\end{frame}

\section{Rehash}
\begin{frame}{Rehash}
	
\begin{itemize}
	
		\item In f.o. logic, we take the grammatical structure of simple sentences into account.
		
		\item  \alert{Singular terms stand for objects, predicates for properties/relations.}
		
		\item \alert{Quantifiers allow us talk about things in generality.}
		
		\item \alert{The signature consists of a set of constants, of function symbols, and of predicate symbols each.}
		
		\item Both terms and formulas are defined inductively in f.o.r logic.
		
		\item \alert{We have induction for formulas and terms and function recursion.}
		
		\item Parsing trees are defined for terms and functions.
		
		\item Nodes in trees have a canonical way of being named: by the directions for how to find them starting at the root. 
		
		\item \alert{Variables are bound by quantifiers.}
		
		\item \alert{A formula without free variables is closed, a sentence.}
		
		\item Just like in propositional logic, there are guidelines for formalization.
					
	\end{itemize}

\end{frame}
		

\section{9. Semantics for First-Order Logic}
\subsection{9.1 Truth, Models, and Assignments}

\begin{frame}{9.1 Truth, Models, and Assignments}

	\begin{itemize}
		
		\item (9.1.1) Our aim is to define models for first-language.
		
		\item (9.1.2) Generally speaking, a model interprets the non-logical vocabulary:
		
		\begin{itemize}
			
			\item in propositional logic the sentence letters
			
			\item in first-order logic the signature
		
		\end{itemize}
		
		\item (9.1.3) We'll first go through the idea of how to build models and then develop these ideas formally:
		
		\begin{itemize}
		
			\item The simplest sentences have subject-predicate form:
			
			\item $P(a)$ formalizes ``the ball is red''
		
		\end{itemize}
		
	\end{itemize}

\end{frame}

\begin{frame}{Towards Models}

	\begin{itemize}
	
		\item (9.1.3--4) Names denote objects, unary predicates express properties:
		
		\begin{itemize}
		
			\item $a$ denotes the ball
				
			\item $P$ expresses $\{x:\text{x is red}\}$
		
		\end{itemize}
		
		\item $P(a)$ is true iff the ball is a member of $\{x:\text{x is red}\}$ 
		
		\item (9.1.4--5) Generally, in a model $\mathcal{M}$,
	
			\begin{itemize}
			
				\item $t$ denotes $\llbracket t\rrbracket^\mathcal{M}$
				
				\item $R$ expresses $R^\mathcal{M}$
				
				\item $R(t_1, \mathellipsis, t_n)$ is true in $\mathcal{M}$ iff $(\llbracket t_1\rrbracket^\mathcal{M}, \mathellipsis, \llbracket t_n\rrbracket^\mathcal{M})\in R^\mathcal{M}$
				
				\item (9.1.6) $t_1=t_2$ is true in $\mathcal{M}$ iff $\llbracket t_1\rrbracket^\mathcal{M}=\llbracket t_2\rrbracket^\mathcal{M}$
			
			\end{itemize}

	\end{itemize}

\end{frame}

\begin{frame}{Denotations of Terms}

	(9.1.7) Different terms have different denotations:
		
			\begin{itemize}
			
				\item Constants denote fixed objects: 
				
				\begin{itemize}
				
					\item $a\in\mathcal{M}$ denotes $a^\mathcal{M}$
				
				\end{itemize}
				
				\item Function symbols express functions: $f^n\in\mathcal{F}$ expresses an $n$-ary function: 
				
				\begin{itemize}
				
					\item Let $f^2$ stand for ``the LCA of \dots and \underline{\phantom{\dots}}''
					
					\item $f^\mathcal{M}$ is a function that (intuitively) maps two individuals to their LCA 
					
					\item If $a$ stands for ``Ada Lovelace,'' $b$ for ``Alan Turing,'' $c$ for ``Angela Merkel,'' then ``the LCA of Ada Lovelace and the LCA of Alan Turing and Angela Merkel,'' is \[f(a,f(b,c))\]
					
					\item Denotations of function terms can be calculated recursively:
					\begin{align*}
	\llbracket f(a,f(b,c))\rrbracket^\mathcal{M}&=f^\mathcal{M}(a^\mathcal{M}, \llbracket f(b,c)\rrbracket^\mathcal{M})\\
	&=f^\mathcal{M}(a^\mathcal{M}, f(b^\mathcal{M}, c^\mathcal{M}))
	\end{align*}
				
				\end{itemize}
			
			\end{itemize}

\end{frame}

\begin{frame}{And the Variables?}

	\begin{itemize}
	
		\item (9.1.8) Variables are pronouns: $x,y,z,\mathellipsis$ are like ``he,'' ``she,'' ``it''
		
		\item ``It is red'' becomes $P(x)$
		
		\item To know where ``it is red'' is true we need to know what ``it'' stands for.
		
		\item In natural language, this information comes from the context:
		
		\begin{itemize}
		
			\item Yesterday, I bought a ball. It is red.
			
			\item My favorite pen is lying there on the table. It's red.
		
		\end{itemize}
		
		\item Formally, this role will be played by \emph{assignments}, functions $\alpha$ from the variables, $x$, to objects, $\alpha(x)$:
		
		\begin{itemize}
		
			\item $\llbracket x\rrbracket^\mathcal{M}_\alpha=\alpha(x)$
		
		\end{itemize}
	
	\item We've interpreted the signature (and a bit more).
	
	\end{itemize}


\end{frame}

\begin{frame}{Truth}

	\begin{itemize}
		
		\item (9.1.9) Truth, then, is recursively defined relative to a model and context:
		
		\begin{itemize}
		
			 \item { $\llbracket R(t_1, \mathellipsis, t_n)\rrbracket^\mathcal{M}_\alpha=1$ iff $(\llbracket t_1\rrbracket^\mathcal{M}_\alpha, \mathellipsis, \llbracket t_n\rrbracket^\mathcal{M}_\alpha)\in R^\mathcal{M}$}
				
			\item {$\llbracket t_1=t_2\rrbracket^\mathcal{M}_\alpha=1$  iff $\llbracket t_1\rrbracket^\mathcal{M}_\alpha=\llbracket t_2\rrbracket^\mathcal{M}_\alpha$}
			
			\item We can use the truth-functions to interpret the connectives: 
			\begin{itemize}
			
			\item $\llbracket \neg\phi\rrbracket^\mathcal{M}_\alpha=f_\neg(\llbracket\phi\rrbracket^\mathcal{M}_\alpha)$	
			
			\item $\llbracket\phi\circ\psi\rrbracket^\mathcal{M}_\alpha=f_\circ(\llbracket\phi\rrbracket^\mathcal{M}_\alpha,\llbracket\psi\rrbracket^\mathcal{M}_\alpha)$
			
			\end{itemize}
					
			\item But what about the quantifiers?
			\[\llbracket \exists x\phi\rrbracket^\mathcal{M}_\alpha=???\]\[\llbracket \forall x\phi\rrbracket^\mathcal{M}_\alpha=???\]
		
		\end{itemize}
		
		\item Idea: Change values in $\alpha$, for $d$ an object, $\alpha[x\mapsto y]$ is the result of changing the value of $x$ to $d$
		
		\begin{itemize}
		
			\item $\forall x\phi$ is true in $\mathcal{M}$ under $\alpha$ iff for every possible value $d$ of $x$, $\phi$ is true under $\alpha[x\mapsto d]$ 
			\item $\exists x\phi$ is true in $\mathcal{M}$ under $\alpha$ iff for somme value $d$ of $x$, $\phi$ is true under $\alpha[x\mapsto d]$ 
		
		\end{itemize}
		
	\end{itemize}

\end{frame}

\begin{frame}{Domains}

	\begin{itemize}
	
		\item (9.1.10) We don't want to consider \emph{all} possible values:
		
		\begin{itemize}
		
			\item ``Everybody passes'' is $\forall xP(x)$
			
			\item The quantifier ``ranges'' over students.
		
		\end{itemize}
		
		\item A model $\mathcal{M}$ comes with a fixed domain $D^\mathcal{M}$
		
		\item There are also technical reasons for why the domain needs to be fixed: paradoxes of universal sets.
			
	\end{itemize}

\end{frame}

\begin{frame}{Ideas in One Place}

	\begin{itemize}
	
		\item A model interprets the signature by assigning denotation to every constant, a function to every function symbol, and an $n$-ary relation to every $n$-ary relation symbol
		
		\item An assignment in a model tells us what the variables denote. It plays the role of the context in natural language. 
	
		\item We can recursively calculate the denotation of arbitrary terms in a model under an assignment.
		
		\item We can recursively calculate the truth-value of a formula relative to a model under an assignment.
	
	\end{itemize}


\end{frame}

\subsection{9.2 Models and Assignments}

\begin{frame}{9.2 Models and Assignments}

	\begin{itemize}
		
		\item \alert{(9.2.1) A \emph{model} for $\mathcal{S}$ is a structure $\mathcal{M}=(D^\mathcal{M},\cdot^\mathcal{M})$, s.t.:
		\begin{enumerate}[(i)]
		
			\item $D^\mathcal{M}$ is a non-empty (!) set, the \emph{domain} of $\mathcal{M}$
			
			\item $\cdot^\mathcal{M}$ is an \emph{interpretation function}, which assigns to:
			\begin{enumerate}[(a)]
			
				\item every constant $c\in\mathcal{M}$ an element $c^\mathcal{M}\in D^\mathcal{M}$ of the domain
				
				\item every function symbol $f^n\in\mathcal{F}$, a function $f^\mathcal{M}:D^n\to D$
				
				\item every predicate $R^n\in\mathcal{R}$, a set $R^\mathcal{M}\subseteq D^n$.
			
			\end{enumerate}
		\end{enumerate}}
			
		\item (9.2.4) An \emph{assignment} in a model $\mathcal{M}=(D^\mathcal{M},\cdot^\mathcal{M})$ of signature $\mathcal{S}$ is a function $\alpha:\mathcal{V}\to D^\mathcal{M}$. 
		
		\item (9.2.6) We define the \emph{denotation} $\llbracket t\rrbracket_\alpha^\mathcal{M}$ of a term $t$ in a model $\mathcal{M}$ under assignment $\alpha$ by the following recursion:
		
			\begin{enumerate}[(a)]
			
				\item		\begin{enumerate}[(i)]

					\item $\llbracket x\rrbracket_\alpha^\mathcal{M}=\alpha(x)$
					\item $\llbracket c\rrbracket_\alpha^\mathcal{M}=c^\mathcal{M}$
				
				\end{enumerate}
				
				\item $\llbracket f(t_1,\mathellipsis,t_n)\rrbracket_\alpha^\mathcal{M}=f^\mathcal{M}(\llbracket t_1\rrbracket_\alpha^\mathcal{M}, \mathellipsis, \llbracket t_n\rrbracket_\alpha^\mathcal{M})$
			
			\end{enumerate}
			
			\item (9.2.8) We $\alpha[x\mapsto d]$, the result of setting the value of variable $x\in\mathcal{V}$ to $d\in D^\mathcal{M}$, by:
		\[\alpha[x\mapsto d](y)=\begin{cases} y &\text{if }y\neq x\\ d & \text{if }y=x\end{cases}\]		

	
	\end{itemize}

\end{frame}

\begin{frame}{Examples (9.2.2)}

Models of signature $\mathcal{S}_{PA}=(\{0\}, \{S^1, +^2, \cdot^2\}, \emptyset)$:
				
				\begin{enumerate}[(a)]
				
					\item The standard, intended model:
					
						\begin{itemize}
					
							\item $D^\mathcal{M}=\mathbb{N}$
							
							\item $0^\mathcal{M}=0$
					
							\item $S^\mathcal{M}(n)=n+1$
							
							\item $+^\mathcal{M}(n,m)=n+m$
							
							\item $\cdot^\mathcal{M}(n,m)=n\cdot m$
					
						\end{itemize}
						
					\item A natural, but non-intended model on the even numbers:
					
						\begin{itemize}
					
							\item $D^\mathcal{M}=\{n\in\mathbb{N}:n\text{ is even}\}$
																					\item $0^\mathcal{M}=0$
					
							\item $S^\mathcal{M}(n)=n+2$
							
							\item $+^\mathcal{M}(n,m)=n+m$
							
							\item $\cdot^\mathcal{M}(n,m)=n\cdot m$
					
						\end{itemize}
\end{enumerate}

\end{frame}

\begin{frame}{Examples (9.2.2)}

Models of signature $\mathcal{S}_{PA}=(\{0\}, \{S^1, +^2, \cdot^2\}, \emptyset)$:
				
				\begin{enumerate}[(a)]
				
				\setcounter{enumi}{2}


					\item A natural, but non-intended model on the odd numbers:
					
						\begin{itemize}
					
							\item $D^\mathcal{M}=\{n\in\mathbb{N}:n\text{ is odd}\}$
																					\item $0^\mathcal{M}=1$
					
							\item $S^\mathcal{M}(n)=n+2$
							
							\item $+^\mathcal{M}(n,m)=\begin{cases}n+m&\text{if }n+m\text{ is odd}\\n+m+1&\text{ otherwise}\end{cases}$
							
							\item $\cdot^\mathcal{M}(n,m)=n\cdot m$
					
						\end{itemize}
					
			\end{enumerate}
			
\vspace{2ex}			
			
With signature $\mathcal{S}_\emptyset=(\emptyset,\emptyset,\emptyset)$, literally every set is a model!

\vspace{2ex}
				
For signature $\mathcal{S}=(\{a,b,c\}, \{f^1, g^2\}, \{P^1, R^2\})$:
							
						
						\begin{itemize}
						
							\item $D^\mathcal{M}=\{1,2,3,4\}$
							
							\item $a^\mathcal{M}=1, b^\mathcal{M}=3, c^\mathcal{M}=2$
						
							\item $f^\mathcal{M}(x)=x$ and $g^\mathcal{M}(x,y)=min(x,y)$
							
							\item $P^\mathcal{M}=\{1,3\}$ and $R^\mathcal{M}=\{(1,1), (1,2),(2,2) (2,3), (3,3)\}$
						
						\end{itemize}
						
						
					
	
	
\end{frame}

\begin{frame}{9.2 Models and Assignments}

	\begin{itemize}
		
		\item {(9.2.1) A \emph{model} for $\mathcal{S}$ is a structure $\mathcal{M}=(D^\mathcal{M},\cdot^\mathcal{M})$, s.t.:
		\begin{enumerate}[(i)]
		
			\item $D^\mathcal{M}$ is a non-empty (!) set, the \emph{domain} of $\mathcal{M}$
			
			\item $\cdot^\mathcal{M}$ is an \emph{interpretation function}, which assigns to:
			\begin{enumerate}[(a)]
			
				\item every constant $c\in\mathcal{M}$ an element $c^\mathcal{M}\in D^\mathcal{M}$ of the domain
				
				\item every function symbol $f^n\in\mathcal{F}$, a function $f^\mathcal{M}:D^n\to D$
				
				\item every predicate $R^n\in\mathcal{R}$, a set $R^\mathcal{M}\subseteq D^n$.
			
			\end{enumerate}
		\end{enumerate}}
			
		\item \alert{(9.2.4) An \emph{assignment} in a model $\mathcal{M}=(D^\mathcal{M},\cdot^\mathcal{M})$ of signature $\mathcal{S}$ is a function $\alpha:\mathcal{V}\to D^\mathcal{M}$.}
		
		\item (9.2.6) We define the \emph{denotation} $\llbracket t\rrbracket_\alpha^\mathcal{M}$ of a term $t$ in a model $\mathcal{M}$ under assignment $\alpha$ by the following recursion:
		
			\begin{enumerate}[(a)]
			
				\item		\begin{enumerate}[(i)]

					\item $\llbracket x\rrbracket_\alpha^\mathcal{M}=\alpha(x)$
					\item $\llbracket c\rrbracket_\alpha^\mathcal{M}=c^\mathcal{M}$
				
				\end{enumerate}
				
				\item $\llbracket f(t_1,\mathellipsis,t_n)\rrbracket_\alpha^\mathcal{M}=f^\mathcal{M}(\llbracket t_1\rrbracket_\alpha^\mathcal{M}, \mathellipsis, \llbracket t_n\rrbracket_\alpha^\mathcal{M})$
			
			\end{enumerate}
			
			\item (9.2.8) We $\alpha[x\mapsto d]$, the result of setting the value of variable $x\in\mathcal{V}$ to $d\in D^\mathcal{M}$, by:
		\[\alpha[x\mapsto d](y)=\begin{cases} y &\text{if }y\neq x\\ d & \text{if }y=x\end{cases}\]		

	
	\end{itemize}

\end{frame}

\begin{frame}{Examples (9.2.5)}

\begin{itemize}

	\item With assignments only the domain counts:
	
	\begin{enumerate}[(i)]
		
			\item Models with domain $D^\mathcal{M}=\mathbb{N}$ (9.2.2.i.a--d):
			
				\begin{enumerate}[(a)]
				
					\item $\alpha(x_1)=0, \alpha(y)=1, \alpha(z)=2$
					
					\item $\alpha(x)=1, \alpha(y)=0, \alpha(z)=3$
					
					\item $\alpha(x)=0, \alpha(y)=0, \alpha(z)=0$
					
					\item For $\mathcal{V}=\{x_i:i\in\mathbb{N}\}$, $\alpha(x_i)=i$.
					\item $\alpha(x)=1$ for all $x\in\mathcal{V}$.
									
				\end{enumerate}
				
			\item Models with domain $D^\mathcal{M}=\mathbb{N}\cup\wp(\mathbb{N})$ (9.2.2.iii.a):

					\begin{enumerate}[(a)]

						\item $\alpha(x)=2, \alpha(y)=\{2\}, \alpha(z)=\mathbb{N}$
						
						\item $\alpha(x)=\{x\in\mathbb{N}:n\text{ is even}\}, \alpha(y)=\{n\in\mathbb{N}:x\text{ is odd}\}, \alpha(x)=\{n\in\mathbb{N}:x\text{ is prime}\}$
						
						\item $\alpha(x)=0$ for all $x\in\mathcal{V}$.

						\end{enumerate}

			\item For $D^\mathcal{M}=\{\ast\}$ (9.2.2.i.e) there is only \emph{one} assignment, which is the constant assignment $\alpha(x)=\ast$ for all $x\in\mathcal{V}$

		\end{enumerate}

\end{itemize}

\end{frame}
\begin{frame}{9.2 Models and Assignments}

	\begin{itemize}
		
		\item {(9.2.1) A \emph{model} for $\mathcal{S}$ is a structure $\mathcal{M}=(D^\mathcal{M},\cdot^\mathcal{M})$, s.t.:
		\begin{enumerate}[(i)]
		
			\item $D^\mathcal{M}$ is a non-empty (!) set, the \emph{domain} of $\mathcal{M}$
			
			\item $\cdot^\mathcal{M}$ is an \emph{interpretation function}, which assigns to:
			\begin{enumerate}[(a)]
			
				\item every constant $c\in\mathcal{M}$ an element $c^\mathcal{M}\in D^\mathcal{M}$ of the domain
				
				\item every function symbol $f^n\in\mathcal{F}$, a function $f^\mathcal{M}:D^n\to D$
				
				\item every predicate $R^n\in\mathcal{R}$, a set $R^\mathcal{M}\subseteq D^n$.
			
			\end{enumerate}
		\end{enumerate}}
			
		\item {(9.2.4) An \emph{assignment} in a model $\mathcal{M}=(D^\mathcal{M},\cdot^\mathcal{M})$ of signature $\mathcal{S}$ is a function $\alpha:\mathcal{V}\to D^\mathcal{M}$.}
		
		\item \alert{(9.2.6) We define the \emph{denotation} $\llbracket t\rrbracket_\alpha^\mathcal{M}$ of a term $t$ in a model $\mathcal{M}$ under assignment $\alpha$ by the following recursion:
		
			\begin{enumerate}[(a)]
			
				\item		\begin{enumerate}[(i)]

					\item $\llbracket x\rrbracket_\alpha^\mathcal{M}=\alpha(x)$
					\item $\llbracket c\rrbracket_\alpha^\mathcal{M}=c^\mathcal{M}$
				
				\end{enumerate}
				
				\item $\llbracket f(t_1,\mathellipsis,t_n)\rrbracket_\alpha^\mathcal{M}=f^\mathcal{M}(\llbracket t_1\rrbracket_\alpha^\mathcal{M}, \mathellipsis, \llbracket t_n\rrbracket_\alpha^\mathcal{M})$
			
			\end{enumerate}}
			
			\item (9.2.8) We $\alpha[x\mapsto d]$, the result of setting the value of variable $x\in\mathcal{V}$ to $d\in D^\mathcal{M}$, by:
		\[\alpha[x\mapsto d](y)=\begin{cases} y &\text{if }y\neq x\\ d & \text{if }y=x\end{cases}\]		

	
	\end{itemize}

\end{frame}

\begin{frame}{Examples (9.2.7)}

\begin{enumerate}[(i)]
		
			\item In the standard model for $\mathcal{S}_{PA}$ (9.1.2.i.a) with $\alpha(x)=0,\alpha(y)=1,\alpha(z)=2$ (9.2.5.i.a):
			\begin{align*}
			\llbracket 0\rrbracket^\mathcal{M}_\alpha&=0\\
			\llbracket x\rrbracket^\mathcal{M}_\alpha&=0\\
			\llbracket S(0)\rrbracket^\mathcal{M}_\alpha&=1\\
			\llbracket y\cdot S(0)\rrbracket^\mathcal{M}_\alpha&=1\\
			\llbracket S(((x\cdot y)+z))\rrbracket^\mathcal{M}_\alpha&=3
			\end{align*}
			
			
\end{enumerate}
\end{frame}
\begin{frame}{Examples (9.2.7)}

\begin{enumerate}[(i)]
\setcounter{enumi}{1}			
			\item In the non-intended model for $\mathcal{S}_{PA}$ (9.1.2.i.b), which has $D^\mathcal{M}=\{x:x\text{ is even}\}$, with $\alpha(x)=0,\alpha(y)=0,\alpha(z)=0$ (9.2.5.i.c):
			\begin{align*}
			\llbracket 0\rrbracket^\mathcal{M}_\alpha&=0\\
			\llbracket x\rrbracket^\mathcal{M}_\alpha&=0\\
			\llbracket S(0)\rrbracket^\mathcal{M}_\alpha&=2\\
			\llbracket y\cdot S(0)\rrbracket^\mathcal{M}_\alpha&=0\\
			\llbracket S(((x\cdot y)+z))\rrbracket^\mathcal{M}_\alpha&=2
			\end{align*}
			
	\end{enumerate}
\end{frame}
\begin{frame}{Examples (9.2.7)}

\begin{enumerate}[(i)]
\setcounter{enumi}{2}
			
			\item In the non-intended model for $\mathcal{S}_{PA}$ (9.1.2.i.c), which has $D^\mathcal{M}=\{x:x\text{ is odd}\}$, with $\alpha(x)=1$ for all $x\in\mathcal{V}$ (9.2.5.i.e):
			
			\begin{align*}
			\llbracket 0\rrbracket^\mathcal{M}_\alpha&=1\\
			\llbracket x\rrbracket^\mathcal{M}_\alpha&=1\\
			\llbracket S(0)\rrbracket^\mathcal{M}_\alpha&=3\\
			\llbracket y\cdot S(0)\rrbracket^\mathcal{M}_\alpha&=3\\
			\llbracket S(((x\cdot y)+z))\rrbracket^\mathcal{M}_\alpha&=5
			\end{align*}
			
	\end{enumerate}
\end{frame}
\begin{frame}{Examples (9.2.7)}

\begin{enumerate}[(i)]
\setcounter{enumi}{3}
			
			\item In the non-intended model for $\mathcal{S}_{PA}$ (9.1.2.i.d), which has $D^\mathcal{M}=\{x:x\text{ is odd}\}$, with $\alpha(x)=0,\alpha(y)=1,\alpha(z)=2$ (9.2.5.i.a):
			
			\begin{align*}
			\llbracket 0\rrbracket^\mathcal{M}_\alpha&=42\\
			\llbracket x\rrbracket^\mathcal{M}_\alpha&=0\\
			\llbracket S(0)\rrbracket^\mathcal{M}_\alpha&=42\\
			\llbracket y\cdot S(0)\rrbracket^\mathcal{M}_\alpha&=1^{42}=1\\
			\llbracket S(((x\cdot y)+z))\rrbracket^\mathcal{M}_\alpha&=(0^1)\cdot 2=2
			\end{align*}

\end{enumerate}


\end{frame}

\begin{frame}{The Term Locality Lemma}

 (9.2.9) \textbf{Lemma}. Let $\mathcal{M}$ be a model and $t$ a term with precisely the variables in set $V$ in it. Then for all valuations $\alpha$ and $\beta$ in $\mathcal{M}$, if $\alpha(x)=\beta(x)$ for all $x\in V$, then $\llbracket t\rrbracket_\alpha^\mathcal{M}=\llbracket t\rrbracket_\beta^\mathcal{M}$.
		
\vspace{2ex}	
		
\emph{Proof}. By induction:

\begin{itemize}

	\item Induction base:
	
	\begin{itemize}
	
	\item If $t$ is a constant $a\in\mathcal{C}$, we can reason that $\llbracket a\rrbracket^\mathcal{M}_\alpha=a^\mathcal{M}=\llbracket a\rrbracket^\mathcal{M}_\beta$ for all assignments $\alpha,\beta$. 
	
	\item If $t$ is a variable $x$, then $V=\{x\}$ and so $\llbracket x\rrbracket^\mathcal{M}_\alpha=\alpha(x)=\beta(x)=\llbracket a\rrbracket^\mathcal{M}_\beta$. 
	
	\end{itemize}
	
	\item Induction step:

\vspace{-2ex}	
	\end{itemize}

	{\small
	\begin{center}
		\begin{tabular}{c c c c c c ll}
		$\llbracket f(t_1, \mathellipsis, t_n)\rrbracket^\mathcal{M}_\alpha$ = & $f^\mathcal{M}($ & $\llbracket t_1\rrbracket^\mathcal{M}_\alpha$ &  \dots & $\llbracket t_n\rrbracket^\mathcal{M}_\alpha)$\\
		 & & \rotatebox{90}{=} & & \rotatebox{90}{=} &(I.H.)\\
		& $f^\mathcal{M}($ & $\llbracket t_1\rrbracket^\mathcal{M}_\beta$ &  \dots & $\llbracket t_n\rrbracket^\mathcal{M}_\beta)$&=$\llbracket f(t_1, \mathellipsis, t_n)\rrbracket^\mathcal{M}_\beta$ \\
		\end{tabular}
		\end{center}}


\end{frame}

\subsection{9.3 Truth in a Model}

\begin{frame}{9.3 Truth in a Model}

(9.3.1) We define the truth-value $\llbracket\phi\rrbracket^\mathcal{M}_\alpha$ of a formula $\phi$ under an assignment $\alpha$ in a model $\mathcal{M}$ by the following recursion:
		
		\begin{enumerate}[(i)]
			
				\item		\begin{enumerate}[(a)]

					\item $\llbracket R(t_1,\mathellipsis, t_n)\rrbracket_\alpha^\mathcal{M}=\begin{cases} 1 & \text{if }(\llbracket t_1\rrbracket^\mathcal{M}_\alpha,\mathellipsis, \llbracket t_1\rrbracket^\mathcal{M}_\alpha)\in R^\mathcal{M}\\0 &\text{otherwise}\end{cases}$
					\item $\llbracket t_1=t_2\rrbracket_\alpha^\mathcal{M}=\begin{cases} 1 & \text{if }\llbracket t_1\rrbracket^\mathcal{M}_\alpha=\llbracket t_1\rrbracket^\mathcal{M}_\alpha)\\0 &\text{otherwise}\end{cases}$				
				\end{enumerate}
				
				\item \begin{enumerate}[(a)]

					\item  $\llbracket\neg \phi\rrbracket_v=f_\neg(\llbracket\phi\rrbracket_v)$
				
				\item  $\llbracket(\phi\circ \psi)\rrbracket_v=f_\circ( \llbracket\phi\rrbracket_v, \llbracket\psi\rrbracket_v)$ for $\circ=\land,\lor,\to,\leftrightarrow$
				
				\item $\llbracket\exists x\phi\rrbracket_\alpha^\mathcal{M}=max(\{\llbracket \phi\rrbracket^\mathcal{M}_{\alpha[x\mapsto d]}: d\in D^\mathcal{M}\})$
				
				\item[] $\llbracket\forall x\phi\rrbracket_\alpha^\mathcal{M}=min(\{\llbracket \phi\rrbracket^\mathcal{M}_{\alpha[x\mapsto d]}: d\in D^\mathcal{M}\})$
								
				\end{enumerate}
			
			\end{enumerate}

\end{frame}

\begin{frame}{Truth as a Property}

	\begin{itemize}
	
		\item (9.3.2) We set $\mathcal{M},\alpha\vDash\phi$ iff $\llbracket\phi\rrbracket^\mathcal{M}_\alpha=1$.
		
		\item (9.3.4) \textbf{Lemma}. For every model $\mathcal{M}$ and assignment $\alpha$, we have:
			
			\begin{enumerate}[(i)]
			
				\item $\mathcal{M},\alpha\vDash R(t_1,\mathellipsis, t_n)$ iff $(\llbracket t_1\rrbracket^\mathcal{M}_\alpha,\mathellipsis, \llbracket t_1\rrbracket^\mathcal{M}_\alpha)\in R^\mathcal{M}$
				
				\item $\mathcal{M},\alpha\vDash t_1=t_2$ iff $\llbracket t_1\rrbracket^\mathcal{M}_\alpha=\llbracket t_1\rrbracket^\mathcal{M}_\alpha$								
				\item $\mathcal{M},\alpha\vDash \neg\phi$ iff $v\nvDash\phi$
					
				\item $\mathcal{M},\alpha\vDash(\phi\land\psi)$ iff $\mathcal{M},\alpha\vDash\phi$ and $\mathcal{M},\alpha\vDash\psi$
				
				\item $\mathcal{M},\alpha\vDash(\phi\lor\psi)$ iff $\mathcal{M},\alpha\vDash\phi$ or $\mathcal{M},\alpha\vDash\psi$
				
				\item $\mathcal{M},\alpha\vDash(\phi\to\psi)$ iff $v\nvDash\phi$ or $\mathcal{M},\alpha\vDash\psi$
				
				\item $\mathcal{M},\alpha\vDash(\phi\leftrightarrow\psi)$ iff   either $\mathcal{M},\alpha\vDash\phi$ and $\mathcal{M},\alpha\vDash\psi$, or $v\nvDash\phi$ and $v\nvDash\psi$.				
				\item $\mathcal{M},\alpha\vDash\exists x\phi$ iff there exists a $d\in D^\mathcal{M}$, s.t.  $\mathcal{M},{\alpha[x\mapsto d]}\vDash \phi$
				
				\item $\mathcal{M},\alpha\vDash\forall x\phi$ iff for all $d\in D^\mathcal{M}$, we have $\mathcal{M},{\alpha[x\mapsto d]}\vDash \phi$
											
			\end{enumerate}
	\end{itemize}

\end{frame}

\begin{frame}{(9.3.4.i) Examples}

In the standard model for $\mathcal{S}_{PA}$:

\begin{enumerate}[(a)]
			
				\item $\mathcal{M},\alpha\vDash S(0)=y$
				
				$\llbracket S(0)\rrbracket^\mathcal{M}_\alpha=S^\mathcal{M}(\llbracket 0\rrbracket^\mathcal{M}_\alpha)=S^\mathcal{M}(0)=1$ and $\llbracket y\rrbracket^\mathcal{M}_\alpha=\alpha(y)=1$.
				
				\setcounter{enumi}{5}
				
%				\item $\mathcal{M},\alpha\nvDash S(0)=x$
%								
%				\item $\mathcal{M},\alpha\vDash S(0)\neq x$
%				
%				
%				\item $\mathcal{M},\alpha\vDash (S(0)\neq x\lor 4\neq 4)$
%									
%				\item $\mathcal{M},\alpha\vDash S(0)=x\to 1\neq 1$
									
				\item $\mathcal{M},\alpha\vDash\forall x(S(x)\neq 0)$

				{We need to show that for each  $n\in D^\mathcal{M}=\mathbb{N}$, we have $\mathcal{M},\alpha[x\mapsto n]\vDash S(x)\neq 0$. So let $n\in \mathbb{N}$ be an arbitrary number. We know that  $\mathcal{M},\alpha[x\mapsto n]\vDash S(x)\neq 0$ iff $\mathcal{M},\alpha[x\mapsto n]\nvDash S(x)=0$. Assume $\mathcal{M},\alpha[x\mapsto n]\vDash S(x)=0$ for contradiction. It follows that $\llbracket S(x)\rrbracket^\mathcal{M}_{\alpha[x\mapsto n]}=\llbracket0\rrbracket^\mathcal{M}_{\alpha[x\mapsto n]}$. We have $\llbracket S(x)\rrbracket^\mathcal{M}_{\alpha[x\mapsto n]}=S^\mathcal{M}(n)=n+1$. And we have $0^\mathcal{M}=0$. So we get that $n+1=0$. But we know that there is no natural number $n\in\mathbb{N}$ s.t. $n+1=0$. So, we can conclude that $\mathcal{M},\alpha[x\mapsto n]\vDash S(x)\neq 0$, as desired.}
				
				\end{enumerate}
				
				\end{frame}

\begin{frame}{(9.3.4.i) Examples}

		{\small\begin{enumerate}[(a)]
			
				
				\setcounter{enumi}{6}
				
				\item $\mathcal{M},\alpha\vDash \exists x S(x)=S(S(0))$
				
				We need to establish that there exists an $n\in\mathbb{N}$ s.t.  $\mathcal{M},\alpha[x\mapsto n]\vDash S(x)=S(S(0))$. We can easily check that $\llbracket S(S(0))\rrbracket_\alpha^\mathcal{M}=2$. So, for $n=1$, we have $\llbracket S(x)\rrbracket^\mathcal{M}_{\alpha[x\mapsto 1]}=S^\mathcal{M}(\alpha[x\mapsto 1](x))=S^\mathcal{M}(1)=2$. By Lemma 9.2.9, we have $\llbracket S(S(0))\rrbracket_{\alpha[x:\mapsto1]}^\mathcal{M}=2$ since $\llbracket S(S(0))\rrbracket_\alpha^\mathcal{M}=2$ and $S(S(0))$ is a ground-term. Hence: $\llbracket S(x)\rrbracket^\mathcal{M}_{\alpha[x\mapsto n]}=\llbracket S(S(0))\rrbracket_{\alpha[x:\mapsto1]}^\mathcal{M},$ as desired.			
				
				\item $\mathcal{M},\alpha\vDash \forall x\exists y x\cdot y=x$.
				
			This holds iff for all $n\in\mathbb{N}$, $\mathcal{M},\alpha[x\mapsto n]\vDash \exists y x\cdot y=x$. We have $\mathcal{M},\alpha[x\mapsto n]\vDash \exists y x\cdot y=x$ iff there exists an $m\in\mathbb{N}$ s.t. $\mathcal{M},\alpha[x\mapsto n, y\mapsto m]\vDash x\cdot y=x$. We need to show that for each $n$, there exists an $m$ s.t. $\mathcal{M},\alpha[x\mapsto n, y\mapsto m]\vDash x\cdot y=x$. So let $n\in\mathbb{N}$ be arbitrary. For $m=1$, we get 
				\begin{align*}
				\llbracket x\cdot y\rrbracket^\mathcal{M}_{\alpha[x\mapsto n, y\mapsto 1]}&=\alpha[x\mapsto n, y\mapsto 1](x)\cdot \alpha[x\mapsto n, y\mapsto 1](y)=n\cdot 1
				\end{align*}
			So $\llbracket x\cdot y\rrbracket^\mathcal{M}_{\alpha[x\mapsto n, y\mapsto 1]}=\llbracket x\rrbracket^\mathcal{M}_{\alpha[x\mapsto n, y\mapsto 1]}$ and  $\mathcal{M},\alpha[x\mapsto n, y\mapsto 1]\vDash x\cdot y=x$. So $\mathcal{M},\alpha[x\mapsto n]\vDash \exists y x\cdot y=x$ and, since $n$ was arbitrary $\mathcal{M},\alpha\vDash \forall x \exists y x\cdot y=x$, as desired.
				
			\end{enumerate}}
	

\end{frame}

\begin{frame}{(9.3.4.ii) Examples}

For signature $\mathcal{S}=(\{a,b,c\}, \{f^1, g^2\}, \{P^1, R^2\})$:
							
						
						\begin{itemize}
						
							\item $D^\mathcal{M}=\{1,2,3,4\}$
							
							\item $a^\mathcal{M}=1, b^\mathcal{M}=3, c^\mathcal{M}=2$
						
							\item $f^\mathcal{M}(x)=x$ and $g^\mathcal{M}(x,y)=min(x,y)$
							
							\item $P^\mathcal{M}=\{1,3\}$ and $R^\mathcal{M}=\{(1,1), (1,2),(2,2) (2,3), (3,3)\}$
						
						\end{itemize}
						
	Examples:
	\begin{enumerate}[(a)]
			
			\item $\mathcal{M},\alpha\vDash P(a)$
				
				Simply note that $\llbracket A\rrbracket^\mathcal{M}_\alpha=a^\mathcal{M}=1\in \{1,3\}=P^\mathcal{M}$		
					
				\item $\mathcal{M},\alpha\vDash P(x)$
				
				Simply note that $\llbracket x\rrbracket^\mathcal{M}_\alpha=\alpha(x)=1\in \{1,3\}=P^\mathcal{M}$
				
				\item $\mathcal{M},\alpha\nvDash P(z)$
				
				Simply note that $\llbracket z\rrbracket^\mathcal{M}_\alpha=\alpha(z)=4\notin \{1,3\}=P^\mathcal{M}$
				
				\item $\mathcal{M},\alpha\vDash R(x,x)$
				
				First,  remember that $\llbracket x\rrbracket^\mathcal{M}_\alpha=1.$ It follows that $(\llbracket x\rrbracket^\mathcal{M}_\alpha,\llbracket x\rrbracket^\mathcal{M}_\alpha)=(1,1)\in\{(1,1), (1,2),(2,2) (2,3), (3,3)\}\in R^\mathcal{M}$.
				
	\end{enumerate}
						

\end{frame}

\begin{frame}{(9.3.4.ii) Examples}

{\small
	\begin{enumerate}[(a)]
	\setcounter{enumi}{4}


\item $\mathcal{M},\alpha\vDash \exists y (y\neq x\land R(y,y))$
				
				We need to show that there exists a $d\in D^\mathcal{M}$ such that $\mathcal{M},\alpha[y\mapsto d]\vDash y\neq x\land R(y,y)$. We have $\llbracket y\rrbracket^\mathcal{M}_{\alpha[y\mapsto 3]}=\alpha[y\mapsto 3](y)=3$  and $3\neq 1=\alpha[y\mapsto 3](x)$. Hence $\mathcal{M},\alpha[y\mapsto 3]\vDash y\neq x$. Since $\llbracket y\rrbracket^\mathcal{M}_{\alpha[y\mapsto 3]}=3,$ we have that $(\llbracket y\rrbracket^\mathcal{M}_{\alpha[y\mapsto 3]},\llbracket y\rrbracket^\mathcal{M}_{\alpha[y\mapsto 3]})=(3,3)\in R^\mathcal{M}$. So, we have $\mathcal{M},\alpha[y\mapsto 3]\vDash R(y,y)$. We have $\mathcal{M},\alpha[y\mapsto 3]\vDash y\neq x\land R(y,y)$. We get $\mathcal{M},\alpha\vDash \exists y (y\neq x\land R(y,y))$, as desired.
				
				\item $\mathcal{M},\alpha\vDash \forall x P(g(a,x))$
				
				 We show that for each $d\in D^\mathcal{M},$ we have $\llbracket g(a,x)\rrbracket^\mathcal{M}_{\alpha[x\mapsto d]}=1$. Since $1\in P^\mathcal{M}$, the claim follows.  We know that $\llbracket g(a,x)\rrbracket^\mathcal{M}_{\alpha[x\mapsto d]}=g^\mathcal{M}(\llbracket a\rrbracket^\mathcal{M}_{\alpha[x\mapsto d]}, \llbracket x\rrbracket^\mathcal{M}_{\alpha[x\mapsto d]})$, since $g^\mathcal{M}(x,y)=min(x,y)$, $\llbracket a\rrbracket^\mathcal{M}_{\alpha[x\mapsto d]}=a^\mathcal{M}$, and $\llbracket x\rrbracket^\mathcal{M}_{\alpha[x\mapsto d]}=\alpha[x\mapsto d](d)=d$, we get $\llbracket g(a,x)\rrbracket^\mathcal{M}_{\alpha[x\mapsto d]}=min(1,d)$. Now, $D^\mathcal{M}=\{1,2,3,4\}$, so for each $d\in D^\mathcal{M}$, we have $\llbracket g(a,x)\rrbracket^\mathcal{M}_{\alpha[x\mapsto d]}=min(1,d)=1$. But that's all we needed to show.
				
				\item $\mathcal{M},\alpha\vDash \forall x P(a)$. 
				
				%We need to show that for each $d\in D^\mathcal{M}$ that $\mathcal{M},\alpha[x\mapsto d]\vDash P(a)$. But the value of $P(a)$ is the same under each valuation: the proof of $\mathcal{M},\alpha\vDash P(a)$ doesn't depend on $\alpha$. So, clearly for each $d\in D^\mathcal{M}$ that $\mathcal{M},\alpha[x\mapsto d]\vDash P(a)$.
			
			\end{enumerate}
			}
	\end{frame}
	
\begin{frame}{Two Lemmas}

	\begin{itemize}

		\item (9.3.7) \textbf{Lemma.} Let $\mathcal{M}$ be a model and $\phi\in\mathcal{L}$ whose free variables form the set $V$. Then for all valuations $\alpha$ and $\beta$, if $\alpha(x)=\beta(x)$ for all $x\in V$, then $\llbracket\phi\rrbracket_\alpha^\mathcal{M}=\llbracket\phi\rrbracket_\beta^\mathcal{M}$.
		

\item (9.3.7) \textbf{Corollary.}	Let $\mathcal{M}$ be a model and $\phi\in\mathcal{L}$ a sentence (i.e. a formula with no free variables). Then for all assignments $\alpha,\beta$, we have $\llbracket\phi\rrbracket_\alpha^\mathcal{M}=\llbracket\phi\rrbracket_\beta^\mathcal{M}$.
		
		
		\item Proofs in the notes. 
		
		\item We can use the Sentence Lemma to justify the following definition of truth in a model:
		
		\begin{itemize}
		
			\item For $\phi$ a sentence, we define $\mathcal{M}\vDash\phi$ as $\mathcal{M},\alpha\vDash\phi$.
		
		\end{itemize}


\end{itemize}

\end{frame}

\subsection{9.4 Consequence and Validity}

\begin{frame}{9.4 Consequence and Validity}

	\begin{itemize}
	
		\item (9.4.1) We set:
		
		\begin{itemize}
		
			\item For $\Gamma$ a set of sentences and $\phi$ a sentence, we say that $\Gamma\vDash\phi$ iff for all models $\mathcal{M}$, if $\mathcal{M}\vDash\psi$ for all $\psi\in\Gamma$, then $\mathcal{M}\vDash\phi$.
			
			\item This gives us: $\Gamma\nvDash\phi$ iff there exists a (counter)model $\mathcal{M}$, such taht $\mathcal{M}\vDash\psi$ for all $\psi\in\Gamma$, but $\mathcal{M}\nvDash\phi$
		
		\end{itemize}
		\item The expression $\phi\equi \psi$ still means both $\phi\vDash\psi$ and $\psi\vDash\phi$
		
		\item $\vDash\phi$ still means $\emptyset\vDash\phi$
	
	\end{itemize}

\end{frame}

\begin{frame}{(9.4.2) Examples}

\begin{enumerate}[(i)]
		
		\item $\forall xP(x)\vDash P(a)$
		
		\item[] To see this, suppose that $\mathcal{M},\alpha\vDash\forall xP(x)$ for some arbitrary model $\mathcal{M}$ and valuation $\alpha$. It follows that for all $d\in D^\mathcal{M}$, we have $\mathcal{M},\alpha[x\mapsto d]\vDash P(x)$. But $a^\mathcal{M}\in D^\mathcal{M}$. So, if we set $d=a^\mathcal{M}$, we get $\mathcal{M},\alpha[x\mapsto a^\mathcal{M}]\vDash P(x)$. But that just means that $\alpha[x\mapsto a^\mathcal{M}](x)=a^\mathcal{M}\in P^\mathcal{M}$, from which it immediately follows that $\mathcal{M},\alpha\vDash P(a)$.
		
		\item $P(a)\vDash\exists xP(x)$
		
		\item[] To see this, suppose that $\mathcal{M},\alpha\vDash P(a)$ for some arbitrary model $\mathcal{M}$ and valuation $\alpha$. This means that $a^\mathcal{M}\in P^\mathcal{M}$. But then, we can simply set $d=a^\mathcal{M}$, and get that $\mathcal{M},\alpha[x\mapsto a^\mathcal{M}]\vDash P(x)$ and so $\mathcal{M},\alpha\vDash\exists xP(x)$, as desired.

\end{enumerate}

\end{frame}

\begin{frame}{(9.4.2) Examples}

\begin{enumerate}[(i)]

\setcounter{enumi}{2}

\item $\exists x(P(x)\land Q(x))\vDash \exists xP(x)\land \exists xQ(x)$

		\item[] Suppose that $\llbracket\exists x(P(x)\land Q(x))\rrbracket^\mathcal{M}_\alpha=1$. That means that we can change the value of only $x$ to $d$ such that $\llbracket P(x)\land Q(x)\rrbracket^\mathcal{M}_{\alpha[x\mapsto d]}=1$. Hence $\llbracket P(x)\rrbracket\rrbracket^\mathcal{M}_{\alpha[x\mapsto d]}=1$ and $\llbracket Q(x)\rrbracket\rrbracket^\mathcal{M}_{\alpha[x\mapsto d]}=1$. So we can change the value of only $x$ to $d$ such that $\llbracket P(x)\rrbracket\rrbracket^\mathcal{M}_{\alpha[x\mapsto d]}=1$, meaning $\llbracket\exists xP(x)\rrbracket^\mathcal{M}_\alpha=1$; and we can change the value of only $x$ to $d$ such that $\llbracket Q(x)\rrbracket\rrbracket^\mathcal{M}_{\alpha[x\mapsto d]}=1$, meaning $\llbracket\exists xQ(x)\rrbracket^\mathcal{M}_\alpha=1$. Hence $\llbracket\exists xP(x)\land \exists xQ(x)\rrbracket^\mathcal{M}_\alpha=1$, as desired.
		
		\end{enumerate}

\end{frame}

\begin{frame}{(9.4.2) Examples}

\begin{enumerate}[(i)]

\setcounter{enumi}{4}

\item $\vDash \forall x(P(x)\lor \neg P(x))$
		
		\item[] To see this, suppose that for some model $\mathcal{M}$ and assignment $\alpha$, we have $\mathcal{M},\alpha\nvDash  \forall x(P(x)\lor \neg P(x))$. This means that there exists a $d\in D^\mathcal{M}$ such that $\mathcal{M},\alpha[x\mapsto d]\nvDash  P(x)\lor \neg P(x)$. But this means that both $\mathcal{M},\alpha[x\mapsto d]\nvDash  P(x)$ and $\mathcal{M},\alpha[x\mapsto d]\nvDash  \neg P(x)$ and so both $\mathcal{M},\alpha[x\mapsto d]\nvDash  P(x)$ and $\mathcal{M},\alpha[x\mapsto d]\vDash  P(x),$ which is a contradiction. Hence $\mathcal{M},\alpha\nvDash  \forall x(P(x)\lor \neg P(x))$ for all $\mathcal{M}$ and  $\alpha$.
		
		\item $\vDash\exists x~x=\mathsf{Batman}$ (it's logically true that Batman exists)
		
		\item[] Let $\mathcal{M}$ be an arbitrary model and $\alpha$ an arbitrary assignment therein. We have $\mathsf{Batman}^\mathcal{M}\in D^\mathcal{M}$; that is, the denotation of $\mathsf{Batman}$ is a member of the domain. So, set set $d=\mathsf{Batman}^\mathcal{M}$ and consider  $\llbracket x\rrbracket^\mathcal{M}_{\alpha[x\mapsto\mathsf{Batman}^\mathcal{M}]}=\mathsf{Batman}^\mathcal{M}=\llbracket\mathsf{Batman}\rrbracket^\mathcal{M}_{\alpha[x\mapsto\mathsf{Batman}^\mathcal{M}]}$. So $\llbracket x=\mathsf{Batman}\rrbracket^\mathcal{M}_{\alpha[x\mapsto\mathsf{Batman}^\mathcal{M}]}=1$. So $\llbracket\exists x~x=\mathsf{Batman}\rrbracket^\mathcal{M}_\alpha=1$, which is what we needed to show.		
		\end{enumerate}

\end{frame}

\begin{frame}{(9.4.2) Examples}

\begin{enumerate}[(i)]

\setcounter{enumi}{7}

\item $\exists xP(x)\land \exists x Q(x)\nvDash \exists x(P(x)\land Q(x))$ 
		
		\item[] Here's a countermodel:
		
		\begin{itemize}
		
			\item $D^\mathcal{M}=\{a,b\}$
			
			\item $P^\mathcal{M}=\{a\}$
			\item $Q^\mathcal{M}=\{b\}$
		
		\end{itemize}
		
		In this model, there we can find $a^\mathcal{M}$ such that $\mathcal{M},\alpha[x\mapsto a^\mathcal{M}]\vDash P(x)$ and so $\mathcal{M},\alpha\vDash\exists x P(x)$, and we can find $b^\mathcal{M}$ such that $\mathcal{M},\alpha[x\mapsto b^\mathcal{M}]\vDash Q(x)$ and so $\mathcal{M},\alpha\vDash\exists x Q(x)$. But neither $a^\mathcal{M}$ nor $b^\mathcal{M}$ is such that $\mathcal{M},\alpha[x\mapsto a^\mathcal{M}/b^\mathcal{M}]\vDash P(x)\land Q(x)$---nothing is both $P$ and $Q$.
		
		\end{enumerate}

\end{frame}

\begin{frame}{(9.4.3) Quantifier Laws}

		
		\begin{enumerate}[(i)]
		
		\item $\forall x\phi\vDash(\phi)[x:=t]$  where $t$ is a ground term
%		
		\item $(\phi)[x:=t]\vDash\exists x\phi$ where $t$ is a ground term

		\item $\forall x\phi\vDash\exists x\phi$
		
		\item $\forall x\phi\equi \neg \exists x\neg\phi$
	
			\item $\exists x\phi\equi \neg \forall x\neg \phi$
		
			\item $\forall x\forall y\phi\equi \forall y\forall x\phi$

	\item $\exists x\exists y\phi\equi \exists y\exists x\phi$

	\item $\exists x\forall y\phi\vDash \forall y \exists x\phi$
		
	\item $(\forall x\phi\land \forall x\psi)\equi \forall x(\phi\land \psi)$

	\item $(\exists x\phi\lor \exists x\psi)\equi \exists x(\phi\lor \psi)$

	\item $\forall x\phi\lor \forall x\psi\vDash \forall x(\phi\lor \psi)$  

	\item  $\exists x(\phi\land \psi)\vDash \exists x\phi\land \exists x \psi$  
	\item $(\phi\to \forall x\psi)\equi \forall x(\phi\to \psi)$ if $x$ is not free in $\phi$

	\item $(\phi\to \exists x\psi)\equi \exists x(\phi\to \psi)$ if $x$ is not free in $\phi$

	\item $(\forall x\phi\to \psi)\equi \exists x(\phi\to \psi)$ if $x$ is not free in $\psi$

	\item $(\exists x\phi\to \psi)\equi \forall x(\phi\to \psi)$ if $x$ is not free in $\psi$

		\end{enumerate}

\end{frame}

\begin{frame}{Deduction and ICGNS}

	\begin{itemize}
	
		\item \textbf{Theorem} (Deduction Theorem). Let $\phi,\psi\in\mathcal{L}$ be formulas and $\Gamma\subseteq\mathcal{L}$ a set of formulas. Then the following two are equivalent:
			\begin{enumerate}[1.]
			
				\item $\Gamma\cup\{\phi\}\vDash\psi$
				
				\item $\Gamma\vDash \phi\to\psi$
			
			\end{enumerate}


		\item \textbf{Theorem} (I Can't Get No Satisfaction).
			Let $\Gamma\subseteq\mathcal{L}$ be a set of formulas and $\phi\in\mathcal{L}$ a formula. Then, the following are equivalent:
			\begin{enumerate}[1.]
			
				\item $\Gamma\vDash\phi$
				
				\item $\Gamma\cup\{\neg\phi\}$ is unsatisfiable
			
			\end{enumerate}

	
	\end{itemize}

\end{frame}

\begin{frame}{Albert, Betty, and Charles}

\begin{itemize}
		
		\item Consider the signature $\mathcal{S}=(\{a,b,c\}, \emptyset, \{M^1, L^2\})$.

		\item Our intended reading is that $M$ stands for ``\dots is married'', $L$ stands for ``\dots looks at \underline{\phantom{\dots}}'', $a$ means ``Albert,'' $b$ stands for ``Betty,'' and $c$ stands for ``Charles.''
		
		\item \emph{Claim}: \[\neg M(a), M(c), L(c,b), L(b,a)\vDash \exists x\exists y(M(x)\land \neg M(y)\land L(x,y)).\]
		
		\item \emph{Proof}: 	
		
		Let $\mathcal{M}$ be a model and $\alpha$ arbitrary, such that $\llbracket M(c)\rrbracket_\alpha^\mathcal{M}=1$, $\llbracket\neg M(a)\rrbracket_\alpha^\mathcal{M}=1$, $\llbracket L(c,b)\rrbracket_\alpha^\mathcal{M}=1$, and $\llbracket L(b,a)\rrbracket_\alpha^\mathcal{M}=1$. So $a^\mathcal{M}\notin M^\mathcal{M}$, $c^\mathcal{M}\in M^\mathcal{M}$, and $( c^\mathcal{M}, b^\mathcal{M}), ( b^\mathcal{M}, a^\mathcal{M})\in L^\mathcal{M}$. We have that $\llbracket \exists x\exists y(M(x)\land \neg M(y)\land L(x,y))\rrbracket_\alpha^\mathcal{M}=1$ holds iff there are changes for $x$ to $d$ and $y$ to $d'$ such that \[\llbracket (M(x)\land \neg M(y)\land L(x,y))\rrbracket_{\alpha[x\mapsto d, y\mapsto d']}^\mathcal{M}=1.\] 
	\end{itemize}

\end{frame}

\begin{frame}{Albert, Betty, and Charles}

\begin{itemize}
				
		\item Now, we know that either (i) $b^\mathcal{M}\in M^\mathcal{M}$ or (ii) $b^\mathcal{M}\notin M^\mathcal{M}$. 	
 \begin{itemize}
			
			\item If (i) $b^\mathcal{M}\in M^\mathcal{M}$, then we can set $d=b^\mathcal{M}$ and $d'=a^\mathcal{M}$. We'd get $d\in M^\mathcal{M}$ and so $\llbracket M(x))\rrbracket_{\alpha[x\mapsto d, y\mapsto d']}^\mathcal{M}=1$; $d'\notin M^\mathcal{M}$ and so $\llbracket\neg M(y)\rrbracket_{\alpha[x\mapsto d, y\mapsto d']}^\mathcal{M}=1$; and $( d,d')\in L^\mathcal{M}$ and so $\llbracket L(x,y))\rrbracket_{\alpha[x\mapsto d, y\mapsto d']}^\mathcal{M}=1$; giving us, $\llbracket \exists x\exists y(M(x)\land \neg M(y)\land L(x,y))\rrbracket_{\alpha[x\mapsto d, y\mapsto d']}^\mathcal{M}=1$. 
			
		\item If (ii) $b^\mathcal{M}\notin M^\mathcal{M}$, we can set $d=c^\mathcal{M}$ and $d'=b^\mathcal{M}$. In a similar way, we get 
				
		\end{itemize}

Either way, we get $\llbracket \exists x\exists y(M(x)\land \neg M(y)\land L(x,y))\rrbracket_{\alpha[x\mapsto d, y\mapsto d']}^\mathcal{M}=1$, which is what we wanted to show.

	\end{itemize}

\end{frame}


\begin{frame}

\begin{center}
\huge \smiley
\end{center}

\end{frame}

\begin{frame}{Core Ideas (Lecture Version)}

	\begin{itemize}
	
		\item A model interprets the signature by assigning denotation to every constant, a function to every function symbol, and an $n$-ary relation to every $n$-ary relation symbol
		
		\item An assignment in a model tells us what the variables denote. It plays the role of the context in natural language. 
	
		\item We can recursively calculate the denotation of arbitrary terms in a model under an assignment.
		
		\item We can recursively calculate the truth-value of a formula relative to a model under an assignment.
		
		\item Validity is defined as in every logic as truth-preservation across models.
		
		\item The Deduction Theorem holds for first-order logic but doesn't lead to decidability.
					
	\end{itemize}

\end{frame}


\begin{frame}

	\begin{center}
	{\huge\bf Thanks!}
	\end{center}

\end{frame}

