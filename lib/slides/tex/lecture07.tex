\setcounter{framenumber}{200}
\begin{frame}
  \LectureNo{7}
  \maketitle
\end{frame}

\begin{frame}{Overview}
\tableofcontents
\end{frame}

\section{Rehash}
\begin{frame}{Rehash}
	
\begin{itemize}

	\item Our proof system is the method of tableaux.

	\item $\Gamma\vDash\phi$ iff $\Gamma\cup\{\neg\phi\}$ is unsatisfiable.
	
	{\tiny\begin{center}
					
					\begin{prooftree}
					{
					line numbering=false,
					line no sep= 2cm,
					for tree={s sep'=5mm},
					single branches=true,
					close with=\xmark
					}
					[\neg\neg \phi [\phi ] ]
					\end{prooftree}
					%
					\begin{prooftree}
					{
					line numbering=false,
					line no sep= 2cm,
					for tree={s sep'=5mm},
					single branches=true,
					close with=\xmark
					}
					[\phi\land\psi [\phi [\psi ] ] ]
					\end{prooftree}
					%
					\begin{prooftree}
					{
					line numbering=false,
					line no sep= 2cm,
					for tree={s sep'=5mm},
					single branches=true,
					close with=\xmark
					}
					[\neg (\phi\land\psi) [\neg \phi ] [\neg \psi ] ]
					\end{prooftree}
					%
					\begin{prooftree}
					{
					line numbering=false,
					line no sep= 2cm,
					for tree={s sep'=5mm},
					single branches=true,
					close with=\xmark
					}
					[\phi\lor\psi [\phi ] [\psi ] ]
					\end{prooftree}
					%
					\begin{prooftree}
					{
					line numbering=false,
					line no sep= 2cm,
					for tree={s sep'=5mm},
					single branches=true,
					close with=\xmark
					}
					[\neg(\phi\lor\psi) [\neg\phi [\neg\psi ] ] ]
					\end{prooftree}

					\vspace{2ex}

					\begin{prooftree}
					{
					line numbering=false,
					line no sep= 2cm,
					for tree={s sep'=5mm},
					single branches=true,
					close with=\xmark
					}
					[\neg (\phi\to\psi) [\phi [\neg \psi ] ] ]
					\end{prooftree}
					%
					\begin{prooftree}
					{
					line numbering=false,
					line no sep= 2cm,
					for tree={s sep'=5mm},
					single branches=true,
					close with=\xmark
					}
					[\phi\to\psi [\neg \phi ] [\psi ] ]
					\end{prooftree}
					%
					\begin{prooftree}
					{
					line numbering=false,
					line no sep= 2cm,
					for tree={s sep'=5mm},
					single branches=true,
					close with=\xmark
					}
					[\phi\leftrightarrow \psi [\phi [\psi] ] [\neg \phi [\neg \psi] ] ]]
					\end{prooftree}
					%
					\begin{prooftree}
					{
					line numbering=false,
					line no sep= 2cm,
					for tree={s sep'=5mm},
					single branches=true,
					close with=\xmark
					}
					[\neg(\phi\leftrightarrow \psi) [\phi [\neg \psi] ] [\neg \phi [ \psi] ] ]]
					\end{prooftree}

				\end{center}}
				
	\item A branch $B$ is closed iff $p,\neg p\in B$ for some $p\in\mathcal{P}$.
	
	\item A tableau is closed iff every branch is closed.
	
	\item $\Gamma\vdash \phi$ iff the complete tableau for $\Gamma\cup\{\neg\phi\}$ is closed.
	
	\item $\Gamma\nvdash\phi$  iff there is at least one open branch.
	
\[v_B(p)=\begin{cases} 1 &\text{if }p\in B\\0&\text{if }p\notin B\end{cases}\]
\end{itemize}


\end{frame}
		

\section{7. Soundness and Completeness}
\subsection{7.1 Soundness and Completeness}

\begin{frame}{7.1 Soundness and Completeness}

	\begin{itemize}
		
		\item (7.1.1) We want to be able to  derive the conclusion from the premises in all and only the valid inferences:
		
				\begin{description}
		
			\item[Soundness Theorem.] If $\Gamma\vdash\phi$, then $\Gamma\vDash\phi$.
			
			\item[Completeness Theorem.] If $\Gamma\vDash\phi$, then $\Gamma\vdash\phi$.
		
		\end{description}
	
		\item (7.1.2) Together, the two reduce semantic validity to syntactic validity.
	
		\item (7.1.3) Soundness is a ``sanity check:'' only if it follows, we can infer it.
		
		\item (7.1.4) Completeness is a strong property: \emph{everything} that follows, you can infer.
		
	\end{itemize}

\end{frame}

\begin{frame}{Proof Idea}

	\begin{itemize}
	
		\item (7.1.6) Remember:
		
		\begin{description}
			
				\item[Down Preservation.] If the formula at the parent node of a rule is true under a valuation, then at least one formula on a newly generated child node is true under the valuation.
				
				\item[Up Preservation.] If a formula at a newly generated child node is true under a valuation, then the formula at the parent node is true.
				
		\end{description}
		
		\item Together, up and down preservation guarantee that when we're applying the rules, we consider precisely the different ways for a formula to be true.
	
	\end{itemize}


\end{frame}

\subsection{7.2 The Soundness Theorem}

\begin{frame}{7.2 The Soundness Theorem --- Proof Idea}

	\begin{itemize}

		\item (7.2.1) We're actually going to prove the contrapositive:
		
			\begin{itemize}
			
				\item If $\Gamma\nvDash\phi$, then $\Gamma\nvdash\phi$.
			
			\end{itemize}
			
		\item If $\Gamma\nvDash\phi$, then there's a valuation $v$ that makes all the members of $\Gamma\cup\{\neg\phi\}$ true.
		
		\item That's the initial list. 
		
		
		\item By Down Preservation, whenever we apply a rule, we get at least one branch all of whose members $v$ makes true.
		
		\item So, in the complete tableau, there has to be one branch, $B$, such that $v$ makes all the formulas on $B$ true.

		\item But then $B$ cannot be closed. So the tableau is open.

	\end{itemize}

\end{frame}

\begin{frame}{The Soundness Lemma}

	\begin{itemize}
	
		\item Let $B$ be a branch of a possibly incomplete tableau and $v$ a valuation:
		
			\begin{itemize}
		
				\item (7.2.2) We say that $v$ is \emph{faithful} to $B$ iff $\llbracket\phi\rrbracket_v=1$, for all $\phi\in B$
	
			\end{itemize}
			
				\begin{center}{\tiny
\begin{prooftree}
{
proof statement format={centered},
to prove={p\to q, q\nvdash p},
line numbering=false,
for tree={s sep'=5mm},
single branches=true,
close with=\xmark
}
[p\to q, grouped [q, grouped [\neg p, grouped [\neg p] [q]]]]
\end{prooftree}

\vspace{2ex}
\emph{Countermodel}: $v_B(q)=1, v_B(p)=0$.}
\end{center}
			
			\item (7.2.3) \textbf{Lemma}. Let $v$ be a valuation that is faithful to a branch $B$ of an incomplete tableau. If a rule is applied to a formula in the tableau, then $v$ is faithful to at least one branch $B'$ which extends $B$ in the new tableau.
		
		\item This is a direct consequence of Down Preservation!
	
	\end{itemize}

\end{frame}

\begin{frame}{Example}

	\begin{itemize}
	
		\item $v_1$ with $v_1(p)=1$ and $v_1(q)=0$
		\item $v_2$ with $v_2(p)=0$ and $v_2(q)=1$
	
	\end{itemize}

	\only<1|handout:0>{\begin{center}
		\begin{prooftree}
					{
					line numbering=false,
					line no sep= 1cm,
					for tree={s sep'=5mm},
					single branches=true,
					close with=\xmark
					}
					[p\lor q, grouped [\neg p\lor \neg q, grouped ] ]
					\end{prooftree}
		\end{center}}
	\only<2|handout:0>{\begin{center}
		\begin{prooftree}
					{
					line numbering=false,
					line no sep= 1cm,
					for tree={s sep'=5mm},
					single branches=true,
					close with=\xmark
					}
					[p\lor q, grouped [\neg p\lor \neg q, grouped [p] [q]  ] ]
					\end{prooftree}
		\end{center}}
	\only<3|handout:0>{\begin{center}
		\begin{prooftree}
					{
					line numbering=false,
					line no sep= 1cm,
					for tree={s sep'=5mm},
					single branches=true,
					close with=\xmark
					}
					[p\lor q, grouped [\neg p\lor \neg q, grouped [p [\neg p] [\neg q] ] [q [\neg p] [\neg q] ]  ] ]
					\end{prooftree}
		\end{center}}
	\only<4|handout:1>{\begin{center}
		\begin{prooftree}
					{
					line numbering=false,
					line no sep= 1cm,
					for tree={s sep'=5mm},
					single branches=true,
					close with=\xmark
					}
					[p\lor q, grouped [\neg p\lor \neg q, grouped [p [\neg p, close] [\neg q] ] [q [\neg p] [\neg q, close] ]  ] ]
					\end{prooftree}
		\end{center}}


\end{frame}

\begin{frame}{(7.2.4)  Proof of the Soundness Lemma}

The proof consists in a one-by-one inspection of the rules.

There are 9 cases. We do: (a) $\phi\to\psi$ and (b) $\neg(\phi\to \psi)$.
		
		\begin{enumerate}[(a)]
					
		\item Let $v$ be faithful to $B$ and $\phi\to\psi\in B$.
		
					\begin{center}
					\begin{prooftree}
					{
					line numbering=false,
					line no sep= 2cm,
					for tree={s sep'=10mm},
					single branches=true,
					close with=\xmark
					}
					[\phi\to\psi [\neg \phi ] [\psi ] ]
					\end{prooftree}
					\end{center}
					
			We get two new branches extending $B$, $B_1$ and $B_2$. 
			
			\begin{itemize}
						
			\item We have $B_1=B\cup\{\neg\phi\}$ and $B_2=B\cup\{\psi\}$. 
			
			\item Since $v$ is faithful to $B$, we get $\llbracket\phi\to\psi\rrbracket_v=1$. 
			
			\item Since $\llbracket\phi\to\psi\rrbracket_v=max(1-\llbracket\phi\rrbracket_v,\llbracket\psi\rrbracket_v)$, we get either $\llbracket\phi\rrbracket_v=0$ or $\llbracket\psi\rrbracket_v=1$.
			
			\begin{itemize}
		
			\item If $\llbracket\phi\rrbracket_v=0$, then $\llbracket\neg\phi\rrbracket_v=1-\llbracket\phi\rrbracket_v=1$. So $v$ is faithful to $B_1=B\cup\{\neg\phi\}$. 
		
			\item If $\llbracket\phi\rrbracket_v=0$ or $\llbracket\psi\rrbracket_v=1$, then $v$ is faithful to $B_2=B\cup\{\psi\}$. 
	
		
			\end{itemize}
		
		\end{itemize}
	So, either way, $v$ is faithful to at least one new branch created by the rule for $\phi\to\psi$.
	
	\end{enumerate}
	
\end{frame}

\begin{frame}{(7.2.4)  Proof of the Soundness Lemma (Cont'd)}
	
	\begin{enumerate}[(a)]

	\setcounter{enumi}{1}
					
		\item Let $v$ be faithful to $B$ and $\neg(\phi\to\psi)\in B$.
				\begin{center}{\small
					\begin{prooftree}
					{
					line numbering=false,
					line no sep= 2cm,
					for tree={s sep'=10mm},
					single branches=true,
					close with=\xmark
					}
					[\neg(\phi\to\psi) [\phi [\neg\psi ] ] ]
					\end{prooftree}}
					\end{center}
			We get one new branch $B'=B\cup\{\phi,\neg\psi\}$.
			
			\begin{itemize}
			
				\item Since $v$ is faithful to $B$ , we get $\llbracket\neg(\phi\to\psi)\rrbracket_v=1$.
				
				\item Since $\llbracket\neg(\phi\to\psi)\rrbracket_v=1-max(1-\llbracket\phi\rrbracket_v, \llbracket\psi\rrbracket_v)$, we get that $max(1-\llbracket\phi\rrbracket_v, \llbracket\psi\rrbracket_v)=0$.
				
				\item So $\llbracket\phi\rrbracket_v=1$ and $\llbracket\psi\rrbracket_v=0$.
			
				\item Since $\llbracket\neg\psi\rrbracket_v=1-\llbracket\psi\rrbracket_v$, we can conclude  $\llbracket\neg\psi\rrbracket_v=1$.
				
				\item So $v$ is faithful to $B'$.
			
			\end{itemize}
					
	\end{enumerate}

\end{frame}

\begin{frame}{The Soundness Theorem}

(7.2.5) \textbf{Theorem} (Soundness)\textbf{.} If $\Gamma\vdash\phi$, then $\Gamma\vDash\phi$.


	\emph{Proof}:

	\begin{itemize}
	
		\item Suppose that $\Gamma\nvDash\phi$.
		
		\item So, there's a $v$ that makes all the members of $\Gamma\cup\{\neg\phi\}$ true.
		
		\item That's the initial list of our tableau.
		
		\item Every time we apply a rule, $v$ remains faithful to at least one branch.
		
		\item So, in the complete tableau, there's a branch $B$ that $v$ is faithful to.
		
		\item Suppose that $B$ is closed, i.e. $p,\neg p\in B$. 
		
		\item It follows that $v(p)=1$ and $v(p)=0$. Contradiction.
		
		\item So $B$ is open, meaning $\Gamma\nvdash\phi$.
	
	\end{itemize}

\end{frame}

\begin{frame}{Tableau Verification 1}

	(7.2.6) \textbf{Theorem.} If the tableau algorithm gives the answer that a set is unsatisfiable, then the set \emph{is} unsatisfiable. 

	\emph{Proof}:
	
	\begin{itemize}
	
		\item Suppose that the algorithm says $\Gamma$ is unsatisfiable.
		
		\item So, the tableau for $\Gamma$ is closed.
		
		\item Suppose, for contradiction, that $\Gamma$ is satisfiable via $v$.
		
		\item Using the soundness lemma, we can infer that there's a branch $B$ in the tableau for $\Gamma$ which is open (as in soundness). 
	
		\item Hence the tableau for $\Gamma$ is open. Contradiction.
		
		\item So $\Gamma$ is not satisfiable.
	
	\end{itemize}

\end{frame}

\subsection{7.3 The Completeness Theorem}

\begin{frame}{7.3 The Completeness Theorem --- Proof Idea}


	\begin{itemize}
	
		\item (7.3.1) We'll prove once more, the contrapositive:
		
		\begin{itemize}
		
			\item If $\Gamma\nvdash\phi$, then $\Gamma\nvDash\phi$.
		
		\end{itemize}
		
		\item The proof idea is simple:
		
			\begin{itemize}
			
				\item If $\Gamma\nvdash\phi$, then the tableau for $\Gamma\cup\{\neg\phi\}$ is open via open branch $B$.
				
				\item We get the associated valuation $v_B$ for $B$:
				
				\[v_B(p)=\begin{cases} 1 &\text{if }p\in B\\0&\text{if }p\notin B\end{cases}\]
 
				
				\item Using Up Preservation, we can inductively argue that $v_B$ makes all $B$'s true.
				
				\item But then $v_B$ satisfies  $\Gamma\cup\{\neg\phi\}$.
				
				\item Which gives us $\Gamma\nvDash\phi$. 
			
			\end{itemize}
	
		\item The devil is in the detail.
	
	\end{itemize}	

\end{frame}

\begin{frame}{The Completeness Lemma}

(7.3.2) \textbf{Lemma} (Completeness Lemma)\textbf{.} Let $B$ be an open branch of a complete tableau. Then $v_B$ is faithful to $B$.

\emph{Proof}: The proof is a (rather complicated) induction.

	\begin{itemize}
	
		\item We note that $v_B$ is faithful to $B$ iff for all $\phi\in\mathcal{L}$:
	\begin{enumerate}[1.]
	
		\item if $\phi\in B$, then $\llbracket\phi\rrbracket_{v_B}=1$, and 
		\item if $\neg \phi\in B$, then $\llbracket\phi\rrbracket_{v_B}=0$.
	
	\end{enumerate}
	
		\item \emph{This} we prove by induction.
		
		\begin{enumerate}[(i)]
	
		\item \emph{Base case}. We need to show that for all $p\in \mathcal{P}$:
		\begin{enumerate}[1.]
		 	\item If $p\in B$, then $\llbracket p\rrbracket_{v_B}=1$
			\item If $\neg p\in B$, then $\llbracket p\rrbracket_{v_B}=0$.
		\end{enumerate}
		
		\begin{itemize}
		
		\item 1. is immediate from the definition of $v_B$. 
		
		\item For 2. note that if $\neg p\in B$, then $p\notin B$ since $B$ is open. So, by definition, $v_B(p)=0$. Since $\llbracket p\rrbracket_{v_B}=v_B(p)$, we get our desired claim.
		
		\end{itemize}
		
		\end{enumerate}
	
	\end{itemize}

\end{frame}

\begin{frame}{The Completeness Lemma (Cont'd)}

\begin{enumerate}[(i)]
	
	\setcounter{enumi}{1}
	
	\item \emph{Induction steps}.
	
	\begin{enumerate}[(a)]
		
			\item We need to prove that if $\phi$ enjoys the property, then $\neg \phi$ enjoys the property. From the induction hypothesis
		
				\begin{enumerate}[1.]
	
					\item If $\phi\in B$, then $\llbracket\phi\rrbracket_{v_B}=1$. 
	
					\item If $\neg \phi\in B$, then $\llbracket\phi\rrbracket_{v_B}=0$.
	
				\end{enumerate}
	We derive:
		\begin{enumerate}[1'.]
	
		\item If $\neg\phi\in B$, then $\llbracket\neg\phi\rrbracket_{v_B}=1$.
			\item If $\neg\neg \phi\in B$, then $\llbracket\neg\phi\rrbracket_{v_B}=0$.	
		\end{enumerate}
		
		\begin{itemize}
	
		\item For 1', suppose that $\neg \phi\in B$. By 2., we have $\llbracket\phi\rrbracket_{v_B}=0$. But $\llbracket\neg\phi\rrbracket_{v_B}=1-\llbracket\phi\rrbracket_{v_B}$ and so $\llbracket\neg\phi\rrbracket_{v_B}=1$.
		
		\item For 2', assume $\neg\neg \phi\in B$. Since $B$ is an open branch of a \emph{complete} tableau, every rule that can be applied has been:
		\begin{center}{\small
					\begin{prooftree}
					{
					line numbering=false,
					line no sep= 2cm,
					for tree={s sep'=10mm},
					single branches=true,
					close with=\xmark
					}
					[\neg\neg\phi [\phi ] ]
					\end{prooftree}}
			\end{center} 
		So, $\phi\in B$. But then, by 1., we get $\llbracket\phi\rrbracket_{v_B}=1$. And since $\llbracket\neg\phi\rrbracket_{v_B}=1-\llbracket\phi\rrbracket_{v_B}$, we get $\llbracket\neg\phi\rrbracket_{v_B}=0$, as desired.
		
	\end{itemize}
	
	\end{enumerate}
			
	\end{enumerate}

\end{frame}


\begin{frame}{The Completeness Lemma (Cont'd)}

\begin{enumerate}[(i)]
	
	\setcounter{enumi}{1}
	
	\item \emph{Induction steps}.
	
		\begin{enumerate}[(a)]
			\setcounter{enumii}{1}

				\item We do the case for $\phi\land \psi$.
				
				We have two pairs of induction hypotheses:
		\begin{enumerate}[1$_\phi$.]
	
		\item if $\phi\in B$, then $\llbracket\phi\rrbracket_{v_B}=1$, and 
		\item if $\neg \phi\in B$, then $\llbracket\phi\rrbracket_{v_B}=0$.
	
	\end{enumerate}
	
	And
	
	\begin{enumerate}[1$_\psi$.]
	
		\item if $\psi\in B$, then $\llbracket\psi\rrbracket_{v_B}=1$, and 
		\item if $\neg \psi\in B$, then $\llbracket\psi\rrbracket_{v_B}=0$.
	
	\end{enumerate}
	What we need to prove are:
	\begin{enumerate}[1$_{\phi\land\psi}$.]
	
		\item if $\phi\land \psi\in B$, then $\llbracket\phi\land \psi\rrbracket_{v_B}=1$, and 
		\item if $\neg (\phi\land \psi)\in B$, then $\llbracket\phi\land \psi\rrbracket_{v_B}=0$.
	\end{enumerate}
		
	
		\end{enumerate}
	
	\end{enumerate}
	
\end{frame}

\begin{frame}{The Completeness Lemma (Cont'd)}

	\begin{itemize}
		
			\item Suppose that $\phi\land \psi\in B$. Since $B$ is an open branch of a complete tableau, every rule that can be applied has been applied:
			\begin{center}
				\begin{prooftree}
					{
					line numbering=false,
					line no sep= 2cm,
					for tree={s sep'=10mm},
					single branches=true,
					close with=\xmark
					}
					[\phi\land\psi [\phi [\psi ] ] ]
					\end{prooftree}
				\end{center}
		\item So, we can conclude that both $\phi,\psi\in B$. 
		
		\item But by 1$_\phi$. and 1$_\psi$., this means that $\llbracket\phi\rrbracket_{v_B}=1$ and $\llbracket\psi\rrbracket_{v_B}=1$. 
		
		\item Since $\llbracket\phi\land \psi\rrbracket_{v_B}=min(\llbracket\phi\rrbracket_{v_B},\llbracket\psi\rrbracket_{v_B})$, we get $\llbracket\phi\land \psi\rrbracket_{v_B}=1,$ as desired.
		
	\end{itemize}
		

\end{frame}

\begin{frame}{The Completeness Lemma (Cont'd)}

	\begin{itemize}
		
		\item Next, suppose that $\neg (\phi\land \psi)\in B$. Again, all rules that can be applied have been applied:
		\begin{center}
				\begin{prooftree}
					{
					line numbering=false,
					line no sep= 2cm,
					for tree={s sep'=10mm},
					single branches=true,
					close with=\xmark
					}
					[\neg(\phi\land\psi) [\neg\phi ] [\neg\psi ] ]
					\end{prooftree}
				\end{center}
		
		\item So, we can conclude that either ($\ast$) $\neg\phi\in B$ or ($\ast\ast$) $\neg\psi\in B$.
		
		\begin{enumerate}
		
			\item[($\ast$)] \begin{itemize}
		
		\item If $\neg\phi\in B$, from 2$_\phi$., we get $\llbracket \phi\rrbracket=0$. 
		
		\item Since $\llbracket\phi\land \psi\rrbracket_{v_B}=min(\llbracket\phi\rrbracket_{v_B},\llbracket\psi\rrbracket_{v_B})$, this means we get $\llbracket\phi\land \psi\rrbracket_{v_B}=0$.
		
				\end{itemize}
			\end{enumerate}
			
		\begin{enumerate}
		
			\item[($\ast\ast$)] \begin{itemize}
			\item If $\neg\psi\in B$, from 2$_\psi$., we get $\llbracket \phi\rrbracket=0$. 
		
		\item So, we get $\llbracket\phi\land \psi\rrbracket_{v_B}=0$.
		
		\end{itemize}
		
		\end{enumerate}
				
		\item So either way,  $\llbracket\phi\land \psi\rrbracket_{v_B}=0$, as desired.
	
	
	\end{itemize}

\end{frame}


\begin{frame}{The Completeness Theorem}

(7.3.3) \textbf{Theorem} (Completeness)\textbf{.} If $\Gamma\vDash\phi$, then $\Gamma\vdash\phi$.

\emph{Proof}:

	\begin{itemize}
		
		\item Assume that $\Gamma\nvdash\phi$.
		
		\item By definition, this means that there's an open branch, $B$, in the complete tableau for $\Gamma\cup\{\neg\phi\}$.
		
		\item Consider the associated valuation $v_B$.
		
		\item By the completeness lemma $v_B$ makes all the members of $B$ true.
		
		\item Since $\Gamma\cup\{\neg\phi\}\subseteq B$, it follows that $v_B$ satisfies $\Gamma\cup\{\neg\phi\}$.
		
		\item So $\Gamma\nvDash\phi$, as desired.
	
	\end{itemize}

\end{frame}

\begin{frame}{Tableau Verification 2}

	(7.3.4) \textbf{Theorem.} If the tableau algorithm gives the answer that a set is unsatisfiable, then the set \emph{is} unsatisfiable. 

	\emph{Proof}:
	
	\begin{itemize}
	
		\item Suppose that the algorithm says $\Gamma$ is satisfiable.
		
		\item So, the tableau for $\Gamma$ is open.
				
		\item We can infer that there's a branch $B$ in the tableau for $\Gamma$ which is open. 
	
		\item By the completeness lemma $v_B$ makes all the members of $B$ true..
		
		\item Since $\Gamma\subseteq B$, we can infer that $v_B$ satisfies $\Gamma$ as desired.
	
	\end{itemize}
	
	(7.3.5) \textbf{Theorem} (Decidability)\textbf{.} 	Propositional logic is decidable, i.e. there exists an algorithm which after finitely many steps correctly determines whether a given inference (with finitely any premises) is valid.


\end{frame}

\subsection{7.4 Infinite Premiss Sets and Compactness}

\begin{frame}{7.4 Infinite Premiss Sets and Compactness}

	\begin{itemize}
	
		\item (7.4.1) For the formal treatment of mathematics, we sometimes need infinite premise sets:
		\begin{itemize}
		
			\item The theory of natural numbers cannot be reduced to finitely many axioms.
		
		\end{itemize}
		
		\item (7.4.2) We'll go infinite, but we assume there is a way of writing $\Gamma$ as the set $\{\phi_i:i\in I\}$, where $I\subseteq\mathbb{N}^+$.
		
		\item (7.4.3) Examples:
		
			\begin{enumerate}[(a)]
		
			\item $\{p, \neg p, \neg\neg p, \mathellipsis\}$ (or more precisely the smallest set $X$ such that $p\in X$ and if $\phi\in X$ then $\neg \phi\in X$).
			
			
			\item $\mathcal{P}=\{p_i: i\in \mathbb{N}\}$
						
			\item $\{\neg p_{2i}, p_{2i+1}: i\in\mathbb{N}\}$ which contains $\neg p_i$ for each even $i$ and $p_i$ for each odd $i\in\mathbb{N}$.
			
			\item Let $v$ be any valuation. Then the set $T_v=\{\phi:\llbracket\phi\rrbracket_v=1\}$ is always infinite! 
					
		\end{enumerate}
		
		\item (7.4.4) $\Gamma\vDash\phi$ works just like before.
	
	\end{itemize}

\end{frame}

\begin{frame}{Truth-Tables Don't Work}
	
	\begin{itemize}
	
		\item (7.4.5) But truth-tables no longer work:
		
			\begin{itemize}
			
				\item With finite $\Gamma$, we can use: \[\phi_1,\mathellipsis, \phi_n\vDash \psi\text{ iff }\vDash (\phi_1\land\mathellipsis\land\phi_n)\to\psi\]
				
				\item But with infinite $\Gamma$, there is no theorem:
				
				\[\phi_1,\phi_2,\mathellipsis\vDash \psi\text{ iff }\vDash \underbrace{(\phi_1\land\phi_2\land\mathellipsis)}_{???}\to\psi\]
			
			\end{itemize}
	
	\end{itemize}

\end{frame}

\begin{frame}{(7.4.6) Infinite Tableaux}

	\begin{itemize}
	
		\item  Tableaux, however, work.
		
		\item To make the tableau for $\{\phi_1,\phi_2,\mathellipsis,\neg\psi\}$ write down $\neg \psi$ as the initial list and do its tableau.
		
		\item Then, add $\phi_1$ to the initial list, and extend the tableau for $\neg\psi$ to the tableau for $\{\phi_1,\neg\phi\}$.
		
		\item Then add $\phi_2$, and extend the tableau to the one for $\{\phi_1,\phi_2,\neg\phi\}$.
		
		\item And so on.
		
		\item $\Gamma\vdash\phi$ iff the complete tableau is closed, $\Gamma\nvdash\phi$ iff the complete (?) tableau is open.
		
		\item Note that a complete tableau (which is open) can be infinite!
	
	\end{itemize}

\end{frame}

\begin{frame}{Example}

$p, \neg\neg p, \neg\neg \neg\neg p, \mathellipsis\nvdash q$

\begin{center}
				\begin{prooftree}
					{
					line numbering=false,
					line no sep= 2cm,
					for tree={s sep'=10mm},
					single branches=true,
					close with=\xmark
					}
					[\vdots, grouped [\neg\neg\neg\neg p, grouped [\neg\neg p, grouped [p, grouped [\neg q, grouped [p [ \neg \neg p [p [\vdots ]]] ] ] ]  ] ] ]
					\end{prooftree}
				\end{center}


\end{frame}

\begin{frame}{Compactness}

But if a tableau closes, it closes after finitely many steps (``at some point''):

\textbf{Theorem} (Compactness)\textbf{.} Let $\Gamma$ be an infinite set of formulas. If $\Gamma\vdash\phi$, then there exists a finite set $\Sigma\subseteq\Gamma$ such that $\Sigma\vdash\phi$. 

\emph{Proof}:

	\begin{itemize}
	
		\item Let $\Gamma=\{\psi_i: i\in I\}$ for some $I\subseteq\mathbb{N}^+$.
	
		\item Assume that $\Gamma\vdash\phi$. 
		
		\item This means that the tableau closes at some point.
		
		\item So there is a smallest $i$ such that $\psi_j$ is in the tableau for all $j\leq i$.
		
		\item We've actually made the tableau for $\{\neg\phi\}\cup\{\psi_j: j\leq i\}$.
		
		\item So $\{\psi_j: j\leq i\}\vdash\phi$.
		
		\item But clearly $\{\psi_j: j\leq i\}$ is finite and $\{\psi_j: j\leq i\}\subseteq\Gamma$.
	
	\end{itemize}


\end{frame}

\begin{frame}{Core Ideas (Lecture Version)}


\begin{itemize}

	\item Soundness and completeness reduce validity to syntax.
	
	\item The soundness theorem is a ``sanity check.''
	
	\item The completeness theorem is a surprising mathematical fact.
		
	\item For tableaux, soundness relies on the fact that in every rule, if the upper formula is true, then at least one of the lower formulas is true (`down preservation').
	
	\item For tableaux, completeness relies on the fact that if one of the lower formulas in a rule is true, so is the upper formula (`up preservation').
	
	\item Truth tables don't work for infinitary premise sets but tableaux do.
	
	\item Compactness tells us that if there's a proof, we can always find a finitary one.

\end{itemize}

\end{frame}



\begin{frame}

	\begin{center}
	{\huge\bf Thanks!}
	\end{center}

\end{frame}

