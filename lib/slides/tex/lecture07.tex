
\begin{frame}
  \setcounter{framenumber}{190}
  \LectureNo{7}
  \maketitle
\end{frame}

\begin{frame}{Overview}
\tableofcontents
\end{frame}

\section{Rehash}
\begin{frame}{Rehash}
	
	\begin{itemize}
	\itemsep=16pt

	\item In PL, an inference is provable by the tableaux algorithm iff the premises and negated conclusion are unsatisfiable.

	\bigskip
	
		\begin{description}
		\itemsep=16pt			

		\item[Down Preservation.] If the formula at the parent node of a rule is true under valuation $v$, the formulas on at least one child node are true under $v$.
				
		\item[Up Preservation.] If the formulas at a child node generated by a rule are true under valuation $v$, the formula at the parent node is also true under $v$.
				
		\end{description}
						
	\end{itemize}

\end{frame}
		
\begin{frame}{Today's Goal}

	\begin{itemize}%[<+->]
	\itemsep=16pt
		
        \item Today, we prove soundness and completeness for PL.
        
        \item Our proof will, effectively, use down-preservation to establish soundness and up-preservation to establish completeness.
          
	\end{itemize}

\end{frame}

\begin{frame}{The Big Picture}

\begin{itemize}%[<+->]
\itemsep=16pt
		
\item Soundness and Completeness show that our proof theory (tableau) aligns with our logical semantics (valuations).

\item It shows that validity, which we define by truth and valuations, can be reduced to a property of tableaux, using nothing but syntactic operations on formulas. Interesting!

\end{itemize} 

\end{frame}

\section{Soundness and Completeness for PL}
\subsection{What is Soundness?}
\begin{frame}{What is Soundness?}

	\begin{itemize}%[<+->]
	\itemsep=16pt
		
        \item The \textbf{Soundness Theorem}: If $\Gamma\vdash\phi$, then $\Gamma\vDash\phi$
	\\
	\alert{\emph{\small{any inference that is provable by tableaux is also semantically valid}}}
	\\
	{\small contrapositive: If $\Gamma\nvDash\phi$ (countermodel), then $\Gamma\nvdash\phi$ (open tableau)}
	
	\item A proof procedure that is sound passes a sort of `sanity check', i.e. we know that it cannot derive any invalid inferences.
	
	\end{itemize}

\end{frame}

\begin{frame}{Proof Sketch}

\begin{itemize}%[<+->]
\itemsep=16pt

\item We first prove the \textbf{Soundness Lemma}: 
If $B$ is a branch of a PL tableau with faithful valuation $v$, when a rule is applied to this branch, $v$ is faithful to at least one of its extensions $B'$.

\smallskip

	\begin{itemize}
	\item[] Where a valuation $v$ (truth-values) is said to be \emph{faithful} to a given branch $B$ (set of formulas) iff $\llbracket\phi\rrbracket_v=1$ for all $\phi\in B$
	\end{itemize}

\item The theorem follows immediately from this lemma.
	\begin{itemize}%[<+->]
	\item If $\Gamma\nvDash\phi$, then this inference has a countermodel $v$.
	\item $v$ is faithful to the initial list of the tableau for $\Gamma\cup\{\neg\phi\}$.
	\item From the SL: each time a rule is applied to this tableau, there exists a model faithful to at least one branch.
	\item So, there is a model faithful to a branch of the finished tableau.
	\item This must be an open branch \alert{(why?)}, which shows that $\Gamma\nvdash\phi$.
	\end{itemize}

\end{itemize}

\end{frame}

\subsection{Proving the Soundness Lemma}
\begin{frame}{Proving the Soundness Lemma}

\begin{itemize}%[<+->]
\itemsep=16pt

\item To establish this important lemma, we go through the tableaux rules one by one and check that the claim is true for each rule.

\item As an example, we will discuss in detail how the reasoning works in the case of positive and negative implication rules.
	
\end{itemize}

\end{frame}

\begin{frame}{Proving the Soundness Lemma}

\begin{enumerate}[(a)]
\itemsep=16pt
					
		\item Let $v$ be faithful to $B$, with $\phi\to\psi\in B$, and consider the following extension by the positive implication rule.
		
					\begin{center}
					\begin{prooftree}
					{
					line numbering=false,
					line no sep= 2cm,
					for tree={s sep'=10mm},
					single branches=true,
					close with=\xmark
					}
					[\phi\to\psi [\neg \phi ] [\psi ] ]
					\end{prooftree}
					\end{center}
					\medskip
								
		\begin{itemize}
		\itemsep=16pt

									
			\item Since $v$ is faithful to $B$, $\llbracket\phi\to\psi\rrbracket_v=max(1-\llbracket\phi\rrbracket_v, \llbracket\psi\rrbracket_v)=1$. 
			
			\item It follows that $\llbracket\phi\rrbracket_v=0$ or $\llbracket\psi\rrbracket_v=1$.
			
			\medskip
			
		\begin{itemize}
		
			\item If $\llbracket\phi\rrbracket_v=0$ , then $\llbracket\neg\phi\rrbracket_v=1$, which means that $v$ is faithful to the left extension of $B$.

			\item If $\llbracket\psi\rrbracket_v=1$ , then $v$ is faithful to the right extension of $B$.	
		
			\end{itemize}

		\item Either way, $v$ is faithful to at least one new branch.
	
		\end{itemize}
		
\end{enumerate}
	
\end{frame}

\begin{frame}{Proving the Soundness Lemma}
	
	\begin{enumerate}[(a)]

	\setcounter{enumi}{1}
					
		\item Let $v$ be faithful to $B$, with $\neg(\phi\to\psi)\in B$, and consider the following extension by the negative implication rule.
				\begin{center}{\small
					\begin{prooftree}
					{
					line numbering=false,
					line no sep= 2cm,
					for tree={s sep'=10mm},
					single branches=true,
					close with=\xmark
					}
					[\neg(\phi\to\psi) [\phi [\neg\psi ] ] ]
					\end{prooftree}}
					\end{center}
					\medskip
								
		\begin{itemize}
		\itemsep=16pt

									
			\item We have $\llbracket\neg(\phi\to\psi)\rrbracket_v=1-max(1-\llbracket\phi\rrbracket_v, \llbracket\psi\rrbracket_v)=1$. 

			\item It follows that $\llbracket\phi\to\psi\rrbracket_v=max(1-\llbracket\phi\rrbracket_v, \llbracket\psi\rrbracket_v)=0$. 
			
			\item So, we have both $1-\llbracket\phi\rrbracket_v=0$ and $\llbracket\psi\rrbracket_v=0$, hence we have $\llbracket\phi\rrbracket_v=1$ and $\llbracket\neg\psi\rrbracket_v=1$.
			
		\item Thus, $v$ is faithful to the new branch.
	
		\end{itemize}
					
	\end{enumerate}

\end{frame}

\begin{frame}{Examples}

\begin{itemize}
\itemsep=16pt

	\item Here are two valuations to illustrate how faithfulness can be preserved over a tableau in different ways (remember: we showed that it is always preserved in \emph{at least one} extension)
	
	\item $v_1$ with $v_1(p)=1$ and $v_1(q)=0$ is faithful to the top node
	\item $v_2$ with $v_2(p)=0$ and $v_2(q)=1$ is faithful to the top node
	
	\end{itemize}
	
	\bigskip

	\only<1|handout:0>{\begin{center}
		\begin{prooftree}
					{
					line numbering=false,
					line no sep= 1cm,
					for tree={s sep'=5mm},
					single branches=true,
					close with=\xmark
					}
					[p\lor q, grouped [\neg p\lor \neg q, grouped ] ]
					\end{prooftree}
		\end{center}}
	\only<2|handout:0>{\begin{center}
		\begin{prooftree}
					{
					line numbering=false,
					line no sep= 1cm,
					for tree={s sep'=5mm},
					single branches=true,
					close with=\xmark
					}
					[p\lor q, grouped [\neg p\lor \neg q, grouped [p] [q]  ] ]
					\end{prooftree}
		\end{center}}
	\only<3|handout:0>{\begin{center}
		\begin{prooftree}
					{
					line numbering=false,
					line no sep= 1cm,
					for tree={s sep'=5mm},
					single branches=true,
					close with=\xmark
					}
					[p\lor q, grouped [\neg p\lor \neg q, grouped [p [\neg p] [\neg q] ] [q [\neg p] [\neg q] ]  ] ]
					\end{prooftree}
		\end{center}}
	\only<4|handout:1>{\begin{center}
		\begin{prooftree}
					{
					line numbering=false,
					line no sep= 1cm,
					for tree={s sep'=5mm},
					single branches=true,
					close with=\xmark
					}
					[p\lor q, grouped [\neg p\lor \neg q, grouped [p [\neg p, close] [\neg q] ] [q [\neg p] [\neg q, close] ]  ] ]
					\end{prooftree}
		\end{center}}


\end{frame}


\begin{frame}{Tableau Verification 1}

(7.2.6) \textbf{Theorem.} If the tableau algorithm indicates that a set is unsatisfiable (closed tableau), then the set \emph{really is} unsatisfiable. 
We can now easily explain why this works as intended.

\bigskip

\begin{itemize}
\itemsep=16pt
	
		\item Suppose that the tableau for $\Gamma$ is closed.
		
		\item Suppose, for contradiction, that $\Gamma$ is satisfiable via some valuation $v$.
		
		\item Using the Soundness Lemma, we can infer that there's a branch $B$ in the tableau for $\Gamma$ which is open.
	
		\item Hence the tableau for $\Gamma$ should be open. Contradiction.
		
		\item Thus, in fact, $\Gamma$ is unsatisfiable.
	
	\end{itemize}

\end{frame}


\subsection{What is Completeness?}
\begin{frame}{What is Completeness?}

	\begin{itemize}%[<+->]
	\itemsep=16pt
		
	\item The \textbf{Completeness Theorem}: If $\Gamma\vDash\phi$, then $\Gamma\vdash\phi$
	\\
	\alert{\emph{\small{any inference that is semantically valid is also provable by tableaux}}}
	\\
	{\small contrapositive: If $\Gamma\nvdash\phi$ (open tableau), then $\Gamma\nvDash\phi$ (countermodel)}

	\item A complete proof theory derives absolutely all valid inferences (in principle, this is trivial, but not if we \emph{also} want soundness).

	\end{itemize}

\end{frame}

\begin{frame}{Proof Sketch}

	\begin{itemize}%[<+->]
	\itemsep=16pt

	\item We first prove the \textbf{Completeless Lemma}: If $B$ is an open branch of a finished tableau, the associated valuation $v_B$ is faithful to this branch.
	
	\smallskip

		\begin{itemize}
		\item[] Where the associated valuation $v_B$ is defined by setting:\[v_B(p):=\begin{cases} 1 &\text{if }p\in B\\0&\text{if }p\notin B\end{cases}\]	
		\end{itemize}
	
	\item The theorem follows immediately from this lemma.
		\begin{itemize}%[<+->]
		\item If $\Gamma\nvdash\phi$, then its finished tableau has an open branch $B$.
		\item From the CL: the associated valuation $v_B$ is faithful to $B$.
		\item In that case, $v_B$ is faithful to the initial list $\Gamma\cup\{\neg\phi\}$.
		\item So, there is a valuation where $v_B(\Gamma)=1$ and $v_B(\phi)=0$.
		\item Which shows what we wanted, namely $\Gamma\nvDash\phi$
		\end{itemize}
                  
	\end{itemize}

\end{frame}



\subsection{Proving the Completeness Lemma}
\begin{frame}{Proving the Completeness Lemma}

\begin{itemize}
\itemsep=16pt
		
	\item The proof strategy for the lemma is simple. 
	If $\Gamma\nvdash\phi$, then the tableau for $\Gamma\cup\{\neg\phi\}$ is open via some branch $B$.
				
	\item We get the associated valuation $v_B$ and use up-preservation to prove the following (by induction over all types of formulas).
	\begin{enumerate}[1.]
		\item if $\phi\in B$, then $\llbracket\phi\rrbracket_{v_B}=1$, and 
		\item if $\neg \phi\in B$, then $\llbracket\phi\rrbracket_{v_B}=0$.
	
	\end{enumerate}

\end{itemize}	

\end{frame}

\begin{frame}{Proving the Completeness Lemma}

\begin{enumerate}[(i)]
\itemsep=16pt
	
\item \emph{Base case}. We need to show that for all atomic $p\in \mathcal{P}$:
	\begin{enumerate}[1.]
	 	\item If $p\in B$, then $\llbracket p\rrbracket_{v_B}=1$
		\item If $\neg p\in B$, then $\llbracket p\rrbracket_{v_B}=0$.
	\end{enumerate}

	\bigskip
	
	\begin{itemize}
	\itemsep=16pt

		\item 1. is immediate from the definition of $v_B$. 
		
		\item For 2. note that if $\neg p\in B$, then $p\notin B$ since $B$ is open. So, by definition, $v_B(p)=0$, and thus $\llbracket p\rrbracket_{v_B}=0$ as desired.
		
		\end{itemize}
		
\end{enumerate}

\end{frame}

\begin{frame}{Proving the Completeness Lemma}

\begin{enumerate}[(i)]
\setcounter{enumi}{1}
	
\item \emph{Induction steps}.
	
\medskip
	
	\begin{enumerate}[(a)]
	\itemsep=10pt

  	\item Now, assuming the claim holds for $\phi$:
		
				\begin{enumerate}[1.]
	
					\item If $\phi\in B$, then $\llbracket\phi\rrbracket_{v_B}=1$. 
	
					\item If $\neg \phi\in B$, then $\llbracket\phi\rrbracket_{v_B}=0$.
	
				\end{enumerate}

	\item We show that it must then hold for $\neg\phi$:
		\begin{enumerate}[1'.]
	
		\item If $\neg\phi\in B$, then $\llbracket\neg\phi\rrbracket_{v_B}=1$.
			\item If $\neg\neg \phi\in B$, then $\llbracket\neg\phi\rrbracket_{v_B}=0$.	
		\end{enumerate}
		
	\begin{itemize}
	\itemsep=10pt

	\medskip
	
	\item For 1', suppose that $\neg \phi\in B$. By 2., we have $\llbracket\phi\rrbracket_{v_B}=0$. But $\llbracket\neg\phi\rrbracket_{v_B}=1-\llbracket\phi\rrbracket_{v_B}$ and so $\llbracket\neg\phi\rrbracket_{v_B}=1$.
		
	\item For 2', assume $\neg\neg \phi\in B$. Since $B$ is a branch of a \emph{finished} tableau, every rule that can be applied has been, including:
		\begin{center}{\small
					\begin{prooftree}
					{
					line numbering=false,
					line no sep= 2cm,
					for tree={s sep'=10mm},
					single branches=true,
					close with=\xmark
					}
					[\neg\neg\phi [\phi ] ]
					\end{prooftree}}
		\end{center} 
		\medskip
		
		So, $\phi\in B$. But then, by 1., $\llbracket\phi\rrbracket_{v_B}=1$ and so $\llbracket\neg\phi\rrbracket_{v_B}=0$.
		
	\end{itemize}
	
	\end{enumerate}
			
	\end{enumerate}

\end{frame}


\begin{frame}{Proving the Completeness Lemma}

\begin{enumerate}[(i)]
	
	\setcounter{enumi}{1}
	
	\item \emph{Induction steps}.
	
		\begin{enumerate}[(a)]
		\setcounter{enumii}{1}

		\item The steps for the rest of the rules all follow a similar pattern.
		For example, for $\phi\land \psi$ we get two induction hypotheses:
		\begin{enumerate}[1$_\phi$.]
	
		\item if $\phi\in B$, then $\llbracket\phi\rrbracket_{v_B}=1$, and 
		\item if $\neg \phi\in B$, then $\llbracket\phi\rrbracket_{v_B}=0$.
	
	\end{enumerate}
	
	And
	
	\begin{enumerate}[1$_\psi$.]
	
		\item if $\psi\in B$, then $\llbracket\psi\rrbracket_{v_B}=1$, and 
		\item if $\neg \psi\in B$, then $\llbracket\psi\rrbracket_{v_B}=0$.
	
	\end{enumerate}
	
	\medskip
	
	What we need to prove are:
	\begin{enumerate}[1$_{\phi\land\psi}$.]
	
		\item if $\phi\land \psi\in B$, then $\llbracket\phi\land \psi\rrbracket_{v_B}=1$, and 
		\item if $\neg (\phi\land \psi)\in B$, then $\llbracket\phi\land \psi\rrbracket_{v_B}=0$.
	\end{enumerate}
		
	
		\end{enumerate}
	
	\end{enumerate}
	
\end{frame}

\begin{frame}{Proving the Completeness Lemma}

\begin{itemize}
\itemsep=16pt

		
			\item Suppose that $\phi\land \psi\in B$. Since $B$ is a branch of a finished tableau, every rule that can be applied has been applied:
			\begin{center}
				\begin{prooftree}
					{
					line numbering=false,
					line no sep= 2cm,
					for tree={s sep'=10mm},
					single branches=true,
					close with=\xmark
					}
					[\phi\land\psi [\phi [\psi ] ] ]
					\end{prooftree}
				\end{center}
		\item So, we can conclude that both $\phi,\psi\in B$. 
		
		\item But by 1$_\phi$. and 1$_\psi$., this means that $\llbracket\phi\rrbracket_{v_B}=1$ and $\llbracket\psi\rrbracket_{v_B}=1$. 
		
		\item So, we have $\llbracket\phi\land \psi\rrbracket_{v_B}=1,$ as desired.
		
	\end{itemize}
		

\end{frame}

\begin{frame}{Proving the Completeness Lemma}

\begin{itemize}
\itemsep=16pt

		
		\item Next, suppose that $\neg (\phi\land \psi)\in B$. Again, all rules that can be applied have been applied:
		\begin{center}
				\begin{prooftree}
					{
					line numbering=false,
					line no sep= 2cm,
					for tree={s sep'=10mm},
					single branches=true,
					close with=\xmark
					}
					[\neg(\phi\land\psi) [\neg\phi ] [\neg\psi ] ]
					\end{prooftree}
				\end{center}
		
		\item So, we can conclude that either $\neg\phi\in B$ or $\neg\psi\in B$.	
		\medskip
		
		\begin{itemize}
		
		\item If $\neg\phi\in B$, from 2$_\phi$., we get $\llbracket \phi\rrbracket=0$ and $\llbracket\phi\land \psi\rrbracket_{v_B}=0$.
		

		\item If $\neg\psi\in B$, from 2$_\phi$., we get $\llbracket \psi\rrbracket=0$ and $\llbracket\phi\land \psi\rrbracket_{v_B}=0$.
		
		\end{itemize}
						
	\item Either way,  $\llbracket\phi\land \psi\rrbracket_{v_B}=0$, as desired.
	
	\end{itemize}

\end{frame}


\begin{frame}{Tableau Verification 2}

(7.3.4) \textbf{Theorem.} If the tableau algorithm indicates that a set is satisfiable (finished open tableau), then the set \emph{really is} satisfiable. 
We can now easily explain why this works as intended.
	
\bigskip

\begin{itemize}
\itemsep=16pt
	
		\item Suppose that the finished tableau for $\Gamma$ is open.
		
		\item Then there is an open branch $B$ in this tableau.
	
		\item By the Completeness Lemma, $B$ is satisfied by $v_B$.
		
		\item Since $\Gamma\subseteq B$, it follows that $\Gamma$ is satisfied by $v_B$.
	
	\end{itemize}

\end{frame}


\subsection{Decidability}
\begin{frame}{Decidability}
	
	(7.3.5) \textbf{Theorem} (Decidability)\textbf{.}
	Propositional logic is decidable, i.e. there is an algorithm that determines in finitely many steps whether a given inference (with finite premises) is valid.

\end{frame}

\subsection{Infinite Premiss Sets and Compactness}

\begin{frame}{Infinite Premiss Sets and Compactness}

\begin{itemize}
\itemsep=16pt

	
		\item We sometimes need infinite premise sets, e.g. the formal theory of natural numbers cannot be reduced to finitely many axioms.
				
		\item Assume countability, i.e. infinite $\Gamma=\{\phi_i:i\in I\}$, for $I\subseteq\mathbb{N}^+$.
		
		\item Examples:
		
			\begin{enumerate}[(a)]
		
			\item $\{p, \neg p, \neg\neg p, \mathellipsis\}$		
			
			\item $\mathcal{P}=\{p_i: i\in \mathbb{N}\}$
			
			\item $T_v=\{\phi:\llbracket\phi\rrbracket_v=1\}$ for any valuation $v$.
					
		\end{enumerate}
			
	\end{itemize}

\end{frame}

\begin{frame}{Truth-Tables Don't Work}
	
\begin{itemize}
\itemsep=16pt

\item Truth-tables no longer work with infinite premises.
		
\medskip
		
\item For finite $\Gamma$, recall, truth-tables rely on the fact: 
\[\phi_1,\mathellipsis, \phi_n\vDash \psi\text{ iff }\vDash (\phi_1\land\mathellipsis\land\phi_n)\to\psi\]
				
\item But with infinite $\Gamma$, this no longer makes sense.
				
\[\phi_1,\phi_2,\mathellipsis\vDash \psi\text{ iff }\vDash \underbrace{(\phi_1\land\phi_2\land\mathellipsis)}_{???}\to\psi\]
			
\end{itemize}

\end{frame}

\begin{frame}{Infinite Tableaux}

\begin{itemize}
\itemsep=16pt

	
		\item Tableaux, however, still work for infinite $\{\phi_1,\phi_2,\mathellipsis,\neg\psi\}$.
		
		\item Write down $\neg\psi$ as the initial list and construct its tableau.
		If that remained open, add $\phi_1$ to the initial list and extend the tableau for $\neg\psi$ to the tableau for $\{\phi_1,\neg\phi\}$.
		Repeat for each $\phi_i$.
		
		\item $\Gamma\vdash\phi$ iff one of these tableau closes.
		
		\item Note, a \emph{finished} tableau of this kind can be infinite!
	
	\end{itemize}

\end{frame}

\begin{frame}{Example}

\begin{center}

$p, \neg\neg p, \neg\neg\neg\neg p, \mathellipsis\nvdash q$

\bigskip\bigskip

				\begin{prooftree}
					{
					line numbering=false,
					line no sep= 2cm,
					for tree={s sep'=10mm},
					single branches=true,
					close with=\xmark
					}
					[\vdots, grouped [\neg\neg\neg\neg p, grouped [\neg\neg p, grouped [p, grouped [\neg q, grouped [p [ \neg \neg p [p [\vdots ]]] ] ] ]  ] ] ]
					\end{prooftree}
				\end{center}

\end{frame}

\begin{frame}{Compactness}

Interestingly enough, if a tableau closes, it still closes after finitely many steps, which leads to another important result.

\bigskip

\textbf{Theorem} (Compactness)\textbf{.} Let $\Gamma$ be infinite. If $\Gamma\vdash\phi$, then there exists a finite set $\Sigma\subseteq\Gamma$ such that $\Sigma\vdash\phi$. 

\bigskip

\begin{itemize}
\itemsep=16pt

	
		\item Represent $\Gamma=\{\psi_i: i\in I\}$.
			
		\item By hypothesis, the tableau for $\Gamma\cup\{\neg\phi\}$ closes at some point.
		
		\item If so, then there must be a smallest $i$ such that $\psi_j$ is in the tableau for all $j\leq i$.
		But that means our closed tableau only used $\{\psi_j: j\leq i\}\cup\{\neg\phi\}$, which shows that $\{\psi_j: j\leq i\}\vdash\phi$.
				
		\item And, clearly, $\{\psi_j: j\leq i\}$ is finite and $\{\psi_j: j\leq i\}\subseteq\Gamma$.
	
	\end{itemize}


\end{frame}

\begin{frame}{Core Ideas (Lecture Version)}

\begin{itemize}
\itemsep=16pt

\item Soundness and Completeness reduce validity to syntax.
	
\item The inter-connection of proof theory and semantics works because of down-preservation over rules, and up-preservation that gives structure to the associated valuation of a branch.
	
\item With infinite premises, tableaux work, but truth tables are out.

\item Any tableau proof has a finite counterpart, via Compactness.

\end{itemize}

\end{frame}



\begin{frame}

	\begin{center}
	{\huge\bf Thanks!}
	\end{center}

\end{frame}
