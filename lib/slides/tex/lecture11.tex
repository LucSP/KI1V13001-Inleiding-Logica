\setcounter{framenumber}{346}
\begin{frame}
  \LectureNo{11}
  \maketitle
\end{frame}

\begin{frame}{Overview}
\tableofcontents
\end{frame}

\section{Rehash}
\begin{frame}{Rehash}
	
\begin{itemize}
	
		\item Tableaux for first-order logic work in similar
                  way as propositional tableaux. Underlying is always
                  the idea that an inference is valid iff the premises
                  and negation of conclusion are jointly
                  unsatisfiable.
                  
		\item We have three tiers of tableau systems: simple
                  tableaux, tableaux with identity, and tableaux with
                  identity and functions. We successively add
                  rules for those.
		
		\item There is also a concept of an associated model,
                  which in first-order logic is a \emph{term model},
                  i.e. a model based on the syntactic strings as the
                  objects in the domain.
		
		\item First-order tableaux can turn out to be
                  infinite, in which case we need to \emph{prove} that
                  they are open and complete.
		
		\item The existence of infinite tableaux destroys our
                  hopes for an easy decidability proof.
		
		\item Alas, first-order logic is actually provably
                  undecidable.
		
					
\end{itemize}

\end{frame}
		

\section{11. Soundness and Completeness}
\subsection{11.1 Overview}

\begin{frame}{11.1 Overview}

	\begin{itemize}
		
        \item (11.1.1) We're now going to prove soundness and
          completeness for first-order logic. And it's going to be
          \emph{hard}.

          \item (11.1.2) The \emph{idea}, however, is essentially the
            same as in propositional logic:
            \begin{itemize}

            \item Soundness follows from down preservation.

            \item Completeness follows from up preservation.

            \end{itemize}

          \item (11.1.3) Infinite tableaux are not really a problem.

           \item In this lecture version of the proofs, I will ignore
             $=$ and function symbols, hopefully making the proofs
             more perspicuous.

            \item We still need to prove to auxilliary lemmas.
          
	\end{itemize}

\end{frame}

\subsection{11.2 The Denotation and Locality Lemma}
\begin{frame}{11.2 The Denotation and Locality Lemma}

  \begin{itemize}

  \item (11.2.1) The denotation lemmma says (in a sense) that substitution and
    variable fixing are equivalent:
    \[\mathcal{M},\alpha\vDash (P(x))[x:=t]\text{ iff
      }\mathcal{M},\alpha[x\mapsto\llbracket
      t\rrbracket^\mathcal{M}_\alpha]\vDash P(x)\]
    In words, replacing $x$ with $t$ is the same as setting the value
    of $x$ to the value of $t$.

    \item (11.2.2) \textbf{Lemma} (Denotation Lemma)\textbf{.}
      Consider a formula $\phi$ with precisely one free
            variable, $x$, and an arbitrary ground term $t$. Further,
            let $\mathcal{M}$ be a model and
            $\alpha$ an assignment. Then we have:
            \[\mathcal{M},\alpha\vDash (\phi)[x:=t]\text{ iff
              }\mathcal{M},\alpha[x\mapsto \llbracket
              t\rrbracket^\mathcal{M}_\alpha]\vDash \phi\]
    
  \end{itemize}
  
\end{frame}

\begin{frame}{Proof of the Denotation Lemmma (Sketch)}

  \begin{itemize}

  \item We prove this by induction. I sketch the proof, details are in
    the reader.

   \item For the base case, consider $R(t_1, \mathellipsis, t,
     \mathellipsis, t_n)$.

     \item This is of the form $(R(t_1, \mathellipsis,x, \mathellipsis,
       t_n))[x:=t]$.

       \item Note that all of $t_1, \mathellipsis, t_n$ are ground
         terms.

         \item We have: {\small\[\mathcal{M},\alpha\vDash R(t_1,
               \mathellipsis, t, 
               \mathellipsis, t_n)\text{ iff }(\llbracket
                t_1\rrbracket^\mathcal{M}_\alpha,\mathellipsis,
                \llbracket
                t\rrbracket^\mathcal{M}_\alpha,\mathellipsis,\llbracket
                t_n\rrbracket^\mathcal{M}_\alpha)\in
                R^\mathcal{M}\]}

            \item Since the $t_i$ are ground terms, their values are
              not affected by changing the value of $x$.

              \item By definition, $\alpha[x\mapsto \llbracket
                t\rrbracket^\mathcal{M}_\alpha](x)=\llbracket
                t\rrbracket^\mathcal{M}_\alpha$.

              \item Together, these two observations give us:
                 {\scriptsize\[\mathcal{M},\alpha[x\mapsto \llbracket
                t\rrbracket^\mathcal{M}_\alpha]\vDash R(t_1,
               \mathellipsis, x, 
               \mathellipsis, t_n)\text{ iff }(\llbracket
                t_1\rrbracket^\mathcal{M}_\alpha,\mathellipsis,
                \llbracket
                t\rrbracket^\mathcal{M}_\alpha,\mathellipsis,\llbracket
                t_n\rrbracket^\mathcal{M}_\alpha)\in
                R^\mathcal{M}\]}
            
  \end{itemize}
  
\end{frame}

\begin{frame}{Proof Cont'd}

  \begin{itemize}

  \item For the induction step, we assume that the claim holds for
    $\phi$:
    \[\mathcal{M},\alpha\vDash (\phi)[x:=t]\text{ iff
              }\mathcal{M},\alpha[x\mapsto \llbracket
              t\rrbracket^\mathcal{M}_\alpha]\vDash \phi\]

   \item We derive that it holds for $\forall y\phi$:
            \[\mathcal{M},\alpha\vDash (\forall
              y\phi)[x:=t]\text{ iff }
              \mathcal{M},\alpha[x\mapsto \llbracket
              t\rrbracket^\mathcal{M}_\alpha]\vDash \forall
              y\phi\]

    \item Note that $(\forall y\phi)[x:=t]=\forall y(\phi)[x:=t]$.

    \item We know that $\mathcal{M},\alpha\vDash \forall
      y(\phi)[x:=t]$ iff for all $d\in
      D^\mathcal{M}$: \[\mathcal{M},\alpha[y\mapsto d]\vDash
        (\phi)[x:=t]\]

    \item By the induction hypothesis, this is equivalent to:
      \[\mathcal{M},\alpha[y\mapsto d, x\mapsto \llbracket
        t\rrbracket^\mathcal{M}_\alpha]\vDash \phi\]

      \item So we get $\mathcal{M},\alpha[x\mapsto \llbracket
              t\rrbracket^\mathcal{M}_\alpha]\vDash \forall
              y\phi$, as desired.
    
  \end{itemize}
  
\end{frame}

\begin{frame}{(11.2.3) Use of the Denotation Lemmma}

  \begin{itemize}
  \item We can use the denotation lemma to show that $\forall
    x\phi\vDash(\phi)[x:=t]$

  \item If everything is $\phi$, then $t$ is $\phi$.

  \item Suppose that $\mathcal{M},\alpha\vDash\forall x\phi$.

    \item It follows that for each $d\in D^\mathcal{M}$, we have
      \[\mathcal{M},\alpha[x\mapsto d]\vDash \phi.\]

    \item Just set the value of $x$ to the value of $t$:
      \[\mathcal{M},\alpha[x\mapsto \llbracket
        t\rrbracket^\mathcal{M}_\alpha]\vDash \phi\]

      \item By the denotation lemma, this gives
        us: \[\mathcal{M},\alpha\vDash(\phi)[x:=t]\] as desired.
        
    \end{itemize}



\end{frame}

\begin{frame}{(11.2.4) Yet Another Locality Lemma}

  \begin{itemize}

  \item \textbf{Lemma} (Locality)\textbf{.} Let $\mathcal{M}$ and
    $\mathcal{N}$ be models with 
              $D^\mathcal{M}=D^\mathcal{N}$. Further, let $\varphi$ be
              a sentence such that for all constants $c$ that occur in
              $\varphi$, $c^\mathcal{M}=c^\mathcal{N}$ and for all
              predicates $R$ that occur in $\varphi$,
              $R^\mathcal{M}=R^\mathcal{N}$. Then for all assignments
              $\alpha$, we have that \[\mathcal{M},\alpha\vDash
                \varphi\text{ iff }\mathcal{N},\alpha\vDash\varphi.\]

              \item The essence of this lemma is that if we changing
                the interpretation of symbols not in the formula,
                doesn't affect the truth-value of the formula.
              
  \end{itemize}
  
\end{frame}

\begin{frame}{Proof of the Locality Lemma}

  \begin{itemize}
  \item We prove the claim by induction.
    \item I do the cases for $R(t_1, \mathellipsis, t_n)$ and $\forall
      x\phi$, the rest is exercise.
      \item For the base case, consider an atomic sentence $R(t_1,
        \mathellipsis, t_n)$.
        \item We know that $\mathcal{M},\alpha\vDash
          R(t_1,\mathellipsis, t_n)$ iff \[(\llbracket
                  t_1\rrbracket_\alpha^\mathcal{M}, \mathellipsis,
                  \llbracket t_n\rrbracket_\alpha^\mathcal{M}) \in
                  R^\mathcal{M}.\]
                  \item But since $R^\mathcal{M}=R^\mathcal{N}$ and
                    $t_i^\mathcal{M}=t_i^\mathcal{N}$, this immediately
                    gives us \[(\llbracket
                  t_1\rrbracket_\alpha^\mathcal{N}, \mathellipsis,
                  \llbracket t_n\rrbracket_\alpha^\mathcal{N}) \in
                  R^\mathcal{N},\] which just means that
                $\mathcal{N},\alpha\vDash R(t_1, \mathellipsis, t_n)$.
  \end{itemize}
  
\end{frame}

\begin{frame}{Proof of Locality (Cont'd)}

  \begin{itemize}
  \item For the induction step, assume the induction hypothesis that
    if $\mathcal{M}$ and 
    $\mathcal{N}$ have the same domain and aggree on the
    interpretation of the non-logical 
    symbols in $\phi$, then $\mathcal{M},\alpha\vDash \phi$ iff
    $\mathcal{N},\alpha\vDash\phi$.

  \item We know that $\mathcal{M},\alpha\vDash\forall x\phi$ iff for
    all $d\in D^\mathcal{M}$, we have that
    \[\mathcal{M},\alpha[x\mapsto d]\vDash\phi.\]

    \item But since $D^\mathcal{M}=D^\mathcal{N}$ by the induction
      hypothesis, this is equivalent to it being the case that  for each $d\in D^\mathcal{N}$, we
      have  \[\mathcal{N},\alpha[x\mapsto d]\vDash\phi.\]

      \item But that just means that $\mathcal{N},\alpha\vDash \forall
        x\phi$, as desired.
    
  \end{itemize}
  
\end{frame}

\subsection{11.3 The Soundness Theorem}

\begin{frame}{11.3 The Soundness Theorem}

  \begin{itemize}
  \item (11.3.1) The idea remains the same as in propositional logic:
    if the premises don't entail the conclusion, the corresponding
    countermodel let's us see that there must be an open branch in the
    corresponding tableau.

  \item (11.3.2) This relies on the down-preservation property:

    \begin{itemize}
    \item If a valuation is faithful to a branch and a rule is
      applied, then the valuation is faithful to at least one new
      branch.
    \item How do we define faithfulness in first-order logic?

      \item A \emph{model} is faithful to a branch $B$ iff for all
        assignments $\alpha$, all formulas on the branch are true.
    \end{itemize}

    \item But, unfortunately, the strict down preservation property
      \emph{fails} in first-order logic:
      \begin{itemize}
      \item If $\exists x\phi$ is on the branch, we get $(\phi)[x:=p]$
        in the branch for some fresh $p$.

      \item But since $p$ wasn't on the branch so far, it's value
        doesn't affect the truth of the formulas on the branch
        (locality).
        \item So, we can have $\exists xP(x)$ is true, but $P(p)$ is false.
      \end{itemize}
  \end{itemize}
  
\end{frame}

\begin{frame}{(11.3.3) The Soundness Lemma}

\textbf{Lemma} (Soundness Lemma)\textbf{.} Let $B$ be a branch of a (possibly incomplete)
                      tableau and suppose further that $\mathcal{M}$
                      is a model that is faithful to $B$. Then, if a
                      rule is applied to $B$, extending the branch to
                      $B'$, then there \emph{exists} a model
                      $\mathcal{N}$ with
                      $D^\mathcal{M}=D^\mathcal{N}$ and
                      $c^\mathcal{M}=c^\mathcal{N}$ for all
                      $c\in\mathcal{C}$ which occur on $B$, such that
                      $\mathcal{N}$ is faithful to $B'$.
      
  
\end{frame}

\begin{frame}{Proof of Soundness Lemma}
  \begin{itemize}
  \item The proof works essentially as in propositional logic: we go
    through the rules one by one.

    \item The propositional rules can be handled just as in
      propositional logic, i.e. we can let $\mathcal{M}=\mathcal{N}$.

    \item The rules for negated quantifiers are easy:
      \begin{center}
                        \begin{prooftree}
                          {
                            line numbering=false,
                            line no sep= 2cm,
                            for tree={s sep'=10mm},
                            single branches=true,
                            close with=\xmark
                          }
                          [\neg \forall x\varphi
                          [\exists x\neg\varphi ]
                          ]
                        \end{prooftree}\hspace{4ex}
                        \begin{prooftree}
                          {
                            line numbering=false,
                            line no sep= 2cm,
                            for tree={s sep'=10mm},
                            single branches=true,
                            close with=\xmark
                          }
                          [\neg \exists x\varphi
                          [\forall x\neg \varphi ]
                          ]
                        \end{prooftree}
                      \end{center}
                     \item  Just note that $\neg\forall
                      x\phi\equi \exists x\neg\phi$ and
                      $\neg\exists x\phi\equi\forall x\neg \phi$.

                    \item So we can let $\mathcal{M}=\mathcal{N}$.

                      \item The interesting cases are the quantifiers.
  \end{itemize}
\end{frame}

\begin{frame}{Soundness ---  Universal Quantifier}

  \begin{itemize}
  \item Suppose that $\mathcal{M}$ is faithful to $B$ and we use


    \begin{center}
                      \begin{prooftree}
                        {
                          line numbering=false,
                          line no sep= 2cm,
                          for tree={s sep'=10mm},
                          single branches=true,
                          close with=\xmark
                        }
                        [\forall x\varphi
                        [{\varphi[x:=a]^\dagger} ]
                        ]
                      \end{prooftree}\quad\raisebox{-3ex}{{\small$\dagger$:
                          $a$ any constant or parameter on the branch}}
                        to extend the branch to $B'$.
                        \end{center}

                        \item Since $\mathcal{M}$ is faithful to $B$
                          and $\forall x\phi\in B$, we have
                          $\mathcal{M},\alpha\vDash\forall x\phi$.

                          \item But we've already observed that
                            $\forall x\phi\vDash (\phi)[x:=t]$, for
                            any ground term $t$, so
                            $\mathcal{M},\alpha\vDash (\phi)[x:=a]$.

                            \item Since $B'=B\cup\{ (\phi)[x:=a]\}$,
                              we get that $\mathcal{M}$ is faithful to
                              $B'$. 
                     
  \end{itemize}
  
\end{frame}

\begin{frame}{Soundness --- Existential Quantifier}

  \begin{itemize}
  \item This is the interesting case.

  \item Suppose $\mathcal{M}$ is faithful to $B$ and we extend $B$ to
    $B'$ using:

    \begin{center}
			\begin{prooftree}
                          {
                            centered,
                            line numbering=false,
                            line no sep= 2cm,
                            for tree={s sep'=10mm},
                            single branches=true,
                            close with=\xmark
                          }
                          [\exists x\varphi
                          [{\varphi[x:=p]^\dagger} ]
                          ]\end{prooftree}
                        \quad\raisebox{-3ex}{{\small$\dagger$: $p$ a fresh parameter}}

                      \end{center}

              \item Since $\mathcal{M}$ is faithful to $B$, we know
                that $\mathcal{M},\alpha\vDash\exists x\phi$.

                \item So, there exists a $d\in D^\mathcal{M}$ such
                  that \[\mathcal{M},\alpha[x\mapsto d]\vDash \phi\]

                  \item So we can change $\mathcal{M}$ to
                    $\mathcal{N}$ by setting $p^\mathcal{N}=d$ and
                    leaving everything else the same.

                    \item By the denotation lemma, we get
                      $\mathcal{N},\alpha\vDash (\phi)[x:=p]$.

                      \item And since $p$ was fresh, for all other
                        formulas in $B$, the condition for the
                        locality lemma are satisfied.

                        \item So, $\mathcal{N},\alpha\vDash \psi$ for
                          all $\psi\in B$.

                          \item Since $B'=B\cup \{(\phi)[x:=p]\}$, we
                            get that $\mathcal{N}$ is faithful to $B'$.
                      
  \end{itemize}
  
\end{frame}

\begin{frame}{(11.3.4) The Soundness Theorem}

\textbf{Theorem} (Soundness)\textbf{.} If $\Gamma\vdash\phi$, then
$\Gamma\vDash\phi$.

\emph{Proof}:

\begin{itemize}
\item By contrapositive proof.
\item Suppose that $\Gamma\nvDash\phi$.
  \item We get a model $\mathcal{M}$, such that all the members of
    $\Gamma$ are true in $\mathcal{M}$ and $\phi$ is false.
  \item So $\mathcal{M}$ is faithful to $\Gamma\cup\{\neg\phi\}$.
    \item Now everytime we apply a rule to construct the tableaux,
      there will be a mmodel $\mathcal{N}$ that is faithful to at
      least one new branch.
      \item So, for the commplete tableau, there is a branch $B$ and
        such a model 
        $\mathcal{N}$ faithful to it.

      \item But that means $B$ can't be closed.
        \item Since if it were, we'd get $R(t_1, \mathellipsis, t_n)$
          both true and false under $\mathcal{N}$.
          \item Hence, $B$ is open, so $\Gamma\nvdash\phi$, as desired.
\end{itemize}
  
\end{frame}

\subsection{11.4 The Completeness Theorem}

\begin{frame}{11.4 The Completeness Theorem}

  \begin{itemize}

    \item The aim is to prove that the associated model
      $\mathcal{M}_B$ for an open branch $B$ in a complete tableau is
      faithful to $B$.

    \item This will give us completeness as in propositional logic.

      \item \textbf{Lemma} (Completeness Lemma)\textbf{.} Let $B$ be an open
      branch in a complete tableau. Then $\mathcal{M}_B$ defined as follows
		\begin{enumerate}[(i)]
		
			\item $D^{\mathcal{M}_B}=\{a\in \mathcal{C}\cup Par: a\text{ occurs on }B\}$
			
			\item $a^{\mathcal{M}_B}=a$ for all $a\in \mathcal{C}\cup Par$
			
			\item $R^{\mathcal{M}_B}=\{(a_1, \mathellipsis, a_n): R(a_1, \mathellipsis, a_n)\in B\}$
		
		\end{enumerate}
                is faithful to $B$.

                \item We prove this by \dots surprise \dots induction.
                
  \end{itemize}
  
\end{frame}

\begin{frame}{Proof of the Completeness Lemma}

  \begin{itemize}
  \item The prove is again similar as in propositional logic,
    i.e. what we actually prove by induction is that for all
    $\phi\in\mathcal{L}$
    \begin{enumerate}[1.]

                      \item if $\phi\in B$, then
                        $\mathcal{M}_B,\alpha\vDash\phi$

                        \item if $\neg\phi\in B$, then
                          $\mathcal{M}_B,\alpha\nvDash\phi$ 
                      
                        \end{enumerate}

             \item I will do the base case and the induction steps for
               $\forall x\phi$ and $\neg\forall x\phi$.

               \item For the base case of 1., suppose that
                 $R(t_1,\mathellipsis, t_n)\in B$.

                 \item We know \[\mathcal{M}_B,\alpha\vDash R(t_1,
                            \mathellipsis, t_n)\text{ iff }(\llbracket
                            t_1\rrbracket^{\mathcal{M}_B}_\alpha,
                            \mathellipsis, \llbracket
                            t_1\rrbracket^{\mathcal{M}_B}_\alpha)\in
                            R^{\mathcal{M}_B}.\]

                            \item By definition of $\mathcal{M}_B$, we
                              know:
                              \begin{itemize}
                              \item $\llbracket t_i\rrbracket^\mathcal{M}_\alpha=t_i^{\mathcal{M}_B}=t_i$
                                \item $R^{\mathcal{M}_B}=\{(t_1, \mathellipsis, t_n): R(t_1, \mathellipsis, t_n)\in B\}$
                                \end{itemize}

                                \item So, since $R(t_1,\mathellipsis,
                                  t_n)\in B$, we get $(t_1,
                                  \mathellipsis, t_n)\in
                                  R^{\mathcal{M}_{B}}$

                                  \item This directly gives us
                                    $\mathcal{M}_B,\alpha\vDash R(t_1,
                                    \mathellipsis, t_n)$.

  \end{itemize}

\end{frame}

\begin{frame}{Completeness Lemma --- Base Case}

  \begin{itemize}

  \item  For the base case of 2., suppose that
                            $\neg R(t_1, \mathellipsis, t_n)\in
                            B$.

                            \item Since $B$ is open, we can conclude
                            that $R(t_1,\mathellipsis, t_n)\notin
                            B$.

                            \item $R^{\mathcal{M}_B}=\{(t_i1,
                              \mathellipsis, t_in): R(t_i1,
                              \mathellipsis, t_in)\in B\}$, it follows
                              that \[(t_i1, \mathellipsis, t_in)\notin
                              R^\mathcal{M}.\]

                              \item Since $\llbracket
                                t_ii\rrbracket^\mathcal{M}_\alpha=t_ii^{\mathcal{M}_B}=t_ii$,
                                we get \[(\llbracket
                                t_ii\rrbracket^\mathcal{M},
                                \mathellipsis, \llbracket
                                t_ii\rrbracket^\mathcal{M})\notin
                                R^{\mathcal{M}_B}.\]

                                \item But that just means that
                                  $\mathcal{M}_B,\alpha\nvDash R(t_i1,
                                  \mathellipsis, t_in)$, as desired.
   
  \end{itemize}


  
\end{frame}

\begin{frame}{Completeness Lemma --- Universal Quantifier}

  \begin{itemize}
  \item For the induction step, assume the hypothesis that for all
    $\alpha$,
     \begin{enumerate}[1.]

                      \item if $\phi\in B$, then
                        $\mathcal{M}_B,\alpha\vDash\phi$

                        \item if $\neg\phi\in B$, then
                          $\mathcal{M}_B,\alpha\nvDash\phi$ 
                      
                        \end{enumerate}

                      \item Now suppose that $\forall x\phi\in B$.

                        \item In order to show that
                          $\mathcal{M}_{B},\alpha\vDash\forall x\phi$, we
                          need to show that for all $d\in
                          D^{\mathcal{M}_B}$, we
                          have \[\mathcal{M}_B,\alpha[x\mapsto
                            d]\vDash \phi\]
                          \item Remember that $D^{\mathcal{M}_B}=\{a:
                            a\text{ occurs in }
                            B\}$, so we need to show \[\mathcal{M}_B,\alpha[x\mapsto
                            a]\vDash \phi\]

                              \item Since the tableau is complete, we
                                know that the universal quantifier
                                rule has been applied.

                                \item It follows that $(\phi)[x:=a]\in
                                  B$.

                                  \item By the induction hypothesis,
                                    we get $\mathcal{M}_{B},\alpha\vDash
                                    (\phi)[x:=a]$.

                                    \item By the denotation lemma, we
                                      get $\mathcal{M}_B,\alpha[x\mapsto
                            a]\vDash \phi$, as desired.
                        
  \end{itemize}
  
\end{frame}

\begin{frame}{The Completeness Lemma --- Negated Universals}

  \begin{itemize}

  \item Now suppose that $\neg\forall x\in B$.

    \item Since the tableau is complete, we know that $\exists
      x\neg\phi$ and $(\neg \phi)[x:=p]$ are on $B$.

    \item We know that $(\neg \phi)[x:=p]=\neg (\phi)[x:=p]$.

      \item So, we can conclude using the induction hypothesis that $\mathcal{M}, \alpha\nvDash
        (\phi)[x:=p]$

        \item By the denotation lemma, we get
          $\mathcal{M},\alpha[x\mapsto \llbracket
          p\rrbracket^\mathcal{M}_\alpha]\nvDash \phi$.

          \item But since $\llbracket
            p\rrbracket^\mathcal{M}_\alpha=p\in D^{\mathcal{M}_B}$,
            this immediately means $\mathcal{M},\alpha\nvDash \forall
            x\phi$, as desired.

            \item This completes our proof. 
    
  \end{itemize}
  
\end{frame}

\begin{frame}{The Completeness Theorem}

  \textbf{Theorem} (Completeness for First-Order Tableaux)\textbf{.}
  If $\Gamma\vDash\phi$, then $\Gamma\vdash\phi$.

  \emph{Proof}:

  \begin{itemize}
  \item We prove the contrapositive.

  \item So suppose that $\Gamma\nvdash\phi$.

    \item This means, by definition, that the complete tableaux for
      $\Gamma\cup\{\neg\phi\}$ has at least one open branch $B$.

      \item By the completeness lemma, we have that $\mathcal{M}_B$ is
        faithful to this branch.

        \item Since $\Gamma\cup\{\neg\phi\}\subseteq B$, we get that
          $\mathcal{M}_B$ makes the $\Gamma$'s true and $\phi$ false.

        \item But that just means $\Gamma\nvDash\phi$.

          \item We're done.
  \end{itemize}

\end{frame}

\begin{frame}

  \vfill

  \begin{center}
  {\Large \textbf{Q.E.D.}}
\end{center}
  
\end{frame}


\begin{frame}{Core Ideas (Lecture Version)}

  \begin{itemize}

  \item (
    
  \end{itemize}

\end{frame}

\begin{frame}

	\begin{center}
	{\huge\bf Thanks!}
	\end{center}

\end{frame}
