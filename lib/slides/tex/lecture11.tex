\begin{frame}
  \setcounter{framenumber}{331}
  \LectureNo{11}
  \maketitle
\end{frame}

\begin{frame}{Overview}
\tableofcontents
\end{frame}

\section{Rehash}
\begin{frame}{Rehash}
	
	\begin{itemize}%[<+->]
	\itemsep=16pt
	
	\item In FOL, an inference is provable by the tableaux algorithm iff its premises and negated conclusion are unsatisfiable.
                  		
	\item An associated model for an FOL talbeau is a \emph{term model}, which uses \emph{syntactic strings} as the objects of its domain.
		
	\item First-order tableaux can be infinite; are not generally decidable.
		
					
	\end{itemize}

\end{frame}
		
\begin{frame}{Today's Goal}

	\begin{itemize}%[<+->]
	\itemsep=16pt
		
        \item Today, we prove soundness and completeness for FOL.
        
        \item From a high level, the proof strategy is familiar---soundness follows from down-preservation, and completeness follows from up-preservation---but the details are \emph{difficult}.

	\item In this lecture, we will only discuss soundness and completeness for the simplest tableaux system (w/o identity and functions). 
          
	\end{itemize}

\end{frame}


\section{Soundness and Completeness for FOL}
\subsection{What is Soundness?}
\begin{frame}{What is Soundness?}

	\begin{itemize}%[<+->]
	\itemsep=16pt
		
        \item The \textbf{Soundness Theorem}: If $\Gamma\vdash\phi$, then $\Gamma\vDash\phi$
	\\
	\alert{\emph{\small{any inference that is provable by tableaux is also semantically valid}}}
	\\
	{\small contrapositive: If $\Gamma\nvDash\phi$ (countermodel), then $\Gamma\nvdash\phi$ (open tableau)}

	\item It represents a sort of `sanity check' on tableau proofs.
	
	\end{itemize}

\end{frame}

\begin{frame}{Proof Sketch}

	\begin{itemize}%[<+->]
	\itemsep=16pt

	\item The hard part is generalizing down-preservation, to prove the \textbf{Soundness Lemma}: 
	If $B$ is a branch of an FOL tableau with faithful model $\mathcal{M}$, when a rule is applied to this branch, there \emph{exists} a model faithful to at least one of its extensions $B'$.

	\item The theorem follows immediately from this lemma.
		\begin{itemize}%[<+->]
		\item If $\Gamma\nvDash\phi$, then this inference has a countermodel $\mathcal{M}$.
		\item $\mathcal{M}$ is faithful to the initial list of the tableau for $\Gamma\cup\{\neg\phi\}$.
		\item From the SL: each time a rule is applied to this tableau, there exists a model faithful to at least one branch.
		\item So, there is a model faithful to a branch of the finished tableau.
		\item This must be an open branch \alert{(why?)}, which shows that $\Gamma\nvdash\phi$.
		\end{itemize}

	\end{itemize}

\end{frame}

\begin{frame}{Tricky Details}

	\begin{itemize}%[<+->]
	\itemsep=16pt
		
	\item The tricky part of this proof concerns the existential quantifier.

	\smallskip

	\begin{center}

			\begin{prooftree}
			{
			line numbering=false,
			line no sep= 2cm,
			for tree={s sep'=10mm},
			single branches=true,
			close with=\xmark
			}
			[\exists x\phi [\phi\lbrack x:\!{=}\, p\rbrack^\dagger ] ]
			\end{prooftree}

	\bigskip
	$\dagger\; p$ any parameter not on the branch

	\end{center}

	\item We cannot prove simple down-preservation here.
		\begin{itemize}%[<+->]	
		\item Assume faithfulness to the top node $\mathcal{M},\alpha\vDash\exists x\phi$.
		\item This implies $\mathcal{M},\alpha[x\mapsto d]\vDash\phi$ for some $d\in\mathcal{D}^\mathcal{M}$.
		\item But since $p$ never occurred up to now, faithfulness `up to now' does not guarantee that $p^\mathcal{M}=d$ as we would `want it to'.
		\item We need to solve this worry to prove the Soundness Lemma.
		\end{itemize}

	\end{itemize}

\end{frame}

\subsection{What is Completeness?}
\begin{frame}{What is Completeness?}

	\begin{itemize}%[<+->]
	\itemsep=16pt
		
	\item The \textbf{Completeness Theorem}: If $\Gamma\vDash\phi$, then $\Gamma\vdash\phi$
	\\
	\alert{\emph{\small{any inference that is semantically valid is also provable by tableaux}}}
	\\
	{\small contrapositive: If $\Gamma\nvdash\phi$ (open tableau), then $\Gamma\nvDash\phi$ (countermodel)}

	\item This ensures our tableau prove absolutely everything we want.

	\end{itemize}

\end{frame}

\begin{frame}{Proof Sketch}

	\begin{itemize}%[<+->]
	\itemsep=16pt
		
	\item The key to this proof is careful inspection of the properties of term models, which support the \textbf{Completeness Lemma}: If $B$ is an open branch of a finished tableau, the associated term model $\mathcal{M}_B$ is faithful to this branch.

	\item The theorem follows immediately from this lemma.
		\begin{itemize}%[<+->]
		\item If $\Gamma\nvdash\phi$, then its finished tableau has an open branch $B$.
		\item From the CL: the term model $\mathcal{M}_B$ is faithful to $B$.
		\item In that case, $\mathcal{M}_B$ is faithful to the initial list $\Gamma\cup\{\neg\phi\}$.
		\item So, there is a model where $\mathcal{M}_B\vDash\Gamma$ and $\mathcal{M}_B\nvDash\phi$.
		\item Which shows what we wanted, namely $\Gamma\nvDash\phi$
		\end{itemize}
                  
	\end{itemize}

\end{frame}
		
\begin{frame}{Tricky Details}

	\begin{itemize}%[<+->]
	\itemsep=16pt
		
	\item The tricky part of this proof concerns the universal quantifier.

	\smallskip

	\begin{center}

			\begin{prooftree}
			{
			line numbering=false,
			line no sep= 2cm,
			for tree={s sep'=10mm},
			single branches=true,
			close with=\xmark
			}
			[\forall x\phi [\phi\lbrack x:\!{=}\, a\rbrack^\dagger ] ]
			\end{prooftree}

	\bigskip
	$\dagger\; a$ any constant or parameter on the branch

	\end{center}

	\item How do we know that up-preservation works here?

		\begin{itemize}%[<+->]
		\item By definition of a finished tableau, we know that if $\forall x\phi$ is on a branch, $\phi[x:=t]$ is on that branch for every term $t\in\mathcal{T}$.
		\item So, by induction hypothesis $\mathcal{M}_B,\alpha\vDash\phi[x:=t]$ for all $t\in\mathcal{T}$.
		\item But how does this syntactic fact translate into the semantic fact we want:
		$\mathcal{M}_B,\alpha[x\mapsto[t]_{\sim B}]\vDash\phi$ for every $[t]_{\sim B}\in\mathcal{D}^{\mathcal{M}_B}$?
		\item We need to solve this worry to prove the Completeness Lemma.
		\end{itemize}

	\end{itemize}

\end{frame}

\subsection{Roadmap}
\begin{frame}{Roadmap}

	\begin{itemize}%[<+->]
	\itemsep=16pt
		
	\item A full proof of the foregoing Soundness and Completeness Lemmas draws on other results we have already read about.
		\begin{itemize}%[<+->]
		\item The Ground Term Lemma (9.2.9)
		\item The Equivalence Class Lemma (10.4.4)
		\item The Denotation Lemma (11.2.2)
		\item The Locality Lemma (11.2.4)
		\end{itemize}
		
	\item The Denotation Lemma is crucial in general.
	Locality gives us Soundness, while the Eq.Class Lemmas gives us Completeness.
	
	\item Let's have a quick glance at two new results on this list.
                  
	\end{itemize}

\end{frame}

\subsection{The Denotation Lemma}
\begin{frame}{The Denotation Lemma}

\begin{itemize}%[<+->]
\itemsep=16pt

\item The Denotation Lemma approximately says that the syntactic operation of substituting for $x$ and the semantic operation of fixing the assignment of $x$ give equivalent results.

\item (11.2.2) \textbf{Lemma} (Denotation Lemma)\textbf{.}
      Consider a formula $\phi$ with precisely one free
            variable, $x$, and an arbitrary ground term $t$. 
            Let $\mathcal{M}$ be a model and
            $\alpha$ an assignment. Then:
            \[\mathcal{M},\alpha\vDash (\phi)[x:=t]\text{ iff
              }\mathcal{M},\alpha[x\mapsto \llbracket
              t\rrbracket^\mathcal{M}_\alpha]\vDash \phi\]
    
\end{itemize}
  
\end{frame}

\begin{frame}{Example Induction Steps for Denotation Lemma}

	\begin{itemize}%[<+->]
	\itemsep=16pt

	\item For the base case, consider atomic $R(t_1, \mathellipsis, t, \mathellipsis, t_n)$.

	\item We have: 
         {\scriptsize \[\mathcal{M},\alpha\vDash (R(t_1, \mathellipsis,x, \mathellipsis, t_n))[x:=t]\text{ iff } (\llbracket t_1\rrbracket^\mathcal{M}_\alpha,\mathellipsis, \llbracket t\rrbracket^\mathcal{M}_\alpha,\mathellipsis,\llbracket t_n\rrbracket^\mathcal{M}_\alpha)\in R^\mathcal{M}\]}
         
         \vspace*{-24pt}

	\item By the Ground Term Lemma, the value of each ground term $t_i$ is unaffected by any change of variable assignment over $x$.

	\item By definition, $\alpha[x\mapsto \llbracket t\rrbracket^\mathcal{M}_\alpha](x)=\llbracket t\rrbracket^\mathcal{M}_\alpha$.

	\item Together, these two observations give us:
	{\scriptsize \[(\llbracket t_1\rrbracket^\mathcal{M}_\alpha,\mathellipsis, \llbracket t\rrbracket^\mathcal{M}_\alpha,\mathellipsis,\llbracket t_n\rrbracket^\mathcal{M}_\alpha)\in R^\mathcal{M} \text{ iff } \mathcal{M},\alpha[x\mapsto \llbracket t\rrbracket^\mathcal{M}_\alpha]\vDash R(t_1, \mathellipsis, x, \mathellipsis, t_n)\]}
               
         \vspace*{-24pt}
            
	\end{itemize}
  
\end{frame}

\begin{frame}{Example Induction Steps for Denotation Lemma}

  \begin{itemize}
  \itemsep=12pt

  \item Now, assuming the claim holds for $\phi$:
    \[\mathcal{M},\alpha\vDash (\phi)[x:=t]\text{ iff
              }\mathcal{M},\alpha[x\mapsto \llbracket
              t\rrbracket^\mathcal{M}_\alpha]\vDash \phi\]

   \item We show that it must then hold for $\forall y\phi$:
            \[\mathcal{M},\alpha\vDash (\forall
              y\phi)[x:=t]\text{ iff }
              \mathcal{M},\alpha[x\mapsto \llbracket
              t\rrbracket^\mathcal{M}_\alpha]\vDash \forall
              y\phi\]

    \item Note, $(\forall y\phi)[x:=t] =\forall y(\phi)[x:=t]$ \, (by def. substitution)

    \item We know that $\mathcal{M},\alpha\vDash \forall y(\phi)[x:=t]$ iff 
      \[\text{for all } d\in D^\mathcal{M},\, \mathcal{M},\alpha[y\mapsto d]\vDash (\phi)[x:=t]\]

    \item By the induction hypothesis, this is equivalent to:
      \[\text{for all } d\in D^\mathcal{M},\, \mathcal{M},\alpha[x\mapsto \llbracket t\rrbracket^\mathcal{M}_\alpha,y\mapsto d]\vDash \phi\]

      \item So we get $\mathcal{M},\alpha[x\mapsto \llbracket
              t\rrbracket^\mathcal{M}_\alpha]\vDash \forall
              y\phi$, as desired.
    
  \end{itemize}
  
\end{frame}

\begin{frame}{Application of the Denotation Lemma}

  \begin{itemize}
  \itemsep=16pt
  
  \item We can use this to show that\, $\forall x\phi\vDash(\phi)[x:=t]$

  \item Suppose $\mathcal{M},\alpha\vDash\forall x\phi$. So, for each $d\in D^\mathcal{M}$, we have
      \[\mathcal{M},\alpha[x\mapsto d]\vDash \phi.\]

    \item Just set the value of $x$ to the value of $t$:
      \[\mathcal{M},\alpha[x\mapsto \llbracket
        t\rrbracket^\mathcal{M}_\alpha]\vDash \phi\]

      \item By the denotation lemma, this gives
        us: \[\mathcal{M},\alpha\vDash(\phi)[x:=t]\]
        
    \end{itemize}

\end{frame}

\subsection{The Locality Lemma}
\begin{frame}{The Locality Lemma}

\begin{itemize}%[<+->]
\itemsep=16pt

              \item The Locality Lemma shows that the interpretation of symbols \emph{not occurring in} a formula have no affect on its truth-value.
  
  \item \textbf{Lemma} (Locality)\textbf{.} Let $\mathcal{M}$ and
    $\mathcal{N}$ be models with 
              $D^\mathcal{M}=D^\mathcal{N}$. Further, let $\phi$ be
              a sentence such that for all constants $c$ that occur in
              $\phi$, $c^\mathcal{M}=c^\mathcal{N}$ and for all
              predicates $R$ that occur in $\phi$,
              $R^\mathcal{M}=R^\mathcal{N}$. Then for all assignments
              $\alpha$, we have that \[\mathcal{M},\alpha\vDash
                \phi\text{ iff }\mathcal{N},\alpha\vDash\phi.\]
            
\end{itemize}
  
\end{frame}

\begin{frame}{Example Induction Steps for Locality Lemma}

  \begin{itemize}
  \itemsep=16pt

	\item For the base case, consider atomic $R(t_1, \mathellipsis, t, \mathellipsis, t_n)$.

        \item We know that $\mathcal{M},\alpha\vDash
          R(a_1,\mathellipsis, a_n)$ iff \[(\llbracket
                  a_1\rrbracket_\alpha^\mathcal{M}, \mathellipsis,
                  \llbracket a_1\rrbracket_\alpha^\mathcal{M}) \in
                  R^\mathcal{M}.\]
                  \item But since $R^\mathcal{M}=R^\mathcal{N}$ and
                    $a_i^\mathcal{M}=a_i^\mathcal{N}$, this immediately
                    gives us \[(\llbracket
                  a_1\rrbracket_\alpha^\mathcal{N}, \mathellipsis,
                  \llbracket a_1\rrbracket_\alpha^\mathcal{N}) \in
                  R^\mathcal{N},\] 
	
	\item Which just means that $\mathcal{N},\alpha\vDash R(t_1, \mathellipsis, t_n)$.
  \end{itemize}
  
\end{frame}

\begin{frame}{Example Induction Steps for Locality Lemma}

  \begin{itemize}
  \itemsep=16pt
  
  \item For the induction step, assume that
    if $\mathcal{M}$ and 
    $\mathcal{N}$ have the same domain and agree on the
    interpretation of non-logical 
    symbols in $\phi$, then $\mathcal{M},\alpha\vDash \phi$ iff
    $\mathcal{N},\alpha\vDash\phi$.

  \item We know that $\mathcal{M},\alpha\vDash\forall x\phi$ iff for
    all $d\in D^\mathcal{M}$, we have that
    \[\mathcal{M},\alpha[x\mapsto d]\vDash\phi.\]

    \item But since $D^\mathcal{M}=D^\mathcal{N}$ by the induction
      hypothesis, this is equivalent to it being the case that  for each $d\in D^\mathcal{N}$, we
      have  \[\mathcal{N},\alpha[x\mapsto d]\vDash\phi.\]

      \item But that just means that $\mathcal{N},\alpha\vDash \forall
        x\phi$, as desired.
    
  \end{itemize}
  
\end{frame}

\begin{frame}{An Important Lesson}

	\begin{itemize}%[<+->]
	\itemsep=16pt
		
	\item What have we learned from the previous results, which solves our worry about proving the Soundness Lemma?
	
	\item The important point: by Denotation and Locality, if a model makes $\phi,\psi,\chi$ true and we change how that model interprets a term \emph{not in those formulas}, then $\phi,\psi,\chi$ will still be true.
	
	\end{itemize} 

\end{frame}

\subsection{Proving the Soundness Lemma}
\begin{frame}{Proving the Soundness Lemma}

Here is a precise formulation of what we now aim to prove.

\bigskip

\textbf{Lemma} (Soundness Lemma)\textbf{.} 
Let $B$ be a branch of a tableau with faithful model $\mathcal{M}$. If a rule is applied to $B$, for at least one of its extensions $B'$, there \emph{exists} model $\mathcal{N}$ with $D^\mathcal{M}=D^\mathcal{N}$ and $c^\mathcal{M}=c^\mathcal{N}$ for all $c\in\mathcal{C}$ which occur on $B$, such that $\mathcal{N}$ is faithful to $B'$.

\end{frame}

\begin{frame}{Proving the Soundness Lemma}

	\begin{itemize}
	\itemsep=16pt
  
	\item Like with propositional logic, we just go through the rules one by one and check that this claim is true for each of them.

	\item It is straightforward for connectives and negated quantifiers.
	\begin{center}
                        \begin{prooftree}
                          {
                            line numbering=false,
                            line no sep= 2cm,
                            for tree={s sep'=10mm},
                            single branches=true,
                            close with=\xmark
                          }
                          [\neg \forall x\varphi
                          [\exists x\neg\varphi ]
                          ]
                        \end{prooftree}\hspace{4ex}
                        \begin{prooftree}
                          {
                            line numbering=false,
                            line no sep= 2cm,
                            for tree={s sep'=10mm},
                            single branches=true,
                            close with=\xmark
                          }
                          [\neg \exists x\varphi
                          [\forall x\neg \varphi ]
                          ]
                        \end{prooftree}
	\end{center}
  
	\item We will just discuss the positive quantifier rules in detail.
	\end{itemize}
	
\end{frame}

\begin{frame}{Proving the Soundness Lemma}

	\begin{itemize}
	\itemsep=16pt
	
	\item Suppose that $\mathcal{M}$ is faithful to $B$ and the `next' rule is the positive universal quantifier rule, which extends to branch $B'$.

	\begin{center}

			\begin{prooftree}
			{
			line numbering=false,
			line no sep= 2cm,
			for tree={s sep'=10mm},
			single branches=true,
			close with=\xmark
			}
			[\forall x\phi [\phi\lbrack x:\!{=}\, a\rbrack^\dagger ] ]
			\end{prooftree}

	\bigskip
	$\dagger\; a$ any constant or parameter on the branch

	\end{center}

                        \item Since $\mathcal{M}$ is faithful to $B$
                          and $\forall x\phi\in B$, we have
                          $\mathcal{M},\alpha\vDash\forall x\phi$.

                          \item We observed that
                            $\forall x\phi\vDash (\phi)[x:=t]$, for
                            any ground term $t$, so
                            $\mathcal{M},\alpha\vDash (\phi)[x:=a]$. 
 
  \end{itemize}
  
\end{frame}

\begin{frame}{Proving the Soundness Lemma}

\medskip

	\begin{itemize}
	\itemsep=10pt
	
	\item Suppose that $\mathcal{M}$ is faithful to $B$ and the `next' rule is the positive existential quantifier rule, which extends to branch $B'$.

	\begin{center}

			\begin{prooftree}
			{
			line numbering=false,
			line no sep= 2cm,
			for tree={s sep'=10mm},
			single branches=true,
			close with=\xmark
			}
			[\exists x\phi [\phi\lbrack x:\!{=}\, p\rbrack^\dagger ] ]
			\end{prooftree}

	\bigskip
	$\dagger\; p$ any parameter not on the branch

	\end{center}
	              \item By faithfulness, $\mathcal{M},\alpha\vDash\exists x\phi$. 
              So, there is $d\in D^\mathcal{M}$ such that \[\mathcal{M},\alpha[x\mapsto d]\vDash \phi\]

                  \item Define $\mathcal{N}$ to be like $\mathcal{M}$ in every way, except $p^\mathcal{N}=d$.

                    \item By the Denotation Lemma, 
                      $\mathcal{N},\alpha\vDash (\phi)[x:=p]$.

                      \item And, since $\mathcal{N}$ at most differs from $\mathcal{M}$ on interpretation of $p$, which was fresh to the tableau, that means $p$ does not occur in prior formulas, so by Locality we have $\mathcal{N},\alpha\vDash \psi$ for all $\psi\in B$.
 
  \end{itemize}
  
\end{frame}

\begin{frame}{The Soundness Theorem, Revisited}

	\begin{itemize}%[<+->]
	\itemsep=16pt

        \item \textbf{Soundness} says that any provable inference---one whose tableau closes---is also a semantically valid inference---it cannot have any countermodel: If $\Gamma\vdash\phi$, then $\Gamma\vDash\phi$
        
        \item Why? Because, as we just showed, an inference with a closed tableau \emph{and} a countermodel is impossible!
        
	\end{itemize}

\end{frame}

\begin{frame}{An Important Lesson}

	\begin{itemize}%[<+->]
	\itemsep=16pt
		
	\item What have we learned from the previous results, which solves our worry about proving the Completeness Lemma?
	
	\item The important point: by Denotation and Eq.Class Lemmas, if $\mathcal{M}_B,\alpha[x\mapsto \llbracket t \rrbracket^{\mathcal{M}_B}_\alpha]\vDash\phi$, then $\mathcal{M}_B,\alpha[x\mapsto[t]_{\sim B}]\vDash\phi$ because a term \emph{denotes itself} (or its equiv-class) in a term model.
	
%	\item We are now ready for the Completeness Lemma.
	
	\end{itemize} 

\end{frame}

\subsection{Proving the Completeness Lemma}
\begin{frame}{Proving the Completeness Lemma}

Here is a precise formulation of what we now aim to prove.

\bigskip

\textbf{Lemma} (Completeness Lemma)\textbf{.} 
Let $B$ be an open branch of a finished tableau.
Then $\mathcal{M}_B$, defined as follows, is faithful to $B$:
\begin{enumerate}[(i)]
\item $D^{\mathcal{M}_B}=\{a\in \mathcal{C}\cup Par: a\text{ occurs on }B\}$
\item $a^{\mathcal{M}_B}=a$ for all $a\in \mathcal{C}\cup Par$
\item $R^{\mathcal{M}_B}=\{(a_1, \mathellipsis, a_n): R(a_1, \mathellipsis, a_n)\in B\}$
\end{enumerate}

\end{frame}

\begin{frame}{Proving the Completeness Lemma}

	\begin{itemize}
	\itemsep=16pt
  
	\item We will sketch here just a few steps of the proof. Similar to propositional logic, we show by induction that for all $\phi\in\mathcal{L}$

    		\begin{enumerate}[1.]

                      \item if $\phi\in B$, then
                        $\mathcal{M}_B,\alpha\vDash\phi$

                        \item if $\neg\phi\in B$, then
                          $\mathcal{M}_B,\alpha\nvDash\phi$ 
                      
		\end{enumerate}

%		\item We will look at the base case and $\forall x\phi$ and $\neg\forall x\phi$ steps.

	\item For the base case of 1., suppose that $R(a_1,\mathellipsis, a_n)\in B$.

	\item By def. of term model $\mathcal{M}_B$, we have $\llbracket a_i\rrbracket^{\mathcal{M}_B}_\alpha=a_i$ for any $\alpha$ and for each ground term $a_i$, as well as $(a_1, \mathellipsis, a_n)\in R^\mathcal{M}$.

	 \item So, we have: $(\llbracket a_1\rrbracket^{\mathcal{M}_B}_\alpha, \mathellipsis, \llbracket a_1\rrbracket^{\mathcal{M}_B}_\alpha)\in R^{\mathcal{M}_B}$
	 
	 \item Thus, $\mathcal{M}_B,\alpha\vDash R(a_1,\mathellipsis, a_n)$.

	\end{itemize}

\end{frame}

\begin{frame}{Proving the Completeness Lemma}

  \begin{itemize}
	\itemsep=16pt

	\item For the base case of 2., suppose that $\neg R(a_1,\mathellipsis, a_n)\in B$.

	\item Since $B$ is open, we have that $R(a_1,\mathellipsis, a_n)\notin B$.

	\item By def. of term model $\mathcal{M}_B$, we have $\llbracket a_i\rrbracket^{\mathcal{M}_B}_\alpha=a_i$ for any $\alpha$ and for each ground term $a_i$, as well as $(a_1, \mathellipsis, a_n)\notin R^\mathcal{M}$.

	 \item So, we have: $(\llbracket a_1\rrbracket^{\mathcal{M}_B}_\alpha, \mathellipsis, \llbracket a_1\rrbracket^{\mathcal{M}_B}_\alpha)\notin R^{\mathcal{M}_B}$
	 
	 \item Thus, $\mathcal{M}_B,\alpha\nvDash R(a_1,\mathellipsis, a_n)$.

	\end{itemize}

\end{frame}

\begin{frame}{Proving the Completeness Lemma}

	\begin{itemize}
	\itemsep=10pt

	\item For induction, assume for $\phi$ and any assignment $\alpha$ we have:
		\begin{enumerate}[1.]

                      \item if $\phi\in B$, then
                        $\mathcal{M}_B,\alpha\vDash\phi$

                        \item if $\neg\phi\in B$, then
                          $\mathcal{M}_B,\alpha\nvDash\phi$ 
                      
		\end{enumerate}

	\item Let $\forall x\phi\in B$. For this to be true, $\phi$ must be satisfied by every object in the domain, which is $D^{\mathcal{M}_B}=\{a: a\text{ occurs in } B\}$.

	\item Since the tableau is finished, the universal quantifier rule was applied to each such term $a$.
	It follows that $(\phi)[x:=a]\in B$.

	\item By induction hypothesis, $\mathcal{M},\alpha\vDash (\phi)[x:=a]$ for each $a$ on $B$.

	\item By the Denotation Lemma, $\mathcal{M}_B,\alpha[x\mapsto a]\vDash \phi$ for each $a$ on $B$, i.e. \emph{each object in} $D^{\mathcal{M}_B}$. 
	Which means that $\mathcal{M}_B,\alpha\vDash\forall x \phi$.
                        
  \end{itemize}
  
\end{frame}

\begin{frame}{Proving the Completeness Lemma}

  \begin{itemize}
	\itemsep=16pt

  \item Now, let $\neg\forall x\in B$. Since the tableau is finished, we know that we also have $\exists x\neg\phi\in B$ and $(\neg \phi)[x:=p]\in B$.

    \item We know that $(\neg \phi)[x:=p]=\neg (\phi)[x:=p]$.

      \item So, by induction hypothesis $\mathcal{M}, \alpha\nvDash
        (\phi)[x:=p]$

        \item By the Denotation Lemma,
          $\mathcal{M},\alpha[x\mapsto \llbracket
          p\rrbracket^\mathcal{M}_\alpha]\nvDash \phi$.

          \item Since $\llbracket
            p\rrbracket^\mathcal{M}_\alpha=p\in D^{\mathcal{M}_B}$,
            this means $\mathcal{M},\alpha\nvDash \forall  x\phi$, as desired.
    
  \end{itemize}
  
\end{frame}

\begin{frame}{The Completeness Theorem, Revisited}

	\begin{itemize}%[<+->]
	\itemsep=16pt

        \item \textbf{Completeness} says that any semantically valid inference---one with no countermodels---is also provable---its finished tableau is closed: If $\Gamma\vDash\phi$, then $\Gamma\vdash\phi$
        
        \item Why? Because, as we just showed, an inference without countermodels \emph{and} a finished open tableau is impossible!
        
	\end{itemize}

\end{frame}

\begin{frame}

\vfill

\begin{center}
{\Large \textbf{Q.E.D.}}
\end{center}
  
\end{frame}
