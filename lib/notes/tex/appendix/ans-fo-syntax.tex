%by Maarten Burger, Alexander Apers, Jos Zuiderwijk, David Bikker

\chapter{Chapter 8. Syntax for First-Order Logic}

\section*{8.9 Exercises}

\begin{enumerate}

  \item[8.9.1]
    \begin{enumerate}
      \item[(i)] \begin{enumerate}
                   \item[(a)]
                     $sub(R(t_1, \dots, t_n)) = \{R(t_1, \dots, t_n)\}$ for all $R^n \in \mathcal{P}$ and \\ $t_1, \dots, t_n \in \mathcal{T}$.
                   \item[(b)]
                     $sub(t_1 = t_2) = \{t_1 = t_2\}$ for all $t_1, t_2 \in \mathcal{T}$.
                 \end{enumerate}
      \item[(ii)] \begin{enumerate}
                    \item[(a)]
                      $sub(\neg \phi) = \{\neg \phi\} \cup sub(\phi)$ for all $\phi \in \mathcal L$.
                    \item[(b)]
                      $sub((\phi \circ \psi)) = \{(\phi \circ \psi)\} \cup sub(\phi) \cup sub(\psi)$ for all $\phi, \psi \in \mathcal L$.
                    \item[(c)]
                      $sub(Qx \phi) = \{Qx\phi\} \cup sub(\phi)$ for all $\phi \in \mathcal L$ and $Q = \forall, \exists$.\\
                  \end{enumerate}
    \end{enumerate}

  \item[8.9.2]
    Suppose $\phi$ is a formula and $x$ is the only free variable in $\phi$, i.e. $x$ is not bound by a quantifier.
    Let's consider $Qx\phi$.
    By definition the root of the corresponding parsing tree is the following occurence: $\langle r, Qx \rangle$.
    Since it's the root, there is a path to the variable $x$.
    Therefore $x$ now is bound by $\langle r, Qx \rangle$.
    Since $x$ was the only free variable, $Qx\phi$ is now closed.\\

  \item[8.9.6]
    No, this is not the case.
    Consider $\exists x P(x)$.
    This is a sentence since all the variables it contains are bound (the $x$ is bound by the $\exists x$).
    The set of sub-formulas is
    $sub(\exists x P(x)) = \{\exists x P(x), P(x)\}$.
    $P(x)$ is a sub-formula, however it is not sentences since it contains an unbound variables, i.e. the $x$.\\


  \item[8.9.9 (i)]
    $$(\forall x(R(x,y) \rightarrow \exists y R(y,y)))[y := x]$$
    $$\forall x((R(x,y) \rightarrow \exists y R(y,y)))[y := x]$$
    $$\forall x((R(x,y))[y := x] \rightarrow (\exists y R(y,y))[y := x])$$
    $$\forall x(R((x)[y := x],(y)[y := x]) \rightarrow \exists y R(y,y))$$
    $$\forall x(R(x,x) \rightarrow \exists y R(y,y))$$

    Vertaalsleutel voor opdracht 8.9.11 t/m 8.9.13: \\
    \begin{tabular}{l|l}
      Denotatie & Betekenis\\
      \hline
      0 & het getal 0 \\
      2 & het getal twee \\
      3 & het getal drie \\
      4 & het getal vier \\
      $E^1(x)$ & $x$ is even \\
      $O^1(x)$ & $x$ is oneven \\
      $N^1(x)$ & ik heb het getal $x$ hier opgeschreven \\
      $G^2(x,y)$ & $x$ is groter dan $y$\\
      $K^2(x,y)$ & $x$ is kleiner dan $y$\\
      $s^2(x,y)$ & De som van $x$ en $y$\\
    \end{tabular}

  \item[8.9.11]
    \begin{enumerate}[(a)]
      \item
        2 is een even getal.\\
        $E(2)$
      \item
        2 is groter dan 3.\\
        $G(2,3)$
      \item
        De som van 2 en 3 is groter dan 4.\\
        $G(s(2,3),4)$
      \item
        Als dit getal groter dan 4 is, dan is het ook groter dan 3.\\
        $G(x,4) \rightarrow G(x,3)$
      \item
        Als dit getal niet groter dan 2 is, dan is het ook niet groter dan 3.\\
        $\neg G(x,2) \rightarrow \neg G(x,3)$
      \item
        Dit getal is kleiner dan 2 of groter dan 4.\\
        $K(x,2) \vee G(x,4)$
    \end{enumerate}
  \item[8.9.12]
    \begin{enumerate}[(a)]
      \item
        Er is een getal groter dan 4 en er is een getal kleiner dan 4.\\
        $(\exists x G(x,4) \land \exists x K(x,4))$
      \item
        Er is een even getal groter dan 3.\\
        $\exists x(E(x) \land G(x,3))$
      \item
        Ieder getal groter dan 4 is ook groter dan 3.\\
        $\forall x (G(x,4) \rightarrow G(x,3))$
      \item
        Geen getal is groter dan 3 en kleiner dan 4.\\
        $\neg \exists x (G(x,3) \land K(x,4)$
      \item
        Als dit getal groter dan 4 is, dan is ieder getal dat ik hier opgeschreven heb groter dan 4.\\
        $G(x,4) \rightarrow \forall x ( N(x) \to G(x,4) )$
      \item
        Een getal dat kleiner dan 3 is, is kleiner dan 4.\\
        $\forall x (K(x,3) \rightarrow K(x,4))$
      \item
        Een getal, dat kleiner dan 3 is, is kleiner dan 4.\\
        $\exists x (K(x,3) \land K(x,4))$
      \item Er is geen getal groter dan 4 en kleiner dan 3.\\
        $\neg \exists x (G(x,4) \land K(x,3))$
    \end{enumerate}
  \item[8.9.13]
    \begin{enumerate}[(a)]
      \item
        Een getal dat groter is dan ieder even getal, is oneven.\\
        $\forall x ( \forall y ( E(y) \rightarrow G(x,y) ) \rightarrow O(x))$
      \item
        Ieder getal is groter dan tenminste één getal.\\
        $\forall x ( \exists y G(x,y) )$
      \item
        Er is een even getal dat kleiner is dan een oneven getal dat groter is dan een oneven getal.\\
        $\exists x ( E(x) \land \exists y ( O(y) \land K(x,y) \land \exists z ( O(z) \land G(y,z) ) ) )$
      \item
        Er is geen getal dat groter is dan ieder getal.\\
        $\neg \exists x ( \forall y G(x,y) )$
      \item
        Geen getal is groter dan zichzelf.\\
        $\neg \exists x G(x,x)$
      \item
        Ieder oneven getal is groter dan 0.\\
        $\forall x ( O(x) \rightarrow G(x,0) )$
      \item
        Ieder oneven getal is groter dan een even getal.\\
        $\forall x ( O(x) \rightarrow \exists y ( E(y) \land G(x,y) )$
    \end{enumerate}
  \item[8.9.14]
    Vertaalsleutel voor opdracht a t/m e: \\
    \begin{tabular}{l|l}
      Denotatie  & Betekenis\\\hline
      m          & mij\\
      $V^1(x)$   & $x$ is verstandig\\
      $B^2(x,y)$ & $x$ bemint $y$
    \end{tabular}
    \begin{enumerate}[(a)]
      \item
        Wie iemand bemint, bemint zichzelf.\\
        $\forall x ( \exists y B(x,y) \rightarrow B(x,x) )$
      \item
        Wie niemand bemint, is niet verstandig.\\
        $\forall x ( \neg \exists y B(x,y) \rightarrow \neg V(x))$
      \item
        Wie verstandig is, wordt door iemand bemind.\\
        $\forall x ( V(x) \to \exists y B(y,x))$
      \item
        Iedereen bemint iemand.\\
        $\forall x \exists y B(x,y)$
      \item
        Wie mij bemint, wordt door mij bemind.\\
        $\forall x ( B(x,m) \to B(m,x))$
        \\\\
        Vertaalsleutel voor opdracht f t/m h: \\
        \begin{tabular}{l|l}
          Denotatie & Betekenis\\
          \hline
          $V^1(x)$ & $x$ is voor mij\\
          $T^1(x)$ & $x$ is tegen mij\\
        \end{tabular}
      \item
        Wie tegen mij is, is niet voor mij.\\
        $\forall x ( T(x) \rightarrow \neg V(x) )$
      \item
        Wie niet voor mij is, is tegen mij.\\
        $\forall x ( \neg V(x) \rightarrow T(x) )$
      \item
        Iedereen is óf voor mij, óf tegen mij.\\
        $\forall x ( ( V(x) \vee T(x) ) \land \neg( V(x) \land T(x) )$
    \end{enumerate}

    \item[8.9.16]

    \begin{enumerate}[(a)]
      \item Er is iemand die sterker is dan Peter.

      \item Als een persoon sterker is dan een ander dan is de ander groter dan deze persoon.

      \item Iemand die groter is dan een ander is ook sterker dan een ander.

      \item Peter is groter dan niemand maar niemand is sterker dan Peter.

      \item Als een persoon blij is dan is er iemand die groter is dan deze persoon.
    \end{enumerate}

\end{enumerate}

%%% Local Variables: 
%%% mode: latex
%%% TeX-master: "../../logic.tex"
%%% End:
