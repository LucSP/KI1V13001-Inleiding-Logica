%by Maarten Burger, Alexander Apers, and Jos Zuiderwijk
\chapter{Chapter 9. Semantics for First-Order Logic}

\section*{9.7 Exercises}

\subsection*{9.7.1}
\begin{enumerate}

\item[(a)] $\llbracket
  x_2\rrbracket^\mathcal{M}_\alpha=\alpha(x_2)=2\cdot 2+1=5$

\item[(b)] $\llbracket S(x_2)\rrbracket^\mathcal{M}_\alpha=S^\mathcal{M}(\llbracket
  x_2\rrbracket^\mathcal{M}_\alpha)=\llbracket
  x_2\rrbracket^\mathcal{M}_\alpha=2\cdot 2+1=5$

\item[(c)] $\llbracket
  (x_1+x_3)\rrbracket^\mathcal{M}_\alpha=+^\mathcal{M}(\llbracket
  x_1\rrbracket^\mathcal{M}_\alpha, \llbracket
  x_3\rrbracket^\mathcal{M}_\alpha)=+^\mathcal{M}(\alpha(x_1),
  \alpha(x_3))=+^\mathcal{M}(2, 7)=2\cdot 7=14.$

\item[(d)] $\llbracket
  S(S(S(x_0)))\rrbracket^\mathcal{M}_\alpha=S^\mathcal{M}(\llbracket
  S(S(x_0))\rrbracket^\mathcal{M}_\alpha=\llbracket
  S(S(x_0))\rrbracket^\mathcal{M}_\alpha=\mathellipsis=\llbracket
  x_0\rrbracket^\mathcal{M}_\alpha=\alpha(x_0)=1$.

\item[(e)] $\llbracket S(0\cdot
  x_1)\rrbracket^\mathcal{M}_\alpha=S^\mathcal{M}(\llbracket 0\cdot
  x_1\rrbracket^\mathcal{M}_\alpha)=\llbracket 0\cdot
  x_1\rrbracket^\mathcal{M}_\alpha=\cdot^\mathcal{M}(\llbracket
  0\rrbracket^\mathcal{M}_\alpha, \llbracket
  x_1\rrbracket^\mathcal{M}_\alpha)=\cdot^\mathcal{M}(0^\mathcal{M},
  \alpha(x_1))=42^3=74088$

\item[(f)] $\llbracket
  2+2\rrbracket^\mathcal{M}_\alpha=+^\mathcal{M}(\llbracket
  2\rrbracket^\mathcal{M},\llbracket
  2\rrbracket^\mathcal{M})=+^\mathcal{M}(\llbracket
  S(S(0))\rrbracket^\mathcal{M}_\alpha, \llbracket
  S(S(0))\rrbracket^\mathcal{M}_\alpha)=+^\mathcal{M}(S^\mathcal{M}(\llbracket
  S(0)\rrbracket^\mathcal{M}_\alpha),S^\mathcal{M}(\llbracket
  S(0)\rrbracket^\mathcal{M}_\alpha))=+^\mathcal{M}(S^\mathcal{M}(\llbracket
  0\rrbracket^\mathcal{M}_\alpha),S^\mathcal{M}(\llbracket
  0\rrbracket^\mathcal{M}_\alpha))=+^\mathcal{M}(42,42)=42\cdot 42=1764$
  
\item[(g)] $\llbracket (x_1 \cdot x_2) +
  x_3\rrbracket_\alpha^\mathcal{M} = \llbracket(3 \cdot
  5)\rrbracket_\alpha^\mathcal{M} +^\mathcal{M}
  \llbracket7\rrbracket_\alpha^\mathcal{M} = (\llbracket
  3\rrbracket_\alpha^\mathcal{M} \cdot^\mathcal{M} \llbracket
  5\rrbracket_\alpha^\mathcal{M}) \cdot 7 = 3^5 \cdot 7 = 1701$

\item[(h)] $\llbracket
    0+0\rrbracket^\mathcal{M}_\alpha=+^\mathcal{M}(\llbracket
    0\rrbracket^\mathcal{M}_\alpha,\llbracket
    0\rrbracket^\mathcal{M}_\alpha)=+^\mathcal{M}(0^\mathcal{M},
    0^\mathcal{M})=42\cdot 42=1764$

\item[(i)] $\llbracket (0\cdot
  0)+1\rrbracket^\mathcal{M}_\alpha=+^\mathcal{M}(\llbracket 0\cdot
  0\rrbracket^\mathcal{M}, \llbracket
  S(0)\rrbracket\rrbracket^\mathcal{M}_\alpha)=+^\mathcal{M}((\cdot^\mathcal{M}(0^\mathcal{M},
  0^\mathcal{M}), S^\mathcal{M}(0^\mathcal{M}))=42^{42}\cdot
  42=42^{43}$

\item[(j)] $\llbracket
  42\rrbracket^\mathcal{M}_\alpha=\llbracket\underbrace{S(\mathellipsis(S}_{42\text{
      times}}(0)\mathellipsis)\rrbracket^\mathcal{M}=S^\mathcal{M}(\llbracket\underbrace{S(\mathellipsis(S}_{41\text{
      times}}(0)\mathellipsis)\rrbracket^\mathcal{M})=\llbracket\underbrace{S(\mathellipsis(S}_{41\text{
      times}}(0)\mathellipsis)\rrbracket^\mathcal{M}=\mathellipsis
    =\llbracket 0\rrbracket^\mathcal{M}_\alpha=0^\mathcal{M}=42$.

\end{enumerate}

\subsection*{9.7.2}

See the added bullet 9.2.10 for (a). (b) and (c) work completely analogously.

\subsection*{9.7.5}
\begin{enumerate}

  \item[(a)] We will first determine
    $\llbracket x \rrbracket ^\mathcal{M}_\alpha$.
    We get:
    $\llbracket x \rrbracket ^\mathcal{M}_\alpha =\alpha(x)= 1$.

    Next, we determine $\llbracket 1 \rrbracket ^\mathcal{M}$.
    We get:
    $\llbracket \underbrace{1}_{=S(0)} \rrbracket ^\mathcal{M} = \llbracket S(0)\rrbracket ^\mathcal{M}_\alpha = S^\mathcal{M}(\llbracket 0 \rrbracket ^\mathcal{M}_\alpha) = S^\mathcal{M}(0) = 0 + 1 = 1$.
    Because $1 = 1$, it follows that $\llbracket x\rrbracket_{\alpha}^{\mathcal{M}}=\llbracket 1\rrbracket_{\alpha}^{\mathcal{M}}$ and so $\mathcal{M},\alpha\vDash x=1$.

\item[(b)]  We want to determine whether
    $\mathcal{M},\alpha\vDash S(x)=S(S(x))$.
    We will first determine
    $\llbracket S(x) \rrbracket ^\mathcal{M}_\alpha $ as follows:
    \[\llbracket S(x) \rrbracket ^\mathcal{M}_\alpha =%
    S^\mathcal{M}(\llbracket x \rrbracket ^\mathcal{M}_\alpha)%
    \underset{\llbracket x\rrbracket_{\alpha}^{\mathcal{M}}=\alpha(x)=1}{=}=%
    S^\mathcal{M}(1) = 1 + 1 = 2\]
    We will then determine $\llbracket S(S(x)) \rrbracket ^\mathcal{M}_\alpha$ as follows:
    \[\llbracket S(S(x)) \rrbracket ^\mathcal{M}_\alpha %
    = S^\mathcal{M}(\llbracket S(x) \rrbracket ^\mathcal{M}_\alpha) %
    = S^\mathcal{M}(S^\mathcal{M}(\llbracket x \rrbracket ^\mathcal{M}_\alpha))\] %
    \[\underset{\llbracket x\rrbracket_{\alpha}^{\mathcal{M}}=1}{=}%
    S^\mathcal{M}(S^\mathcal{M}(1)) = 1 + 1 + 1 = 3.\]
    Because $2 \neq 3$, it follows that
    $\llbracket S(x) \rrbracket ^\mathcal{M}_\alpha \neq \llbracket S(S(x)) \rrbracket ^\mathcal{M}_\alpha$
    and so $\mathcal{M}, \alpha \nvDash x = 1$.

  \item[(c)] $\mathcal{M},\alpha\vDash2+2=4$\\
    We will first determine
    $\llbracket 2 + 2 \rrbracket ^\mathcal{M}_\alpha$
    as follows:
    \[\llbracket 2 + 2 \rrbracket ^\mathcal{M}_\alpha = +^\mathcal{M}(\llbracket 2 \rrbracket ^\mathcal{M},\llbracket 2 \rrbracket ^\mathcal{M})= +^\mathcal{M}(\llbracket S(S(0))\rrbracket ^\mathcal{M}_\alpha, \llbracket S(S(0))\rrbracket ^\mathcal{M}_\alpha) \]
    \[= +^\mathcal{M}(S^\mathcal{M}(\llbracket S(0)\rrbracket ) ^\mathcal{M}_\alpha, S^\mathcal{M}(\llbracket S(0)\rrbracket) ^\mathcal{M}_\alpha)\]
    \[=+^\mathcal{M}(S^\mathcal{M}(S^\mathcal{M}(\llbracket 0 \rrbracket ^\mathcal{M}_\alpha)),S^\mathcal{M}(S^\mathcal{M}(\llbracket 0 \rrbracket ^\mathcal{M}_\alpha)))\]
    \[ = +^\mathcal{M}(S^\mathcal{M}(S^\mathcal{M}(0)),S^\mathcal{M}(S^\mathcal{M}(0)))\]
    \[= +^\mathcal{M}(0 + 1 + 1,0 + 1 + 1) = 2 + 2= 4.\]
    We will then determine
    $\llbracket 4 \rrbracket ^\mathcal{M}_\alpha$
    as follows:
    \[\llbracket 4 \rrbracket ^\mathcal{M}_\alpha = \llbracket S(S(S(S(0))))\rrbracket^\mathcal{M}_\alpha = S^\mathcal{M}(\llbracket S(S(S(0))) \rrbracket ^\mathcal{M}_\alpha ) \]
    \[= S^\mathcal{M}(S^\mathcal{M}(\llbracket S(S(0)) \rrbracket ^\mathcal{M}_\alpha))\]
    \[= S^\mathcal{M}(S^\mathcal{M}(S^\mathcal{M}(\llbracket S(0) \rrbracket ^\mathcal{M}_\alpha )))\]
    \[= S^\mathcal{M}(S^\mathcal{M}(S^\mathcal{M}(S^\mathcal{M}(\llbracket 0 \rrbracket ^\mathcal{M}_\alpha ))))\]
    \[= S^\mathcal{M}(S^\mathcal{M}(S^\mathcal{M}(S^\mathcal{M}(0)))) = 0 + 1 + 1 + 1 + 1 = 4.\]
   
    Because $4 = 4$, it follows that
    $\llbracket 2+2\rrbracket_{\alpha}^{\mathcal{M}}=\llbracket 4\rrbracket^{\mathcal{M}}_{\alpha}$
    and so
    $\mathcal{M},\alpha\vDash2+2=4$.


    \item[(g)] %We will show that the claim holds for every $x,\ y$ by induction over $D^\mathcal{M} = \mathcal{M}athbb{N}$. We need to show that if $S(x) = S(y)$, then $x = y$ \\
    %Base case: Suppose $x=0$. $S(x) = x + 1 = 0 + 1 = 1$. As we assumed $S(x) = S(y), S(y) = 1$. $S(y) = 1 = y + 1$, so $y = 0$, therefore $x = y$.\\
    %Induction step: we assume that $S(n) = S(m) \rightarrow n = m$. We need to show that $S(n+1) = S(m+1) \rightarrow n+1 = m+1$ holds. $S(n+1) = n + 1 + 1 = S(n) + 1$ and $S(m+1)= m +1+1 = S(m) +1$. This can only be the case if $S(m) = S(n)$. But that just means by our induction hypothesis that $n = m$, Then it must also be the case that $n + 1 = m + 1$, which is what we needed to show. $\square$
    We will show that for any $x,\ y \in D^\mathcal{M}$, $S(x) = S(y)
    \rightarrow x = y$ holds. We will use proof by
    contradiction. Assume there is some pair $a,\ b \in D^\mathcal{M}$
    for which the claim doesn't hold. For an implication to be false,
    the premises have to be true, where the conclusion is false. So
    $S(a) = S(b)$ and $ a \neq b$. But $S(a) = S(b)$ means just that
    $a + 1 = b + 1$. As $a \neq b$, we've arrived at the conclusion
    that $1 \neq 1$, which is a contradiction. Hence such a pair $a,\
    b$ cannot exist and therefore the claim is true for all $x,\ y$.
\end{enumerate}
\subsection*{9.7.6}

\begin{enumerate}

  \item[(a)] In order to show that $\mathcal{M},\alpha\vDash \exists
    y~y\in x$, we need to establish (by definition) that there exists
    a $d\in D^\mathcal{M}$ such that $\mathcal{M},\alpha[y\mapsto
    d]\vDash y\in x$. This, in turn, is the case iff $(\llbracket
    y\rrbracket^\mathcal{M}_{\alpha[y\mapsto d}, \llbracket
    x\rrbracket^\mathcal{M}_{\alpha[y\mapsto d})=(d, \alpha(x))=(d,
    \{x:x\text{ is even}\})\in \in ^\mathcal{M}$. Since we have
    $\in^\mathcal{M}=\{(x,X): x\in \mathbb{N}, X\in \wp(\mathbb{N}),
    x\in X\}$, we get that $\mathcal{M},\alpha[y\mapsto
    d]\vDash y\in x$ iff $d\in \{x:x\text{ is even}\}$. So, let
    $d=2$. Clearly, $2\in \{x:x\text{ is even}\}$ and so $\mathcal{M},\alpha[y\mapsto
    2]\vDash y\in x$. So, $\mathcal{M},\alpha\vDash \exists
    y~y\in x$.

  \item[(b)] We will show $\mathcal{M}, \alpha \vDash \forall x \neg x
    \in \emptyset$ holds by using proof by contradiction. Assume there
    is some $x$ such that $x \in \emptyset$ is true in our
    model. $\llbracket x \in \emptyset \rrbracket^\mathcal{M} =
    \llbracket x \rrbracket_\beta \in^\mathcal{M} \llbracket
    \emptyset\rrbracket^\mathcal{M}$. As $\in^\mathcal{M}$ denotes
    just the set theoretical relation $\in$ and
    $\emptyset^\mathcal{M}$ denotes the empty set, we can infer that
    $x \in \emptyset$ is true iff $x$ (whatever $x$ is), is in the
    empty set. We know that the empty set doesn't contain any
    elements, so this cannot be the case. But we assumed that $x \in
    \emptyset$ is true, so we've arrived at a contradiction. Hence
    such an $x$ cannot exist. Therefore, $\forall x \neg x \in
    \emptyset$ is valid in our model.
\end{enumerate}

\subsection*{9.7.10}

\begin{enumerate}
  \item[(i)] We will prove $\mathcal{M} \vDash \exists x \exists y (P(x) \wedge P(y) \wedge x \neq y)$ iff $P^{\mathcal{M}}$ has at least two elements.

    \emph{Left-to right}:
    We will prove this using conditional proof, followed by proof by contradiction.
    We begin with the conditional proof.
    Let $\mathcal{M}$ and  $\alpha$ be arbitrary such that
    $\mathcal{M}, \alpha \vDash \exists x \exists y (P(x) \wedge P(y) \wedge x \neq y)$.
    Now, we assume the negation of our conclusion for the proof by contradiction.
    Thus, we assume $P^\mathcal{M}$ does not have at least two elements.
    This means $P^\mathcal{M}$ has strictly fewer than 2 elements, meaning we can distinguish two cases: $P^\mathcal{M}$ has 0 elements (1) and $P^\mathcal{M}$ has 1 element (2).
    \begin{enumerate}[(1)]
      \item Note that we assumed
        $\mathcal{M}, \alpha \vDash \exists x \exists y (P(x) \wedge P(y) \wedge x \neq y)$.
        This means there has to be a $d \in D^\mathcal{M}$ such that
        $\mathcal{M},\alpha_{[x\mapsto d]}\vDash P(x)$.
        Now, for case (1), we assume $P^\mathcal{M}$ has 0 elements, so $P^\mathcal{M} = \emptyset$.
        Then there is no $d \in D^\mathcal{M}$ such that $d \in P^\mathcal{M}$.
        This is a contradiction to our original assumption.
        It follows that if
        $P^\mathcal{M} = \emptyset$, $\mathcal{M}, \alpha \nvDash \exists x \exists y (P(x) \wedge P(y) \wedge x \neq y)$.

      \item We assume $P^\mathcal{M}$ has 1 element.
        Because we assumed
        \[\mathcal{M}, \alpha \vDash \exists x \exists y (P(x) \wedge P(y) \wedge x \neq y),\]
        there must exist $d \in D^\mathcal{M}$ and $d' \in \mathcal{M}$ such that
        $\mathcal{M}, \alpha_{[x \mapsto d, y \mapsto d']} \vDash P(x) \land P(y) \land x\neq y$.
        It follows that
        $\mathcal{M}, \alpha_{[x \mapsto d, y \mapsto d']} \vDash P(x)$
        and
        $\mathcal{M}, \alpha_{[x \mapsto d, y \mapsto d']} \vDash P(y)$
        and
        $\mathcal{M}, \alpha_{[x \mapsto d, y \mapsto d']} \vDash x \neq y$.
        The latter tells us that $d$ and $d'$ are two distinct elements of $D^\mathcal{M}$.
        But, this is in contradiction to our assumption that $P^\mathcal{M}$ has 1 element.
        Thus, it follows that if $P^\mathcal{M}$ has 1 element,
        $\mathcal{M}, \alpha \nvDash \exists x \exists y (P(x) \wedge P(y) \wedge x \neq y)$.

    \end{enumerate}

    \emph{Right-to-left}:
    We will (again) prove this using conditional proof, followed by proof by contradiction.
    First, for the conditional proof, we assume $P^\mathcal{M}$ has at least two elements.
    Now, for our proof by contradiction, we assume $\mathcal{M}, \alpha \nvDash \exists x \exists y (P(x) \wedge P(y) \wedge x \neq y)$.
    The latter can be rewritten:
    \begin{align*}
        \mathcal{M}, \alpha & \nvDash \exists x \exists y (P(x) \wedge P(y) \wedge x \neq y) \\
        \mathcal{M}, \alpha & \vDash \neg \exists x \exists y (P(x) \wedge P(y) \wedge x \neq y) \\
        \mathcal{M}, \alpha & \vDash \forall x \neg \exists y (P(x) \wedge P(y) \wedge x \neq y) \\
        \mathcal{M}, \alpha & \vDash \forall x \forall y \neg (P(x) \wedge P(y) \wedge x \neq y) \\
        \mathcal{M}, \alpha & \vDash \forall x \forall y (\neg P(x) \vee \neg P(y) \vee x = y) \\
    \end{align*} 
    So, we have
    $\mathcal{M}, \alpha \vDash \forall x \forall y (\neg P(x) \vee \neg P(y) \vee x = y)$
    and we know $P^\mathcal{M}$ has at least two elements.
    We will give a counterexample.
    Assume $p, q \in D^\mathcal{M}$ and $p, q \in P^\mathcal{M}$, where $p \neq q$.
    It follows that, in order for our claim to be valid,
    $\mathcal{M}, \alpha_{[x \mapsto p, y \mapsto q]} \vDash (\neg P(x) \vee \neg P(y) \vee x = y)$
    must hold.
    This means that either
    $\mathcal{M}, \alpha_{[x \mapsto p, y \mapsto q]} \vDash \neg P(x)$,
    or
    $\mathcal{M}, \alpha_{[x \mapsto p, y \mapsto q]} \vDash \neg P(x)$
    or
    $\mathcal{M}, \alpha_{[x \mapsto p, y \mapsto q]} \vDash x = y$
    must be true.
    Let's look at them individually: since we have $p \in P^\mathcal{M}$, we know
    $\mathcal{M}, \alpha_{[x \mapsto p, y \mapsto q]} \vDash \neg P(x)$
    cannot be true.
    The same goes for $q \in P^\mathcal{M}$ and
    $\mathcal{M}, \alpha_{[x \mapsto p, y \mapsto q]} \vDash \neg P(y)$.
    Lastly, we assumed $p \neq q$, so we cannot have
    $\mathcal{M}, \alpha_{[x \mapsto p, y \mapsto q]} \vDash x = y$
    either.
    Thus, we have found a counterexample.
    It follows that if $P^{\mathcal{M}}$ has at least two elements then
    $\mathcal{M} \vDash \exists x \exists y (P(x) \wedge P(y) \wedge x \neq y)$.
    $\qedsymbol$
\end{enumerate}

\subsection*{9.7.13}

\begin{enumerate}

\item[(a)] We want to show that $\forall x P(x)\vDash \forall
  yP(y)$. So, let $\mathcal{M}$ and $\alpha$ be arbitrary such that
  $\mathcal{M}\vDash\forall xP(x)$. This means that for all $d\in
  D^\mathcal{M}$, $\mathcal{M},\alpha[x\mapsto d]\vDash P(x)$,
  i.e. for all $d\in D^\mathcal{M}$, $\llbracket
  x\rrbracket^\mathcal{M}_{\alpha[x\mapsto d]}\in P^\mathcal{M}$. But
  that just means that for all $d\in D^\mathcal{M}$, $d\in
  P^\mathcal{M}$. Hence, we also have that for all $d\in
  D^\mathcal{M}$,   $d=\llbracket
  y\rrbracket^\mathcal{M}_{\alpha[y\mapsto d]}\in P^\mathcal{M}$,
  which gives us that $\mathcal{M},\alpha\vDash\forall y P(y)$.
  
\item[(c)] We will prove the contrapositive. Assume for arbitrary $\mathcal{M}, \alpha$ that $\mathcal{M}, \alpha \vDash \neg \forall x (P(x) \rightarrow Q(x))$. We will show that $\mathcal{M}, \alpha \vDash \neg \neg \exists x P(x)$. As we know $\neg \neg \phi$ is equivalent to $\phi$, showing $\llbracket\exists P(x)\rrbracket_\alpha^\mathcal{M} =1$ is sufficient.\\\textit{Proof:} $\neg \forall x (P(x) \rightarrow Q(x)) $ means just that there is some element $d \in D^\mathcal{M}$ such that $P(d) \rightarrow Q(d)$ is false. This can only be the case if $P(d)$ is true and $Q(d)$ is false. From $\llbracket P(d)\rrbracket_\alpha^\mathcal{M}=1$,\ follows that $\llbracket \exists x P(x)\rrbracket_\alpha^\mathcal{M} =1$, which is what we needed to show. $\square$

\end{enumerate}


    

\subsection*{9.7.14}
\begin{enumerate}
    \item[(a)] Deze claim is niet waar. Dit kan geïllustreerd worden met het onderstaande tegenvoorbeeld:\\
    $ D^\mathcal{M} = \{a,b\} \\
    P^\mathcal{M} = \{a\} \\
    Q^\mathcal{M} = \{a,b\}
    \\ \forall x (P(x)\rightarrow Q(x))$ is waar: als we $x$ vervangen door $a$ dan hebben we dat $P(a)$ en $Q(a)$ waar zijn, dus $P(a)\rightarrow Q(a)$ is ook waar. Als we $x$ vervangen door $b$ krijgen we $\neg P(b)$, dus dan is de implicatie $P(b)\rightarrow Q(b)$ waar. $\neg \exists P(x)$ is ook waar, want $\neg P(b)$ is waar. $\forall x \neg Q(x)$ is echter niet waar, want er is een element waarvoor geldt $Q(X)$, namelijk $a$. 
\end{enumerate}
	
%%% Local Variables: 
%%% mode: latex
%%% TeX-master: "../../logic.tex"
%%% End:
