\chapter{Soundness and Completeness}

\emph{This chapter is even shorter than the soundness and completeness
  chapter in propositional logic. But it's a bit like with strawberries:
  the smaller the sweeter. This chapter marks the end of our
  investigation into logical theory and it contains one of the most
  important results of first-order logic---a milestone achievement of
  the mathematical study of valid reasoning: the completeness
  proof.}

\section{Overview}

	\begin{enumerate}[\thesection.1]

		\item In this chapter, we're going to prove soundness
                  and completeness for first-order tableaux. A
                  daunting task, for sure, but we can do it! The
                  soundness and completeness theorem will round off
                  our treatment of first-order logic. In fact, it will
                  mark the end of our concrete discussion of logical
                  systems. In the following chapter, we'll conclude
                  the course with discussion and outlook. There we
                  will look at theorems we \emph{didn't} prove,
                  systems we \emph{didn't} study, and so on. So,
                  prepare yourself for one last-ditch effort.

                 \item The proofs of the soundness and completeness
                   theorems for first-order logic run very much along
                   the same lines as the proofs for propositional
                   logic. We'll essentially spell out an up
                   preservation lemma and a down preservation lemma,
                   which together guarantee soundness and completeness.
                   
                 \item As you'll see, nothing in the proofs depends in
                   an interesting way on the fact that infinite
                   tableaux exist. In a sense, this shouldn't be very
                   surprising. Infinite tableaux can occur, but as we
                   discussed before, from a mathematical perspective,
                   there is nothing really weird about them. They are
                   tableaux like all the others. We've  already
                   observed that infinite tableaux are always
                   open (a closing tableau closes after finitely many
                   steps), but we can define associated models for
                   infinite branches just like in the finitary
                   case. In fact, the existence of associated models
                   is what ultimately gives us soundness and
                   completeness.

                   \item But before we prove our soundness and
                     completeness lemmas, we'll have to supply some
                     auxilliary lemmas. These two lemmas, the
                     denotation lemma and the locality lemma, are
                     semantic in nature and will figure into our proof
                     that the associated model works in various
                     places. So, let's get cracking.

		
	\end{enumerate}
	
	
\section{The Denotation and Locality Lemma}

	\begin{enumerate}[\thesection.1]

        \item The first lemma that we're going to prove is the
          denotation lemma. Let's first pause for a second and think
          about the content of the lemma (before having stated it
          precisely). The core of the lemma is a kind of equivalence
          between the syntactic method of fixing the reference of
          variables via substitution and the semantic method of
          changing assignments. Taking the simple (open) formula
          $P(x)$ as our example, the idea is as follows. Suppose that
          there's an object $d$ in the domain of a model $\mathcal{M}$, which we can
          refer to, under assignment $\alpha$,  by means of a term
          $t$, i.e. $\llbracket t\rrbracket^\mathcal{M}_\alpha=d$. The
          point of the lemma is that the following two ways of fixing
          the reference of $x$ in $P(x)$ to be $t$  are equivalent:
          \begin{enumerate}[1.]
          \item We can substitute the term $t$ in $P(x)$, i.e. consider
            $(P(x))[x:=t]$ which is just $P(t)$.
            \item We can change the value of $x$ under the assignment
              $\alpha$ to be $\llbracket
              t\rrbracket^\mathcal{M}_\alpha$ and consider $P(x)$
              under the changed assignment $\alpha[x\mapsto \llbracket
              t\rrbracket^\mathcal{M}_\alpha]$.
            \end{enumerate}
            The way in which the two ways are equivalent is that they
            either both lead to a true statement or they both lead to
            a false statement:
            \begin{itemize}
            \item $\mathcal{M},\alpha\vDash (P(x))[x:=t]$ iff
              $\mathcal{M},\alpha[x\mapsto \llbracket
              t\rrbracket^\mathcal{M}_\alpha]\vDash P(x)$
            \end{itemize}
          The denotation lemma generalizes this idea to arbitrary
          formulas.

          \item Here is the precise statement of the denotation lemma
            and its proof:
          
          \begin{lemma}[The Denotation Lemma]
            Consider a formula $\phi$ with precisely one free
            variable, $x$, and an arbitrary ground term $t$. Further,
            let $\mathcal{M}$ be a model and
            $\alpha$ an assignment. Then we have:
            \[\mathcal{M},\alpha\vDash (\phi)[x:=t]\text{ iff
              }\mathcal{M},\alpha[x\mapsto \llbracket
              t\rrbracket^\mathcal{M}_\alpha]\vDash \phi\]
          \end{lemma}

          \begin{proof}

            We prove the claim by induction on formulas.

            \begin{enumerate}[(i)]

              \item  For the base case,
                consider a formula \[R(t_1, \mathellipsis,t,
                  \mathellipsis, t_n),\] which is of the
                form \[(R(t_1, \mathellipsis,x, \mathellipsis,
                  t_n))[x:=t].\] Note that by assumption (that $x$ is
                the only free variable in $\phi$) none of
                $t_1,\mathellipsis, t_n$ contain a variable, i.e. they
                are all ground terms. Now consider the claim
                $\mathcal{M},\alpha\vDash R(t_1, \mathellipsis,t,
                \mathellipsis, t_n)$. By definition, this means that
                $(\llbracket
                t_1\rrbracket^\mathcal{M}_\alpha,\mathellipsis,
                \llbracket
                t\rrbracket^\mathcal{M}_\alpha,\mathellipsis,\llbracket
                t_n\rrbracket^\mathcal{M}_\alpha)\in
                R^\mathcal{M}$. Now consider the formula $R(t_1,
                \mathellipsis, t, \mathellipsis, t_n)$  under the
                assignment $\alpha[x\mapsto \llbracket
                t\rrbracket^\mathcal{M}_\alpha]$. By definition, we
                have that $\mathcal{M}, \alpha[x\mapsto \llbracket
                t\rrbracket^\mathcal{M}_\alpha ] \vDash R(t_1,
                \mathellipsis, t, \mathellipsis, t_n)$ iff
                \[(\llbracket
                t_1\rrbracket^\mathcal{M}_{\alpha[x\mapsto \llbracket
                t\rrbracket^\mathcal{M}_\alpha]},\mathellipsis,
                \llbracket
                t\rrbracket^\mathcal{M}_\alpha,\mathellipsis,\llbracket
                t_n\rrbracket^\mathcal{M}_{\alpha[x\mapsto \llbracket
                t\rrbracket^\mathcal{M}_\alpha]})\in
                R^\mathcal{M}.\] Now, since the terms $t_1,
              \mathellipsis, t_n$ %$
              are all ground terms, by the ground
              term lemma, they all have the same denotation under
              every assignment (cf. 9.2.9). In particular, we get: \[\llbracket
                t_n\rrbracket^\mathcal{M}_{\alpha[x\mapsto \llbracket
                t\rrbracket^\mathcal{M}_\alpha]}=\llbracket
                t_n\rrbracket^\mathcal{M}_{\alpha}.\]
              But that just means that the condition for $\mathcal{M},
              \alpha[x\mapsto \llbracket 
                t\rrbracket^\mathcal{M}_\alpha\vDash R(t_1,
                \mathellipsis, t, \mathellipsis, t_n)$ reduces to
                $(\llbracket
                t_1\rrbracket^\mathcal{M}_\alpha,\mathellipsis,
                \llbracket
                t\rrbracket^\mathcal{M}_\alpha,\mathellipsis,\llbracket
                t_n\rrbracket^\mathcal{M}_\alpha)\in
                R^\mathcal{M}$, which, as we've observed above, is
                precisely the condition for
                $\mathcal{M},\alpha\vDash R(t_1, \mathellipsis,t,
                \mathellipsis, t_n)$. The base case for $t_1=t_2$ is
                completely analogous (in fact, simpler) and is left as
                an exercise. 

                \item We only consider the induction step
                $\forall x\phi$, the rest is left as an
                exercise. So, assume the induction hypothesis that if
                $\phi$ is a formula that contains $x$ as the only free
                variable, then $\mathcal{M},\alpha\vDash (\phi)[x:=t]\text{ iff
              }\mathcal{M},\alpha[x\mapsto \llbracket
              t\rrbracket^\mathcal{M}_\alpha]\vDash \phi$. Now let
              $\forall y\phi$ be a formula with $x$ as its only free
              variable.\footnote{Why don't we write $\forall x\phi$?
                Because in that formula $x$ is not free!} Note that since $x$
              is free in $\forall y\phi$, the variable must 
              also be free in $\phi$. We want to
              show that $\mathcal{M},\alpha\vDash (\forall
              y\phi)[x:=t]\text{ iff }
              \mathcal{M},\alpha[x\mapsto \llbracket
              t\rrbracket^\mathcal{M}_\alpha]\vDash \forall
              y\phi$. So, consider the claim that
              $\mathcal{M},\alpha\vDash (\forall y\phi)[x:=t]$. By the
              recursive definition of substitution, we know
              that $(\forall y\phi)[x:=t]=\forall
              y(\phi[x:=t])$. Hence, we have that $\mathcal{M},\alpha\vDash (\forall
              y\phi)[x:=t]$ iff $\mathcal{M},\alpha\vDash \forall
              y(\phi)[x:=t]$, which is the case iff for all $d\in
              D^\mathcal{M}$, we have that
              $\mathcal{M},\alpha[y\mapsto d]\vDash (\phi)[x:=t]$. But
              by the induction hypothesis, we have that
              $\mathcal{M},\alpha[y\mapsto d]\vDash (\phi)[x:=t]$ iff
              $\mathcal{M},\alpha[y\mapsto d, x\mapsto \llbracket
              t\rrbracket^\mathcal{M}_\alpha]\vDash \phi$. But that just
              means that we have for all $d\in
              D^\mathcal{M}$ that $\mathcal{M},\alpha[y\mapsto d, x\mapsto \llbracket
              t\rrbracket^\mathcal{M}_\alpha]\vDash \phi,$ which by
              definition just means that $\mathcal{M},\alpha[x\mapsto \llbracket
              t\rrbracket^\mathcal{M}_\alpha]\vDash \forall
              y\phi$, as desired.

            \end{enumerate}

          \end{proof}%$

          \item One of the main uses of the denotation lemma is that it
          allows us to derive the quantifier law $\forall x\phi\vDash
          (\phi)[x:=t]$ in a natural way. To see this, suppose that
          $\mathcal{M},\alpha\vDash \forall x\phi$. This means, by
          definition, that for all $d\in D^\mathcal{D},$ we have that
          $\mathcal{M},\alpha[x\mapsto d]\vDash \phi$. So let
          $d=\llbracket t\rrbracket^\mathcal{M}_\alpha$. We get that
          $\mathcal{M},\alpha[x\mapsto \llbracket
          t\rrbracket^\mathcal{M}_\alpha]\vDash \phi$, which by the
          denotation lemma is equivalent to
          $\mathcal{M},\alpha\vDash(\phi)[x:=t]$. This law is one of
          the most natural quantifier laws: it states that if
          everything is such that $\phi$, then $t$ is such that
          $\phi$. But, in order to prove it in full generality, we
          need the denotation lemma. In essence, this will also be the
          role of the denotation lemma in our soundness and
          completeness proof: the lemma allows us to infer the truth
          of instances of a universal generalization from the
          generalization itself.

          \item The second lemma that we're going to prove is (yet)
            another version of the locality lemma (we've proved
            locality lemmas for terms and formulas with respect to
            free variables in \S9). The locality lemma that we need in
            this chapter, is slightly different:

            \begin{lemma}
              Let $\mathcal{M}$ and $\mathcal{N}$ be models with
              $D^\mathcal{M}=D^\mathcal{N}$. Further, let $\varphi$ be
              a sentence such that for all constants $c$ that occur in
              $\varphi$, $c^\mathcal{M}=c^\mathcal{N}$, for all
              function symbols $f$ that occur in $\phi$,
              $f^\mathcal{M}=f^\mathcal{N}$, and for all
              predicates $R$ that occur in $\varphi$,
              $R^\mathcal{M}=R^\mathcal{N}$. Then for all assignments
              $\alpha$, we have that $\mathcal{M},\alpha\vDash \varphi$ iff
              $\mathcal{N},\alpha\vDash\varphi$.
              
            \end{lemma}
            This lemma states, in words, that if two models have the
            same domain and interpret the non-logical symbols in a
            sentence in exactly the same way, then they interpret the
            whole sentence in the same way. Note that since the domains
            of the two models in the lemma are the same, an assignment
            in the one model is also an assignment in the other
            (remember, assignments only depend on domains).

            We shall now prove this lemma:

            \begin{proof}
              We prove this fact by \dots surprise \dots induction. I
              will cover the base case for $R(t_1, \mathellipsis,
              t_n)$ and the case for $\forall x\phi$ and leave the
              remaining cases as exercises.

              \begin{enumerate}

                \item Consider an atomic sentence  $R(t_1,
                  \mathellipsis, t_n)$. We know, by definition, that
                  $\mathcal{M},\alpha\vDash R(t_1, \mathellipsis,
                  t_n)$ iff $(\llbracket
                  t_1\rrbracket_\alpha^\mathcal{M}, \mathellipsis,
                  \llbracket t_1\rrbracket_\alpha^\mathcal{M}) \in
                  R^\mathcal{M}$. First, note that since
                  $c^\mathcal{M}=c^\mathcal{N}$ for all $c\in
                  \mathcal{C}$ and $f^\mathcal{M}=f^\mathcal{N}$ for
                  all $f\in\mathcal{F}$, a simple induction on terms
                  establishes that $\llbracket
                  t\rrbracket^\mathcal{M}_\alpha=\llbracket
                  t\rrbracket^\mathcal{N}_\alpha$ for all terms $t\in
                  \mathcal{T}$ (exercise!).
                  Since additionally  $R^\mathcal{M}=R^\mathcal{N}$,
                  we can conclude that $(\llbracket
                  t_1\rrbracket_\alpha^\mathcal{N}, \mathellipsis,
                  \llbracket t_1\rrbracket_\alpha^\mathcal{N}) \in
                  R^\mathcal{N}$, which is precisely the condition for
                  $\mathcal{N},\alpha\vDash R(t_1, \mathellipsis,
                  t_n)$.

                  \item Next, consider a sentence of the form $\forall
                    x\phi$. Now suppose the induction hypothesis that
                    if $\mathcal{M}$ interprets the non-logical
                    symbols in a sentence $\phi$ exactly like $\mathcal{N}$, then
                    for all assignments $\alpha$, we have that
                    $\mathcal{M},\alpha\vDash\phi$ iff
                    $\mathcal{N},\alpha\vDash\phi$. We want to derive
                    from this assumption that
                    $\mathcal{M},\alpha\vDash\forall x\phi$ iff
                    $\mathcal{N},\alpha\vDash\forall x\phi$. Now, we
                    note that $\mathcal{M},\alpha\vDash\forall x\phi$
                    iff for all $d\in D^\mathcal{M}(=D^\mathcal{N})$,
                    $\mathcal{M},\alpha[x\mapsto d]\vDash\phi$. Now,
                    note that if $\mathcal{M}$ interprets the
                    non-logical symbols in $\forall x\phi$ the same as
                    $\mathcal{N}$, then $\mathcal{M}$ also interprets
                    the non-logical symbols in $\phi$ as
                    $\mathcal{N}$. So, we can infer by the induction
                    hypothesis that for all $d\in D^\mathcal{N}$,
                    $\mathcal{N},\alpha[x\mapsto d]\vDash\phi$. But
                    that is just the condition for
                    $\mathcal{N},\alpha\vDash\forall x\phi$, as desired.

                \end{enumerate}
            \end{proof}
            Now the role of the locality lemma is less intuitive than
            the role of the denotation lemma. But as we'll see in a
            second, it plays a central role in our proof of the
            soundness lemma.
		
	\end{enumerate}

\section{The Soundness Theorem}

	\begin{enumerate}[{\thesection}.1]

		\item We will now go ahead and prove soundness for
                  first-order tableaux. Remember the core
                  concept of our soundness (and completeness) theorem
                  for propositional tableaux was the idea of a
                  faithful valuation, i.e. a valuation that makes all
                  the formulas on a branch true. To obtain soundness,
                  we simply observed that we can use the countermodel
                  for an invalid inference to infer that there must be
                  an open branch in the tableau. Hence the conclusion
                  can't be derived from the premises, meaning
                  contrapositively that if we can derive the
                  conclusion from the premises, they do in fact
                  follow---i.e. soundness. The way we could use the
                  countermodel to obtain our open branch was by means
                  of the central soundness lemma, which states that if
                  we have a faithful valuation and apply a rule, we
                  get a new branch to which our valuation remains
                  faithful---which is just an iterated version of the
                  down preservation principle. Now in first-order
                  tableaux, things aren't \emph{quite} as simple,
                  although the general idea remains the same.

                  \item The
                  concept of a faithful valuation is generalized to
                  first-order logic easy enough: we say that a model
                  $\mathcal{M}$ is \emph{faithful} to a branch iff for all
                  assignments $\alpha$ we have that
                  $\mathcal{M},\alpha\vDash \phi$ for all $\phi\in
                  B$. The major
                  complication that we're facing is that when we apply
                  the rule for the existential quantifier, we're
                  introducing a \emph{new} parameter. And even if
                  model $\mathcal{M}$ was faithful to the branch to
                  begin with,
                  it might now turn out to be unfaithful after the
                  introduction of the new parameter: the model might
                  interpret the parameter in the wrong way.  What will
                  save us is a relaxation of the strict
                  condition that our initial model needs to remain
                  faithful. It turns out that it's enough that we get
                  \emph{a} faithful model everytime we apply a
                  rule. From this, we can infer in the same way that
                  we get an open branch in the complete tableau.

                  \item So, here is the official version of our
                    soundness lemma:

                    \begin{lemma}
                      Let $B$ be a branch of a (possibly incomplete)
                      tableau and suppose further that $\mathcal{M}$
                      is a model that is faithful to $B$. Then, if a
                      rule is applied to $B$, extending the branch to
                      $B'$, then there \emph{exists} a model
                      $\mathcal{N}$ with
                      $D^\mathcal{M}=D^\mathcal{N}$ and
                      $c^\mathcal{M}=c^\mathcal{N}$ for all
                      $c\in\mathcal{C}$ which occur on $B$, such that
                      $\mathcal{N}$ is faithful to $B'$. 
                     
                    \end{lemma}

                    \begin{proof}
                      The proof proceeds, analogously to the proof of the
                      soundness lemma in propositinal logic, by going
                      through the rules one by one. Now, the
                      propositional rules, we can deal with exactly as
                      in propositional logic (replacing the talk of
                      valuations with talk of models) and letting
                      $\mathcal{M}=\mathcal{N}$.

                      The interesting cases are the quantifier rules:

                      The rules for the negated quantifiers are
                      handled quickly:
                      \begin{center}
                        \begin{prooftree}
                          {
                            line numbering=false,
                            line no sep= 2cm,
                            for tree={s sep'=10mm},
                            single branches=true,
                            close with=\xmark
                          }
                          [\neg \forall x\varphi
                          [\exists x\neg\varphi ]
                          ]
                        \end{prooftree}\hspace{4ex}
                        \begin{prooftree}
                          {
                            line numbering=false,
                            line no sep= 2cm,
                            for tree={s sep'=10mm},
                            single branches=true,
                            close with=\xmark
                          }
                          [\neg \exists x\varphi
                          [\forall x\neg \varphi ]
                          ]
                        \end{prooftree}
                      \end{center}
                      We've already observed that $\neg\forall
                      x\phi\equi \exists x\neg\phi$ and that
                      $\neg\exists x\phi\equi\forall x\neg \phi$
                      (since it follows quickly from 9.4.3 iv and
                      v). But that means that in either case, we can
                      just let $\mathcal{M}=\mathcal{N}$.

                      Now for the $\exists x\phi$ rule. Suppose that
                      the last rule that's been applied was:
                      \begin{center}
			\begin{prooftree}
                          {
                            centered,
                            line numbering=false,
                            line no sep= 2cm,
                            for tree={s sep'=10mm},
                            single branches=true,
                            close with=\xmark
                          }
                          [\exists x\varphi
                          [{\varphi[x:=p]^\dagger} ]
                          ]\end{prooftree}
                        \quad\raisebox{-3ex}{{\small$\dagger$: $p$ a fresh parameter}}

                      \end{center}
                    Now, we know that $\mathcal{M}$ is faithful to
                    $B$. Since $\exists x\phi\in B$, we can conclude
                    that $\mathcal{M},\alpha\vDash\exists x \phi$. So,
                    this means that there exists a $d\in
                    D^\mathcal{M}$ such that
                    $\mathcal{M},\alpha[\alpha\mapsto d]\vDash
                    \phi$. Now, we define our model $\mathcal{N}$ just
                    like $\mathcal{M}$, except that we set
                    $p^\mathcal{N}=d$. Now, note that by the
                    denotation lemma, it easily follows that
                    $\mathcal{N},\alpha\vDash
                    (\phi)[x:=p]$. Furthermore, since $p$ has to be
                    new to the branch, there cannot be any formula in
                    $B$ containing $p$. But that means that for each
                    formula in $B$, the condition for the locality
                    lemma is satisfied: the only difference between
                    $\mathcal{M}$ and $\mathcal{N}$ is how we
                    interpret $p$. Hence $\mathcal{N},\alpha\vDash
                    \psi$ for all $\psi\in B$. Since
                    $B'=B\cup\{(\phi)[x:=p]\}$, it follows that
                    $\mathcal{N}$ is faithful to $B'$.


                    Next, consider the case where the last rule has
                    been a case of
                    \begin{center}
                      \begin{prooftree}
                        {
                          line numbering=false,
                          line no sep= 2cm,
                          for tree={s sep'=10mm},
                          single branches=true,
                          close with=\xmark
                        }
                        [\forall x\varphi
                        [{\varphi[x:=a]^\dagger} ]
                        ]
                      \end{prooftree}\quad\raisebox{-3ex}{{\small$\dagger$:
                          $a$ any ground term on the branch}}

                    \end{center}
                    Now, since $\mathcal{M}$ is faithful to $B$ and
                    $\forall x\phi\in B$, we can infer that
                    $\mathcal{M},\alpha\vDash\forall x\phi$. But then
                    it follows immediately via the law $\forall
                    x\phi\vDash (\phi)[x:=t]$ that
                    $\mathcal{M},\alpha\vDash(\phi)[x:=t]$. So, we can
                    simply let $\mathcal{M}=\mathcal{N}$.

                    Finally, we need to consider the case where the
                    last rule was an identity rule. The case where the
                    last rule was the $t=t$ rule, which introduces
                    $t=t$ for every ground term on every branch is
                    easily taken care of since it's easily checked
                    that for each model $\mathcal{M}$ under every
                    assignment $\alpha$, $\mathcal{M},\alpha\vDash
                    t=t$.

                    Slightly more interesting (but not much) is the
                    case of the subsitution rule:
                    \begin{center}
                      \begin{prooftree}
                        {
                          line numbering=false,
                          line no sep= 2cm,
                          for tree={s sep'=10mm},
                          single branches=true,
                          close with=\xmark
                        } 
                        [{s=t}, grouped
                        [{\sigma^\ddagger[x:=s]}, grouped
                        [{\sigma[x:=t]}
                        ]
                        ]
                        ]
                      \end{prooftree}
                    \end{center}
                    $\ddagger$: $\sigma$ here is any atomic formula, i.e. a formula of the form $R(t_1, \mathellipsis, t_n)$ or $t_1=t_2$.

                    Now suppose that $\mathcal{M},\alpha\vDash
                    (\sigma)[x:=s]$. By the denotation lemma, we get
                    that $\mathcal{M},\alpha[x\mapsto \llbracket
                    s\rrbracket^\mathcal{M}_\alpha]\vDash\sigma$. Now
                    since $s=t\in B$ and $\mathcal{M}$ is faithful to
                    $B$, it follows that $\mathcal{M},\alpha\vDash
                    s=t$ and so $\llbracket
                    s\rrbracket^\mathcal{M}_\alpha= \llbracket
                    t\rrbracket^\mathcal{M}_\alpha$. So, we get
                    $\mathcal{M},\alpha[x\mapsto \llbracket 
                    t\rrbracket^\mathcal{M}_\alpha]\vDash\sigma$,
                    which by the denotation lemma gives us $\mathcal{M},\alpha\vDash
                    (\sigma)[x:=t]$, as desired.
                  
                    
                  \end{proof}
                  This completes the proof of our soundness lemma.

                  \item We can infer soundness from this pretty
                    straight-forwardly:

                    \begin{theorem}[Soundness for First-Order
                      Tableaux] If $\Gamma\vdash\phi$, then $\Gamma\vDash\phi$.
                      
                    \end{theorem}
                    \begin{proof}
                      We prove this by contraposition. So, suppose
                      that $\Gamma\nvDash\phi$. This means that there
                      exists a model $\mathcal{M}$  such that
                      $\mathcal{M}\vDash \psi$ for all $\psi\in\Gamma$ but
                      $\mathcal{M}\nvDash\phi$. Now consider
                      the tableau for $\Gamma\cup\{\neg\phi\}$. We
                      immediately have that $\mathcal{M}$ is faithful
                      to the initial list. Now, every time that we
                      apply a rule to construct the complete tableau,
                      we obtain \emph{a} model (possibly
                      different from $\mathcal{M}$) which is
                      faithful to at least one branch of that
                      tree. This means that there exists a model
                      $\mathcal{N}$ which is faithful to some branch
                      in the complete tableau for
                      $\Gamma\cup\{\neg\phi\}$. Now reasoning by
                      contradiction quickly shows that this branch
                      cannot be closed. For suppose that it were. Then
                      there would be an atomic formula and its
                      negation on the branch, which would then both
                      need to be true in $\mathcal{N}$, which is
                      impossible. So, the branch is open, hence the
                      tableau, hence $\Gamma\nvdash\phi$, as desired. 
                    \end{proof}

				
	\end{enumerate}


\section{The Completeness  Theorem}

	\begin{enumerate}[\thesection.1]

		\item We move to completeness. The proof strategy
                  remains the same as in propositional logic: we want
                  to show that the associated model for an open branch
                  in a complete tableaux is indeed faithful to that
                  branch. This is the content of the following
                  completeness lemma:

                  \begin{lemma}[Completeness Lemma]
                    Let $B$ be an open branch in a complete
                    tableau. Then the associated model $\mathcal{M}_B$
                    defined as in 10.3.7 is faithful to $B$. 
                  \end{lemma}
                  \begin{proof}
                    We prove this fact in a similar way as in
                    propositional logic, viz. we prove that for all
                    $\phi\in\mathcal{L}$:
                    \begin{enumerate}[1.]

                      \item if $\phi\in B$, then
                        $\mathcal{M}_B,\alpha\vDash\phi$

                        \item if $\neg\phi\in B$, then
                          $\mathcal{M}_B,\alpha\nvDash\phi$ 
                      
                        \end{enumerate}

                        I will do the base case for $R(t_1,
                        \mathellipsis, t_n)$ and the cases for
                        $\forall x\phi$ and $\neg\forall x\phi$. The
                        cases for $t_1=t_2$ and for $\exists x\phi$ and
                        $\neg\exists x\phi$ are left as \emph{useful} (!)
                        exercises. The propositional cases work just
                        like in propositional logic.

                        \begin{enumerate}[(i)]

                          \item For the base case of 1., suppose that
                            $R(t_1,\mathellipsis, t_n)\in B$. We know
                            that $\mathcal{M}_B,\alpha\vDash R(t_1,
                            \mathellipsis, t_n)$ iff $(\llbracket
                            t_1\rrbracket^{\mathcal{M}_B}_\alpha,
                            \mathellipsis, \llbracket
                            t_1\rrbracket^{\mathcal{M}_B}_\alpha)\in
                            R^{\mathcal{M}_B}$. 
                            By the definition of $\mathcal{M}_B$, we
                            know that
                            $R^{\mathcal{M}_B}=\{([t_1]_{\sim_B},
                            \mathellipsis, [t_n]_{\sim_B}): R(t_1,
                            \mathellipsis, t_n)\in B\}$. And you
                            proved as an exercise in 10.4.4 that
                            $\llbracket
                            t\rrbracket_\alpha^\mathcal{M}=[t]_{\sim_B}$. Putting
                            these two things together, the claim
                            follows immediately. 

                            For the base case of 2., suppose that
                            $\neg R(t_1, \mathellipsis, t_n)\in
                            B$. Since $B$ is open, we can conclude
                            that $R(t_1,\mathellipsis, t_n)\notin
                            B$. Since $R^{\mathcal{M}_B}=\{([t_1]_{\sim_B},
                            \mathellipsis, [t_n]_{\sim_B}): R(t_1,
                            \mathellipsis, t_n)\in B\}$, it follows
                            that \[([t_1]_{\sim_B},
                            \mathellipsis, [t_n]_{\sim_B})\notin
                            R^{\mathcal{M}_B}.\] Since $\llbracket
                            t\rrbracket_\alpha^\mathcal{M}=[t]_{\sim_B}$,
                            we conclude that $(\llbracket
                            t_1\rrbracket^{\mathcal{M}_B}_\alpha,
                            \mathellipsis, \llbracket
                            t_1\rrbracket^{\mathcal{M}_B}_\alpha)\notin
                            R^{\mathcal{M}_B}$, which gives us
                            $\mathcal{M},\alpha\nvDash
                            R(t_1,\mathellipsis, t_n)$ as desired.

                            \item Now, for the induction step, assume
                              the induction hypothesis that for all
                              assignments $\alpha$:

                              \begin{enumerate}[1.]

                              \item if $\phi\in B$, then
                        $\mathcal{M}_B,\alpha\vDash\phi$

                              \item if $\neg\phi\in B$, then
                          $\mathcal{M}_B,\alpha\nvDash\phi$ 
                      
                             \end{enumerate}

                             Now suppose that $\forall x\phi\in B$. In
                             order to show that
                             $\mathcal{M}_B,\alpha\vDash\forall x\phi$
                             as desired, we need to show that for each
                             $d\in D^{\mathcal{M}_B}$, we have that
                             $\mathcal{M}_B,\alpha[x\mapsto
                             d]\vDash\phi$. Remember that
                             $D^{\mathcal{M}_B}=\{[t]_{\sim_B}:t\in\mathcal{T}\}$.
                             %$
                             So, we have to show that for each term
                             $t$, we have $\mathcal{M}_B,\alpha[x\mapsto
                             [t]_{\sim_B}]\vDash\phi$. Now, since
                             $\forall x\phi\in B$ and $B$ is complete,
                             every rule that can be applied has been
                             applied. Since $t=t\in B$ for all
                             $t\in\mathcal{T}$, we can conclude that
                             the $\forall \phi$ rule has been applied
                             for $t$, i.e. for each $t\in\mathcal{T}$,
                             we have $(\phi)[x:=t]\in B$. Hence, by
                             the induction hypothesis,
                             $\mathcal{M},\alpha\vDash(\phi)[x:=t]$. 
                             Note that
                             since we're in tableaux, we can assume
                             that $\forall 
                             x\phi$ is a sentence and so $\phi$
                             contains at most one free variable,
                             $x$. So, by the denotation lemma, we have
                             $\mathcal{M},\alpha[x\mapsto \llbracket
                             t\rrbracket_\alpha^{\mathcal{M}_B}]\vDash
                             \phi$. But, we have already observed that $\llbracket
                            t\rrbracket_\alpha^\mathcal{M}=[t]_{\sim_B}$,
                            so for each $t\in\mathcal{T}$, we get that
                            $\mathcal{M},\alpha[x\mapsto [t]_{\sim_B}]\vDash
                            \phi$, as desired.

                            The remaining case is $\neg\forall
                            x\phi\in B$. Since the tableau is
                            complete, we can conclude that the $\neg
                            \forall x\phi$ rule has been applied,
                            giving us that $\exists x\neg \phi\in
                            B$. Again, since the tableau is complete,
                            we know that the $\exists$ rule has been
                            applied. This means that for some (at some
                            point fresh) parameter $p$, we have
                            $(\neg\phi)[x:=p]\in B$. We know that
                            $(\neg\phi)[x:=p]=\neg (\phi)[x:=t]$.  By
                            the induction hypothesis,
                            we know that $\mathcal{M},\alpha\nvDash
                            (\phi)[x:=t]$. By the denotation lemma, we
                            get $\mathcal{M},\alpha[x\mapsto
                            \llbracket
                            t\rrbracket^{\mathcal{M}_B}_\alpha]\nvDash
                            \phi$. Since $\llbracket
                            t\rrbracket_\alpha^\mathcal{M}=[t]_{\sim_B}$,
                            this gives us for some $[t]_{\sim_B}\in
                            D^{\mathcal{M}_B}$ that $\mathcal{M},\alpha[x\mapsto
                            \llbracket
                            t\rrbracket^{\mathcal{M}_B}_\alpha[t]_{\sim_B}]\nvDash
                            \phi$. From this, it follows immediately
                            that $\mathcal{M},\alpha\nvDash\forall
                            x\phi$, as desired. 

                          \end{enumerate}
                        
                  \end{proof}

                  \item From the completeness lemma, the actual
                    completeness proof follows quickly:

                    \begin{theorem}[Completeness for First-Order
                      Tableaux] If $\Gamma\vDash\phi$, then $\Gamma\vdash\phi$.
                    \end{theorem}
                    \begin{proof}
                      We prove the contrapositive. So, suppose that
                      $\Gamma\nvdash\phi$. By definition, that gives
                      us an open branch in the complete tableau for
                      $\Gamma\cup\{\neg\phi\}$. By the completeness
                      lemma, we get that $\mathcal{M}_B$ is faithfull
                      to at least one branch in this tableau, which,
                      of course, contains
                      $\Gamma\cup\{\neg\phi\}$. Hence, we get that
                      $\mathcal{M}\vDash \psi$, for all
                      $\psi\in\Gamma$ and
                      $\mathcal{M}\nvDash\phi$. So,
                      $\Gamma\nvDash\phi$, as desired,
                    \end{proof}

                    This completes our completeness proof and our
                    investigation into logical theory.
				
	\end{enumerate}        

\section{Core Ideas}

\begin{itemize}
	
		\item The soundness and completeness proof for
                  first-order tableaux are really just a
                  generalization of the propositional case. 

                  \item The devil is in the details.
	
\end{itemize}


\section{Self Study Questions}

Revisit the self study questions of ch. 7.6.

\newpage

\section{Exercises}

\begin{itemize}
\item The exercises for this chapter are different. We're getting
 close to the endterm exam and it's high-time that we think a bit more
 concretely about exam preparation. The exercises for this and next lecture will just be that: exam preparation.

 \item In a moment, I will describe the ``recipe'' for the exam to you
so that, in principle, you can make as many mock exams from the
questions as you'd like. The main purpose of all of this is that you
get a more specific idea of the kind of questions I'll ask in the
exam, their difficulty level, and so on. Fully worked out answers will be
provided at the end of the week. They will come along with a marking
scheme to allow you to gauge my expectations.

\item Here's the exam recipe. The exam will consist of two parts, Part A and Part
  B. The questions in Part A will be questions where I ask you to
  \emph{do} certain things with concrete terms, formulas, models,
  etc. The questions in Part B instead are questions where I ask you
  to \emph{prove} things. Throughout the course, you've been
  doing exercises that prepare you for these two kinds of
  questions. Some more details:

  \begin{itemize}

    \item Below is an example of what Part A might look like. The
      questions in 11.7.1 are roughly of the same length and
      difficulty as the questions in the exam. The questions test
      basic understanding of the core concepts and whether you can
      write up answers in a mathematically appropriate fashion. Note
      that it's not enough to simply write down the result, you need
      to \emph{explain} how you got there, what you did, etc. Check
      Lecture 3 (and possibly the recommended readings there) for
      reference. Note that for each part of logical theory---syntax,
      semantics, proof theory---there is at least one question.

    \item Part B will contain three questions with proofs. There will
      be one question of each difficulty level from 1 to 3. Questions
      can come from any part of logical theory. In answering these
      questions, pay attention to writing proper proofs. Check Lecture
      3! 

    \end{itemize}
  
    \item The two parts of the exam will be weighed in such a way that
      if you answer all the questions in Part A correct,
      you'll pass the exam with a 6.5. The questions in Part B will then
      differentiate your grade. The idea is that to get a 7.5, you
      should get the difficulty 1 proof correct, to get an 8.5, you
      should also get the difficulty 2 proof correct, and to get an 9.5,
      you should get all three proofs essentially correct. A 10 you
      can get if all answers are flawless, including the difficulty 3
      proof. I will not go more into the details of points etc. since
      this encourages the wrong kind of learning: please don't
      calculate ``how much do I need to answer to get this 'n that
      grade''---work hard and try your best!
   
\end{itemize}

\subsection{Part A --- Doing}

\begin{enumerate}[\thesubsection.1]
  
  \item Consider the following formula: \[\forall x\exists
      y(R(x,y)\to \forall x(P(x)\land \exists yR(x,y)))\] Determine
    which variable occurrence is bound by which quantifier occurrence
    in the formula.

    \item Formalize the following sentence as accurately as possible in the language of first-order logic. Use the following translation key:

		\begin{tabular}{l c l l c l}

		$L(x)$ & : & $x$ is a logician \\
		
		$D(x)$ & : & $x$ is an animal lover \\

		$V(x)$ & : & $x$ is friendly \\ $I(x,y)$ & : & $x$ is smarter than $y$ (or as smart as)\\ 

		$k$ & : & Kurt \\ $a$ & : & Ada\\

		\end{tabular}
		
		discussiedomein = de verzameling van alle mensen

		\begin{enumerate}[(i)]

			\item Only logicians are friendly.
			
			\item Logicians aren't smarter than other people.

			\item He's a logician that's not friendly and he's also not an animal lover.
			\item Somebody, who's not a logician, is not an animal lover.

		\end{enumerate}

  \item Take the signature $\mathcal{S}_{PA}=(\{0\},
    \{S,+,\cdot\},\emptyset)$ and the following model $\mathcal{M}$ for it:
    \begin{itemize}
    \item $D^\mathcal{M}=\{n\in\mathbb{N}:n\text{ is odd}\}$
    \item $0^\mathcal{M}=1$
    \item $S^\mathcal{M}(n)=n+2$
    \item $+^\mathcal{M}(n,m)=\begin{cases} n+m & \text{if }
        n+m\text{is odd}\\ n+m+1&\text{otherwise}\end{cases}$
      \item $\cdot^\mathcal{M}(n,m)=n\cdot m$
      
    \end{itemize}
    Further, let $\alpha:\mathcal{V}\to\mathbb{N}$ be an assignment in
    $\mathcal{M}$ such that $\alpha(x)=3$. In this model and under
    this assignment, determine the denotation of the term $(0+x)\cdot S(0)$.

  \item Consider the signature $\mathcal{S}=(\emptyset, \{f^1\},
    \{R^2\})$ and its model $\mathcal{M}$ given by:
    \begin{itemize}

    \item $D^\mathcal{M}=\{1,2\}$
      
    \item $f^\mathcal{M}(1)=2$ and $f^\mathcal{M}(2)=1$

    \item $R^\mathcal{M}=\{(1,2)\}$
      
     \end{itemize}

     Let $\alpha$ be an arbitrary assignment. Determine the
     truth-value of the following formula in this model under that
     assignment: \[\forall x\forall y(R(x,y)\to R(f(y), f(x))\] 

   \item Show the following two derivability facts:
     
     \begin{enumerate}[(a)]

     \item $\exists x(P(x)\land x=c), \forall x(P(x)\to Q(x))\vdash Q(c)$
       
       \item $P(c)\lor (P(c)\land Q(c)), \forall x(Q(x)\to \neg
         P(x))\nvdash \neg P(c)$
         
       \end{enumerate}
      
       In (b), also determine the associated model of at least one
       open branch.


     \item Consider the following inference:
       \begin{itemize}
       \item The ball is round and everything round comes from
         Mars. So, the ball comes from Mars.
       \end{itemize}
       Use the formal methods we developed in this course to determine
       whether the inference is valid.

    
  \end{enumerate}

\subsection{Part B --- Proving}
	
\begin{enumerate}[\thesubsection.1]
  
\item (Difficulty 1) Suppose that $\phi$ is a formula with one free
  variable, $y$. Prove that $\forall x\phi$ is \emph{not} a sentence.
  
\item (Difficulty 1) Provide an argument that for all sets of formulas $\Gamma$,
      we have $\Gamma\vdash P(c)\lor\neg P(c)$.

\item (Difficulty 1) Show that $\forall x(\phi\to \psi)\vDash \neg
  \exists x(\phi\land\neg\psi)$.

  \item (Difficulty 1) Consider the formula $\forall x\exists yR(x,y)$ of an
    appropriate signature. Find a model $\mathcal{M}^+$ (of that
    signature), such that the formula is true in the model (under
    arbitrary assignment) and find a model $\mathcal{M}^-$ such that
    the formula is false in the model.

\item (Difficulty 1) Suppose that a set of formulas $\Gamma$ is
            such that $\Gamma\vDash c\neq c$ for some constant
            $c\in\mathcal{C}$. Prove that $\Gamma$ is unsatisfiable.

\item (Difficulty 2) Let $\mathcal{S}=(\{a\}, \{f^1\},\emptyset)$ be a
  signature, $\mathcal{M}$ a model with $f(d)=d$ for all $d\in
  D^\mathcal{M}$, and $\alpha$ an assignment with
  $\alpha(x)=a^\mathcal{M}$ for all $x\in \mathcal{V}$. Prove by
  induction on terms that for all terms $s,t$, we have $\llbracket s\rrbracket^\mathcal{M}_\alpha=\llbracket t\rrbracket^\mathcal{M}_\alpha$.

\item (Difficulty 2) Suppose that $\phi$ is formula whose only free variable
          is $x$ (which may occur more than once in $\phi$). Show,
          using induction on formulas, that
          $(\phi)[x:=c]$, where $c\in\mathcal{C}$ is a
          constant, is a sentence.

  \item (Difficulty 2) Remember that a binary relation $R$ over a set $X$ is a set
  $R\subseteq X^2$. If $R$ is a binary relation over $X$, we say that $R$ is
  \emph{reflexive} iff for all objects $x\in X$, we have that
  $(x,x)\in R$. Now consider a signature $\mathcal{S}=(\mathcal{C},\mathcal{F},
  \mathcal{R})$ with $R^2\in\mathcal{R}$. Let further
  $\mathcal{M}=(D^\mathcal{M}, \cdot^\mathcal{M})$ be
  a model of our signature and consider $R^\mathcal{M}$. Clearly,
  $R^\mathcal{M}$ is a binary relation over $D^\mathcal{M}$. Prove
  that $R^\mathcal{M}$ is reflexive iff $\mathcal{M},\alpha\vDash
  \forall xR(x,x)$ for all assignments $\alpha$.

\item (Difficulty 2) Let $\mathcal{S}=(\{a,b,c\},
  \emptyset,\{P^1\})$ be a signature and
    $\mathcal{M}=(D^\mathcal{M},\cdot^\mathcal{M})$a model of that
    signature with $D^\mathcal{M}=\{1,2,3\}$ such that $a^\mathcal{M}=1,
    b^\mathcal{M}=2,$ and $c^\mathcal{M}=3$. Prove that
    $\mathcal{M},\alpha\vDash \exists xP(x)\to P(a)\lor
    P(b)\lor P(c)$.

  \item (Difficulty 2) Prove that if $\phi\vdash\psi$ and $\psi\vdash\theta$, then
    $\phi\vdash\theta$.

 \item (Difficulty 3) We say that a set of formulas $\Gamma$ is \emph{semantically
      complete} iff for all formulas $\phi\in\mathcal{L}$, either
    $\Gamma\vDash\phi$ or $\Gamma\vDash\neg\phi$. Suppose that
    $\Gamma\vDash c\neq c$. Show that $\Gamma$ is semantically
    complete.

    \item (Difficulty 3) Recursively define a function
      $c:\mathcal{L}\to\mathbb{N}$ such that $c(\phi)$ measures the
      number of nodes in the parsing tree for $\phi$. Prove that your
      answer works.

\item (Difficulty 3) Somebody says to Bertrand: ``There's a barber that shaves all
  and only those people who don't shave themselves.'' ``No,''
  Bertrand responds, ``that's impossible!'' Show that such a barber
  can indeed not exist (make use of the \emph{formal} methods of the
  course: formalization and either semantics or proof theory).
	
	\end{enumerate}

%%% Local Variables: 
%%% mode: latex
%%% TeX-master: "../../logic.tex"
%%% End:
