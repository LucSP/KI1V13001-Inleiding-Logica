\section*{Your Job Description}

Your primary job as teaching assistants in this course is to help the students understand the material. You know the material (otherwise we wouldn't have selected you as a TA), so on that front, we're good.

But, at the same time, we want to ensure that the students develop healthy study habits: that they learn \emph{how} to learn logic and mathematics.
My impression is that most first-year students are still very much in a high-school mindset:
they prioritize memorization over understanding,
they study (only) for exams (if at all),
they expect exercises/exam questions to be straight-forward applications of known procedures,
\dots.
With this attitude,
they will not succeed at university (or on the job market, for that matter),
so something needs to be done about that.
To help the students out here is your secondary job.

So what are healthy study habits, precisely?
Kevin Houston, in the preface to his fantastic tutorial \emph{How to Think Like a Mathematician} (OUP, ) gives some ``friendly advice'' for students,
which I'm going to repeat here (this is a direct quote from p. x of that book):
		\begin{itemize}
		
			\item \emph{It’s up to you} --- Your actions are likely to be the greatest determiner of the outcome of your studies. Consider the ancient proverb: The teacher can open the door, but you must enter by yourself.
			
			\item  \emph{Be active} --- Read the book. Do the exercises set.

			\item  \emph{Think for yourself} --- Always good advice.

			\item  \emph{Question everything} --- Be sceptical of all results presented to you. Don’t accept them until you are sure you believe them.

			\item  \emph{Observe} --- The power of Sherlock Holmes came not from his deductions but his observations.

			\item  \emph{Prepare to be wrong} --- You will often be told you are wrong when doing mathematics. Don’t despair; mathematics is hard, but the rewards are great. Use it to spur yourself on.

			\item  \emph{Don't memorize} --- seek to understand --- It is easy to remember what you truly understand.
			
			\item  \emph{Develop your intuition} --- But don’t trust it completely.

			\item  \emph{Collaborate} --- Work with others, if you can, to understand the mathematics. This isn't a competition. Don’t merely copy from them though!
			
			\item  \emph{Reflect} --- Look back and see what you have learned. Ask yourself how you could have
done better.

		\end{itemize}

So, how can \emph{you} help students develop healthy study habits?
Analogously to Houston's advice for students, here's my advice for you as teaching assistants:
\begin{itemize}

  \item \emph{Expect preparation} --- If a student asks for help with a problem, ask them what they've tried so far.
	If it becomes evident that they didn't put it enough effort, make this crystal clear to them!

	\emph{Example}:

	\begin{description}
	  \item[S:] Hey, I'm stuck trying to show that $X\subseteq Y$ iff $X\cup Y=Y$.
	  \item[TA:] Ok, what have you done so far?
	  \item[S:] Nothing actually, I'm really stuck.
	  \item[TA:] So, what does $X\subseteq Y$ mean?
	  \item[S:] Hm... no idea!
	\end{description}

	That's an example where the student hasn't put in the work.
	Tell them that they need to check the relevant definitions ($\subseteq, \cup,$ iff).
	Don't help them until they've shown the work.

	\item \emph{Ensure precision} --- Make sure that students use the key concepts in the precise, mathematical sense in which they were defined in class. Convey the (high) standards of precision in mathematical reasoning by correcting even small mistakes.
	
	\item \emph{Be interactive} --- Don't just give (pre-formulated) answers. Try to answer a question with another question to find out where the lack of understanding derives from. Try to figure out the correct answer together.
	
	\item \emph{Refer to the material} --- Ensure that the students read the slides and the lecture notes. For example, when a student doesn't know the definition of a term, ask them where in the notes it's given.
		 
	 \item \emph{Encourage independent thinking} --- Ask the students to come up with their own examples or exercises and go through them together. Don't just rely on the pre-formulated examples and exercises from the notes.
	
	\item \emph{Be demanding} --- Have high standards and expectations: don't ``cuddle'' the students. But reward effort. 
	
	\item \emph{Be human} --- Admit to your own mistakes, everything doesn't need to be perfect. This teaches the students intellectual honesty. When a student points out a mistake, thank them and correct it. When you don't know a definition or the answer to a question, look it up.

\end{itemize}

When in doubt, talk to me!

\section*{Your Responsibilities}

Here are your more hands-on responsibilities:

	\begin{itemize}
	
		\item \emph{Run your workgroup}
		
		Answer questions about the material covered in class,
		go through the solutions to the assigned exercises,
		let students do additional exercises from the book or of your own design.
		
		\item \emph{Give formative feedback on assignments}
		
		Students are asked to submit solutions to the assigned exercises for feedback.
		Give constructive feedback on those, don't assign a grade.
		Make clear that you're not an assessor but an assistant (you don't grade them, you help them understand).
		You can decide whether your students need to submit digitally or on paper.
		% We provide pigeon holes, both digital and brick-and-mortar, for each group (details to follow).
		We don't control whether students submit, but make it clear to them that without your feedback, they're \emph{very} unlikely to succeed.
		
		For the notion of formative feedback, see \url{https://en.wikipedia.org/wiki/Formative_assessment}.
		
		\item \emph{Keep a record of presentations}
		
		Once per term, every student has to give a short presentation on their solution to one of the exercises. Assign a session to each student in the first meeting (it should work out to about 2 students per workgroup meeting, feel free to shift things around). If a student presents, they have to submit their solution to you for feedback at the latest in the workgroup meeting before the one in which they're going to present. Don't grade the assignment or presentation, but give constructive feedback. Keep a record of who presented when and pass it on to us at the end of term. If a student doesn't present, they can't complete the course.

	 	\textbf{Not sure I can have this requirement this year.}
		
		\item \emph{Attend the weekly TA meeting}
		
		We will run a weekly meeting with all of you (there'll be a separate email about that).
		In the meetings, you discuss how the groups are going, any problems that pop up, and possible unclarities about the material (exercises, slides, notes).

		% \item \emph{Write up solutions for the exercises for the lecture notes}

		% I'm writing up new lecture notes for the course, which include my own exercises. I'd like to include solutions to the exercises with some explanations. That's where I'd like to ask your help. In preparation for your workgroups, I'd like to ask you guys to write up solutions to the exercises in such a way that they can be included in the notes. Clarificatory remarks are welcome. I'll double check the solutions before including them. You can do this together or individually, I only need one solution at the time the solutions are being discussed in the workgroup.

	  \item \emph{Improve the lecture notes}

		The lecture notes are still very much a work in progress.
		We need to improve them (typos, solutions to exercises, etc.).
		For this, I'd like to ask your help.
		We'll do this via github, as explained in our first meeting.

	\end{itemize}
	
\section*{Feedback Categories}

Please use the following summative feedback categories for the assignments (in addition to formative feedback you'd like to give). We will also use them in grading, so this will make the procedure more transparent for the students:
		\begin{longtable}{c | l}
		Abbreviation & Mistake \\
		\hline
		\lightning & Error/mistake (generic)\\
		Df. & Incorrect or imprecise definition \\
		Q\textbf{?} & Question not read correctly\\
		$\not\Rightarrow$ & Non-sequitur, reasoning mistake\\
		$\neq$ & Calculation mistake \\
		$\qedsymbol$? & QED missing, reasoning incomplete\\
		\textbf{x}? & Undeclared variables\\
		$\Rightarrow$\textbf{?} & Right-to-left direction missing \\
		$\Leftarrow$\textbf{?} & Left-to-right direction missing \\
		$\underline{\lor}$ & Distinction by cases not exhaustive\\
		$abc$ & Write complete sentences.\\
		 & No paintings!
		\end{longtable}

\noindent Please circle the mistake and put the corresponding symbol on the margins. This only concerns summative feedback (what went wrong). I'll think about whether we standardize formative feedback categories as well (input welcome).

\end{document}
